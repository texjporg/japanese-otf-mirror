% -*- coding: utf-8 -*-
%%%%%%%%
% To control hyperref on command line,
% you can select one of (1),(2a),(2b),(3).
%   (1) do not treat hyperref
%   $ uplatex uotf-sp-utf8.tex
%   (2a) hyperref + dvipdfmx           (with    CMap conversion)
%   $ uplatex "\def\withhyperref{dvipdfmx}\input" uotf-sp-utf8.tex
%   (2b) hyperref + dvipdfmx + out2uni (without CMap conversion)
%   $ uplatex "\def\withhyperref{dvipdfmx}\nocmap{true}\input" uotf-sp-utf8.tex
%   (3) hyperref + dvips + convbkmk.rb + distiller/ps2pdf
%   $ uplatex "\def\withhyperref{dvips}\input" uotf-sp-utf8.tex
%%%%%%

\newif\ifuptexmode\uptexmodefalse
\ifnum\jis"2121="3000
 \uptexmodetrue
 \def\tounicode{pdf:tounicode UTF8-UTF16}
\else
 \ifnum\jis"2121="A1A1
  \def\tounicode{pdf:tounicode EUC-UCS2}
 \fi
 \ifnum\jis"2121="8140
  \def\tounicode{pdf:tounicode 90ms-RKSJ-UCS2}
 \fi
\fi

\makeatletter

\def\@opt@{multi}
\def\@default{default}
\def\@jarticle{jarticle}
\def\@tarticle{tarticle}
\def\@ujarticle{ujarticle}
\def\@noreplace{noreplace}

\ifx\option\@undefined
 \def\option{default}
\fi
\ifx\option\@noreplace
 \ifuptexmode
  \ifx\class\@ujarticle
   \def\@enc@{JY2}\def\@dir@{h}
  \else
   \def\@enc@{JT2}\def\@dir@{v}
  \fi
  \DeclareFontFamily{\@enc@}{mcw}{}
  \DeclareFontFamily{\@enc@}{gtw}{}
  \DeclareFontShape{\@enc@}{mcw}{m}{n}{<->s*[0.962216]upjpnrm-\@dir@}{}
  \DeclareFontShape{\@enc@}{gtw}{m}{n}{<->s*[0.962216]upjpngt-\@dir@}{}
  \DeclareFontShape{\@enc@}{gt}{m}{n}{<->s*[0.962216]upjpngt-\@dir@}{}
  \DeclareFontShape{\@enc@}{mcw}{bx}{n}{<->ssub*gt/m/n}{}
  \DeclareFontShape{\@enc@}{gtw}{bx}{n}{<->ssub*gt/m/n}{}
  \DeclareFontShape{\@enc@}{gt}{bx}{n}{<->ssub*gt/m/n}{}
  \DeclareRobustCommand\mcw{\kanjifamily{mcw}\selectfont}
  \DeclareRobustCommand\gtw{\kanjifamily{gtw}\selectfont}
  \renewcommand\mcdefault{mcw}
  \renewcommand\gtdefault{gtw}
 \fi
\fi
\ifx\option\@default
\else
 \edef\@opt@{\option,\@opt@}
\fi

\ifx\class\@undefined
 \ifuptexmode
  \def\class{ujarticle}
 \else
  \def\class{jarticle}
 \fi
\fi
\ifuptexmode
 \edef\@opt@{uplatex,\@opt@}
\fi
\ifx\class\@jarticle
  \documentclass[a4paper,titlepage]{\class}
\else
 \ifx\class\@ujarticle
  \documentclass[a4paper,titlepage]{\class}
 \else
  \documentclass[a4paper,titlepage,landscape]{\class}
 \fi
\fi

\usepackage[\@opt@]{otf}

\def\@dvipdfmx{dvipdfmx}
\def\@dvips{dvips}

\ifx\withhyperref\@undefined
 \def\withhyperref{undefined}
 \edef\texorpdfstring#1#2{#1}
\else
 \ifx\withhyperref\@dvipdfmx
  \def\@hyperrefkeyval{dvipdfm}
  \usepackage{atbegshi}
  \ifx\nocmap\@undefined
   \AtBeginShipoutFirst{\special{\tounicode}}
  \fi
 \fi
 \ifx\withhyperref\@dvips
  \def\@hyperrefkeyval{dvips}
 \fi

\usepackage[\@hyperrefkeyval,%
bookmarks=true,%
bookmarksnumbered=true,%
bookmarkstype=toc,%
%pdfstartview={FitBH -32768},%
pdftitle={japanese-otfのテスト},%
pdfsubject={Unicode supplementary plane編},%
pdfauthor={upTeXプロジェクト},%
pdfkeywords={TeX; dvips; dvipdfmx; bookmark; hyperref; しおり; pdf}%
]{hyperref}

\fi

\makeatother

\usepackage{redeffont}

\ajUTFVarDef{叱}{20B9F}
\ajUTFVarDef{亭}{20158}
\ajUTFVarDef{吉}{20bb7}
\ajUTFVarDef{座}{2B776}

\AtBeginDvi{\special{papersize=\the\paperwidth,\the\paperheight}}
\pagestyle{empty}
\makeatletter
\ifx\rubyfamily\@undefined\let\rubyfamily=\relax\fi
\ifx\mgfamily\@undefined\let\mgfamily=\relax\fi
\makeatother

\edef\bs{$\backslash$\kern0em}
\setlength\parindent{0em}

\begin{document}
\section{見出し}

コンパイラー:\ifuptexmode upLaTeX\else pLaTeX\fi\\
クラス:\texttt{\class}\\
オプション:\texttt{\option}

\vspace{\baselineskip}
\ifuptexmode
\begin{tabular}{l||ccccccc}
フォント & 仮名 & 漢字 & UTF-8 & \bs kchar & \bs UTF & \bs CID\\
\hline
mc/m & ひらかな & 漢字 & 𠮟𠅘 & \kchar"20B9F\kchar"20158 & \UTF{20B9F}\UTF{20158} & \CID{13803}\CID{20075} \\
mc/bx & {\bfseries ひらかな} & {\bfseries 漢字} & {\bfseries 𠮟𠅘} & {\bfseries\kchar"20B9F\kchar"20158} & {\bfseries\UTF{20B9F}\UTF{20158}} & {\bfseries\CID{13803}\CID{20075}} \\
gt/m & {\gtfamily ひらかな} & {\gtfamily 漢字} & {\gtfamily 𠮟𠅘} & {\gtfamily\kchar"20B9F\kchar"20158} & {\gtfamily\UTF{20B9F}\UTF{20158}} & {\gtfamily\CID{13803}\CID{20075}} \\
gt/bx & {\gtfamily\bfseries ひらかな} & {\gtfamily\bfseries 漢字} & {\gtfamily\bfseries 𠮟𠅘} & {\gtfamily\bfseries\kchar"20B9F\kchar"20158} & {\gtfamily\bfseries\UTF{20B9F}\UTF{20158}} & {\gtfamily\bfseries\CID{13803}\CID{20075}} \\
mg/m & {\mgfamily ひらかな} & {\mgfamily 漢字} & {\mgfamily 𠮟𠅘} & {\mgfamily\kchar"20B9F\kchar"20158} & {\mgfamily\UTF{20B9F}\UTF{20158}} & {\mgfamily\CID{13803}\CID{20075}} \\
\end{tabular}
\else
\begin{tabular}{l||ccccc}
フォント & 仮名 & 漢字 & \bs UTF & \bs CID\\
\hline
mc/m & ひらかな & 漢字 & \UTF{20B9F}\UTF{20158} & \CID{13803}\CID{20075} \\
mc/bx & {\bfseries ひらかな} & {\bfseries 漢字} & {\bfseries\UTF{20B9F}\UTF{20158}} & {\bfseries\CID{13803}\CID{20075}} \\
gt/m & {\gtfamily ひらかな} & {\gtfamily 漢字} & {\gtfamily\UTF{20B9F}\UTF{20158}} & {\gtfamily\CID{13803}\CID{20075}} \\
gt/bx & {\gtfamily\bfseries ひらかな} & {\gtfamily\bfseries 漢字} & {\gtfamily\bfseries\UTF{20B9F}\UTF{20158}} & {\gtfamily\bfseries\CID{13803}\CID{20075}} \\
mg/m & {\mgfamily ひらかな} & {\mgfamily 漢字} & {\mgfamily\UTF{20B9F}\UTF{20158}} & {\mgfamily\CID{13803}\CID{20075}} \\
\end{tabular}
\fi
\vspace{\baselineskip}

日本:\UTF{20509}\UTF{241FE} 簡体字:\UTFC{20509}\UTFC{241FE}  多言語:\UTFM{20509}\UTFM{241FE}

日本:\UTF{20b9f}\UTF{26402} 繁體字:\UTFT{20b9f}\UTFT{26402}  多言語:\UTFM{20b9f}\UTFM{26402}

簡体字:\UTFC{20087}\UTFC{200cc} 繁體字:\UTFT{20087}\UTFT{200cc}  多言語:\UTFM{20087}\UTFM{200cc}

\vspace{\baselineskip}

\ifuptexmode
 \kchar"20B9Fる。
 𠮟る。
\fi
\ajVar{叱}る。
叱る。

\ifuptexmode
 らいおん\kchar"20158。
 らいおん𠅘。
\fi
らいおん\ajVar{亭}。
らいおん亭。

\ifuptexmode
 \kchar"20BB7野家。
 𠮷野家。
\fi
\ajVar{吉}野家。
吉野家。

\ifuptexmode
 銀\kchar"2B776アスター。
 銀𫝶アスター。
\fi
銀\ajVar{座}アスター。
銀座アスター。

\makeatletter
\ifx\withhyperref\@undefined
\else

\section{見出しに\texorpdfstring{\bs}{\134}UTF, \texorpdfstring{\bs}{\134}UTFC, \texorpdfstring{\bs}{\134}UTFMなど}
\subsection{日本:\UTF{9aa8}\UTF{6D77} 簡体字:\UTFC{9aa8}\UTFC{6D77} 繁體字:\UTFT{9AA8}\UTFT{6d77} 朝鮮:\UTFK{9AA8}\UTFK{6d77}}
日本:\UTF{9aa8}\UTF{6D77} 簡体字:\UTFC{9aa8}\UTFC{6D77} 繁體字:\UTFT{9AA8}\UTFT{6d77} 朝鮮:\UTFK{9AA8}\UTFK{6d77}

\subsection{ハングル:\UTFK{c548}\UTFK{b155}\UTFK{d558}\UTFK{C138}\UTFK{C694}}
ハングル:\UTFK{c548}\UTFK{b155}\UTFK{d558}\UTFK{C138}\UTFK{C694}

\subsection{日本:\UTF{20509}\UTF{241FE} 簡体字:\UTFC{20509}\UTFC{241FE}  多言語:\UTFM{20509}\UTFM{241FE}}
日本:\UTF{20509}\UTF{241FE} 簡体字:\UTFC{20509}\UTFC{241FE}  多言語:\UTFM{20509}\UTFM{241FE}

\subsection{日本:\UTF{20509}\UTF{241FE} 簡体字:\UTFC{20509}\UTFC{241FE}  多言語:\UTFM{20509}\UTFM{241FE}}
日本:\UTF{20509}\UTF{241FE} 簡体字:\UTFC{20509}\UTFC{241FE}  多言語:\UTFM{20509}\UTFM{241FE}

\subsection{日本:\UTF{20b9f}\UTF{26402} 繁體字:\UTFT{20b9f}\UTFT{26402}  多言語:\UTFM{20b9f}\UTFM{26402}}
日本:\UTF{20b9f}\UTF{26402} 繁體字:\UTFT{20b9f}\UTFT{26402}  多言語:\UTFM{20b9f}\UTFM{26402}

\subsection{簡体字:\UTFC{20087}\UTFC{200cc} 繁體字:\UTFT{20087}\UTFT{200cc}  多言語:\UTFM{20087}\UTFM{200cc}}
簡体字:\UTFC{20087}\UTFC{200cc} 繁體字:\UTFT{20087}\UTFT{200cc}  多言語:\UTFM{20087}\UTFM{200cc}
\fi
\makeatother

\clearpage
[mc/m]

\ifuptexmode
 %
% This file is generated from the data of UniJIS-UTF32
% in cid2code.txt (Version 02/04/2012)
% for Adobe-Japan1-6
%
% Reference:
%   http://sourceforge.net/adobe/cmap/home/Home/
%   cmapresources_japan1-6.tar.z
%
% A newer CMap may be required for some code points.
%


Adobe-Japan1-0\\
𨳝櫛𥡴𨻶杓巽屠兔冕冤
𡨚𤏐爨🄀

Adobe-Japan1-4\\
🄐🄑🄒🄓🄔🄕🄖🄗🄘🄙
🄚🄛🄜🄝🄞🄟🄠🄡🄢🄣
🄤🄥🄦🄧🄨🄩🅐🅑🅒🅓
🅔🅕🅖🅗🅘🅙🅚🅛🅜🅝
🅞🅟🅠🅡🅢🅣🅤🅥🅦🅧
🅨🅩🄰🄱🄲🄳🄴🄵🄶🄷
🄸🄹🄺🄻🄼🄽🄾🄿🅀🅁
🅂🅃🅄🅅🅆🅇🅈🅉🈂🈷
🅰🅱🅲🅳🅴🅵🅶🅷🅸🅹
🅺🅻🅼🅽🅾🅿🆀🆁🆂🆃
🆄🆅🆆🆇🆈🆉眞𠤎𦥑𫟘
沿芽槪割𦈢𠮷𩵋卿𫞎憲
𠩤浩𫝆𫝷滋𠮟勺爵周将
𠀋城𩙿真𠆢𫝑成𧾷𣳾炭
𥫗彫潮𡈽冬𤴔姬𫞉諭輸
𥙿𦚰𠘨𠂊𠦄卉寃拔𦦙𣏌
杞𪧦𫞽絣𠔿𦉪𠂰𨦇𨸗𫠚
𤋮桒𣲾𠘑嶲你𣘺𣏾𢘉

Adobe-Japan1-5\\
𡌛𡑮𡢽𡚴𡸴𣇄𣗄𣜿𣝣𤟱
𥒎𥔎𥝱𥧄𥶡𦫿𦹀𧃴𧚄𨉷
𨏍𪆐𠂉𠂢𠂤𠈓𠌫𠎁𠍱𠏹
𠑊𠔉𠗖𠝏𠠇𠠺𠢹𠥼𠦝𠫓
𠬝𠵅𠷡𠺕𠹭𠹤𠽟𡈁𡉕𡉻
𡉴𡋤𡋗𡋽𡌶𡍄𡏄𡑭𡗗𦰩
𡙇𡜆𡝂𡧃𡱖𡴭𡵅𡵸𡵢𡶡
𡶜𡶒𡶷𡷠𡸳𡼞𡽶𡿺𢅻𢌞
𢎭𢛳𢡛𢢫𢦏𢪸𢭏𢭐𢭆𢰝
𢮦𢰤𢷡𣇃𣇵𣆶𣍲𣏓𣏒𣏐
𣏤𣏕𣏚𣏟𣑊𣑑𣑋𣑥𣓤𣕚
𣖔𣘹𣙇𣘸𣜜𣜌𣝤𣟿𣟧𣠤
𣠽𣪘𣱿𣴀𣵀𣷺𣷹𣷓𣽾𤂖
𤄃𤇆𤇾𤎼𤘩𤚥𤢖𤩍𤭖𤭯
𤰖𤸎𤸷𤹪𤺋𥁊𥁕𥄢𥆩𥇥
𥇍𥈞𥉌𥐮𥓙𥖧𥞩𥞴𥧔𥫤
𥫣𥫱𥮲𥱋𥱤𥸮𥹖𥹥𥹢𥻘
𥻂𥻨𥼣𥽜𥿠𥿔𦀌𥿻𦀗𦁠
𦃭𦉰𦊆𣴎𦐂𦙾𦜝𦣝𦣪𦥯
𦧝𦨞𦩘𦪌𦪷𦱳𦳝𦹥𦾔𦿸
𦿶𦿷𧄍𧄹𧏛𧏚𧏾𧐐𧑉𧘕
𧘔𧘱𧚓𧜎𧜣𧝒𧦅𧪄𧮳𧮾
𧯇𧲸𧶠𧸐𨂊𨂻𨊂𨋳𨐌𨑕
𨕫𨗈𨗉𨛗𨛺𨥉𨥆𨥫𨦈𨦺
𨦻𨨞𨨩𨩱𨩃𨪙𨫍𨫤𨫝𨯁
𨯯𨴐𨵱𨷻𨸟𨸶𨺉𨻫𨼲𨿸
𩊠𩊱𩒐𩗏𩛰𩜙𩝐𩣆𩩲𩷛
𩸕𩺊𩹉𩻄𩻩𩻛𩿎𩿗𪀯𪀚
𪃹𪂂𢈘𪎌𪐷𪗱𪘂𪚲𠃵𤸄
𤿲𧵳再善形慈栟軔𪊲𠅘
𠖱𠛬𫝓𠵘𫝚𫝜𥧌𫝶𢹂𫝼
𠟈𢿫𧦴𫞂𫞋𣟱𫞔𤁋𫞬𫞯
𫟉𫟏𫟒𦲞𧰼𫟰𫝥𫠍𫠗𦍌
𩸽𪘚

% end

\fi
%
% This file is generated from the data of UniJIS-UTF32
% in cid2code.txt (Version 07/30/2018)
% for Adobe-Japan1-7
%
% Reference:
%   https://github.com/adobe-type-tools/cmap-resources/
%   Adobe-Japan1-7/cid2code.txt
%
% A newer CMap may be required for some code points.
%


Adobe-Japan1-0\\
\UTF{28CDD}\UTF{2F8ED}\UTF{25874}\UTF{28EF6}\UTF{2F8DC}\UTF{2F884}\UTF{2F877}\UTF{2F80F}\UTF{2F8D3}\UTF{2F818}%
\UTF{21A1A}\UTF{243D0}\UTF{2F920}\UTF{1F100}

Adobe-Japan1-4\\
\UTF{1F110}\UTF{1F111}\UTF{1F112}\UTF{1F113}\UTF{1F114}\UTF{1F115}\UTF{1F116}\UTF{1F117}\UTF{1F118}\UTF{1F119}%
\UTF{1F11A}\UTF{1F11B}\UTF{1F11C}\UTF{1F11D}\UTF{1F11E}\UTF{1F11F}\UTF{1F120}\UTF{1F121}\UTF{1F122}\UTF{1F123}%
\UTF{1F124}\UTF{1F125}\UTF{1F126}\UTF{1F127}\UTF{1F128}\UTF{1F129}\UTF{1F150}\UTF{1F151}\UTF{1F152}\UTF{1F153}%
\UTF{1F154}\UTF{1F155}\UTF{1F156}\UTF{1F157}\UTF{1F158}\UTF{1F159}\UTF{1F15A}\UTF{1F15B}\UTF{1F15C}\UTF{1F15D}%
\UTF{1F15E}\UTF{1F15F}\UTF{1F160}\UTF{1F161}\UTF{1F162}\UTF{1F163}\UTF{1F164}\UTF{1F165}\UTF{1F166}\UTF{1F167}%
\UTF{1F168}\UTF{1F169}\UTF{1F130}\UTF{1F131}\UTF{1F132}\UTF{1F133}\UTF{1F134}\UTF{1F135}\UTF{1F136}\UTF{1F137}%
\UTF{1F138}\UTF{1F139}\UTF{1F13A}\UTF{1F13B}\UTF{1F13C}\UTF{1F13D}\UTF{1F13E}\UTF{1F13F}\UTF{1F140}\UTF{1F141}%
\UTF{1F142}\UTF{1F143}\UTF{1F144}\UTF{1F145}\UTF{1F146}\UTF{1F147}\UTF{1F148}\UTF{1F149}\UTF{1F202}\UTF{1F237}%
\UTF{1F170}\UTF{1F171}\UTF{1F172}\UTF{1F173}\UTF{1F174}\UTF{1F175}\UTF{1F176}\UTF{1F177}\UTF{1F178}\UTF{1F179}%
\UTF{1F17A}\UTF{1F17B}\UTF{1F17C}\UTF{1F17D}\UTF{1F17E}\UTF{1F17F}\UTF{1F180}\UTF{1F181}\UTF{1F182}\UTF{1F183}%
\UTF{1F184}\UTF{1F185}\UTF{1F186}\UTF{1F187}\UTF{1F188}\UTF{1F189}\UTF{2F945}\UTF{2090E}\UTF{26951}\UTF{2B7D8}%
\UTF{2F8FC}\UTF{2F995}\UTF{2F8EA}\UTF{2F822}\UTF{26222}\UTF{20BB7}\UTF{29D4B}\UTF{2F833}\UTF{2B78E}\UTF{2F8AC}%
\UTF{20A64}\UTF{2F903}\UTF{2B746}\UTF{2B777}\UTF{2F90B}\UTF{20B9F}\UTF{2F828}\UTF{2F921}\UTF{2F83F}\UTF{2F873}%
\UTF{2D544}\UTF{2000B}\UTF{2F852}\UTF{2967F}\UTF{2F947}\UTF{201A2}\UTF{2E569}\UTF{2B751}\UTF{2F8B2}\UTF{27FB7}%
\UTF{23CFE}\UTF{2F91A}\UTF{25AD7}\UTF{2F89A}\UTF{2F90F}\UTF{2123D}\UTF{2F81A}\UTF{24D14}\UTF{2F862}\UTF{2B789}%
\UTF{2F9D0}\UTF{2F9DF}\UTF{2567F}\UTF{266B0}\UTF{20628}\UTF{2008A}\UTF{20984}\UTF{2F82C}\UTF{2F86D}\UTF{2F8B6}%
\UTF{26999}\UTF{233CC}\UTF{2F8DB}\UTF{2A9E6}\UTF{2B7BD}\UTF{2F96C}\UTF{2E278}\UTF{2053F}\UTF{2626A}\UTF{200B0}%
\UTF{2E6EA}\UTF{28987}\UTF{28E17}\UTF{2B81A}\UTF{242EE}\UTF{2F8E1}\UTF{23CBE}\UTF{20611}\UTF{2F9F4}\UTF{2F804}%
\UTF{2363A}\UTF{233FE}\UTF{22609}

Adobe-Japan1-5\\
\UTF{2131B}\UTF{2146E}\UTF{218BD}\UTF{216B4}\UTF{21E34}\UTF{231C4}\UTF{235C4}\UTF{2373F}\UTF{23763}\UTF{247F1}%
\UTF{2548E}\UTF{2550E}\UTF{25771}\UTF{259C4}\UTF{25DA1}\UTF{26AFF}\UTF{26E40}\UTF{270F4}\UTF{27684}\UTF{28277}%
\UTF{283CD}\UTF{2A190}\UTF{20089}\UTF{200A2}\UTF{200A4}\UTF{20213}\UTF{2032B}\UTF{20381}\UTF{20371}\UTF{203F9}%
\UTF{2044A}\UTF{20509}\UTF{205D6}\UTF{2074F}\UTF{20807}\UTF{2083A}\UTF{208B9}\UTF{2097C}\UTF{2099D}\UTF{20AD3}%
\UTF{20B1D}\UTF{20D45}\UTF{20DE1}\UTF{20E95}\UTF{20E6D}\UTF{20E64}\UTF{20F5F}\UTF{21201}\UTF{21255}\UTF{2127B}%
\UTF{21274}\UTF{212E4}\UTF{212D7}\UTF{212FD}\UTF{21336}\UTF{21344}\UTF{213C4}\UTF{2146D}\UTF{215D7}\UTF{26C29}%
\UTF{21647}\UTF{21706}\UTF{21742}\UTF{219C3}\UTF{21C56}\UTF{21D2D}\UTF{21D45}\UTF{21D78}\UTF{21D62}\UTF{21DA1}%
\UTF{21D9C}\UTF{21D92}\UTF{21DB7}\UTF{21DE0}\UTF{21E33}\UTF{21F1E}\UTF{21F76}\UTF{21FFA}\UTF{2217B}\UTF{2231E}%
\UTF{223AD}\UTF{226F3}\UTF{2285B}\UTF{228AB}\UTF{2298F}\UTF{22AB8}\UTF{22B4F}\UTF{22B50}\UTF{22B46}\UTF{22C1D}%
\UTF{22BA6}\UTF{22C24}\UTF{22DE1}\UTF{231C3}\UTF{231F5}\UTF{231B6}\UTF{23372}\UTF{233D3}\UTF{233D2}\UTF{233D0}%
\UTF{233E4}\UTF{233D5}\UTF{233DA}\UTF{233DF}\UTF{2344A}\UTF{23451}\UTF{2344B}\UTF{23465}\UTF{234E4}\UTF{2355A}%
\UTF{23594}\UTF{23639}\UTF{23647}\UTF{23638}\UTF{2371C}\UTF{2370C}\UTF{23764}\UTF{237FF}\UTF{237E7}\UTF{23824}%
\UTF{2383D}\UTF{23A98}\UTF{23C7F}\UTF{23D00}\UTF{23D40}\UTF{23DFA}\UTF{23DF9}\UTF{23DD3}\UTF{23F7E}\UTF{24096}%
\UTF{24103}\UTF{241C6}\UTF{241FE}\UTF{243BC}\UTF{24629}\UTF{246A5}\UTF{24896}\UTF{24A4D}\UTF{24B56}\UTF{24B6F}%
\UTF{24C16}\UTF{24E0E}\UTF{24E37}\UTF{24E6A}\UTF{24E8B}\UTF{2504A}\UTF{25055}\UTF{25122}\UTF{251A9}\UTF{251E5}%
\UTF{251CD}\UTF{2521E}\UTF{2524C}\UTF{2542E}\UTF{254D9}\UTF{255A7}\UTF{257A9}\UTF{257B4}\UTF{259D4}\UTF{25AE4}%
\UTF{25AE3}\UTF{25AF1}\UTF{25BB2}\UTF{25C4B}\UTF{25C64}\UTF{25E2E}\UTF{25E56}\UTF{25E65}\UTF{25E62}\UTF{25ED8}%
\UTF{25EC2}\UTF{25EE8}\UTF{25F23}\UTF{25F5C}\UTF{25FE0}\UTF{25FD4}\UTF{2600C}\UTF{25FFB}\UTF{26017}\UTF{26060}%
\UTF{260ED}\UTF{26270}\UTF{26286}\UTF{23D0E}\UTF{26402}\UTF{2667E}\UTF{2671D}\UTF{268DD}\UTF{268EA}\UTF{2696F}%
\UTF{269DD}\UTF{26A1E}\UTF{26A58}\UTF{26A8C}\UTF{26AB7}\UTF{26C73}\UTF{26CDD}\UTF{26E65}\UTF{26F94}\UTF{26FF8}%
\UTF{26FF6}\UTF{26FF7}\UTF{2710D}\UTF{27139}\UTF{273DB}\UTF{273DA}\UTF{273FE}\UTF{27410}\UTF{27449}\UTF{27615}%
\UTF{27614}\UTF{27631}\UTF{27693}\UTF{2770E}\UTF{27723}\UTF{27752}\UTF{27985}\UTF{27A84}\UTF{27BB3}\UTF{27BBE}%
\UTF{27BC7}\UTF{27CB8}\UTF{27DA0}\UTF{27E10}\UTF{2808A}\UTF{280BB}\UTF{28282}\UTF{282F3}\UTF{2840C}\UTF{28455}%
\UTF{2856B}\UTF{285C8}\UTF{285C9}\UTF{286D7}\UTF{286FA}\UTF{28949}\UTF{28946}\UTF{2896B}\UTF{28988}\UTF{289BA}%
\UTF{289BB}\UTF{28A1E}\UTF{28A29}\UTF{28A71}\UTF{28A43}\UTF{28A99}\UTF{28ACD}\UTF{28AE4}\UTF{28ADD}\UTF{28BC1}%
\UTF{28BEF}\UTF{28D10}\UTF{28D71}\UTF{28DFB}\UTF{28E1F}\UTF{28E36}\UTF{28E89}\UTF{28EEB}\UTF{28F32}\UTF{28FF8}%
\UTF{292A0}\UTF{292B1}\UTF{29490}\UTF{295CF}\UTF{296F0}\UTF{29719}\UTF{29750}\UTF{298C6}\UTF{29A72}\UTF{29DDB}%
\UTF{29E15}\UTF{29E8A}\UTF{29E49}\UTF{29EC4}\UTF{29EE9}\UTF{29EDB}\UTF{29FCE}\UTF{29FD7}\UTF{2A02F}\UTF{2A01A}%
\UTF{2A0F9}\UTF{2A082}\UTF{22218}\UTF{2A38C}\UTF{2A437}\UTF{2A5F1}\UTF{2A602}\UTF{2A6B2}\UTF{200F5}\UTF{24E04}%
\UTF{24FF2}\UTF{27D73}\UTF{2F815}\UTF{2F846}\UTF{2F899}\UTF{2F8A6}\UTF{2F8E5}\UTF{2F9DE}\UTF{2A2B2}\UTF{20158}%
\UTF{205B1}\UTF{206EC}\UTF{2B753}\UTF{20D58}\UTF{2B75A}\UTF{2B75C}\UTF{259CC}\UTF{2B776}\UTF{22E42}\UTF{2B77C}%
\UTF{207C8}\UTF{22FEB}\UTF{279B4}\UTF{2B782}\UTF{2B78B}\UTF{237F1}\UTF{2B794}\UTF{2404B}\UTF{2B7AC}\UTF{2B7AF}%
\UTF{2B7C9}\UTF{2B7CF}\UTF{2B7D2}\UTF{26C9E}\UTF{27C3C}\UTF{2B7F0}\UTF{2B765}\UTF{2B80D}\UTF{2B817}\UTF{2634C}%
\UTF{29E3D}\UTF{2A61A}

% end


{\bfseries%
[mc/bx]

\ifuptexmode
 %
% This file is generated from the data of UniJIS-UTF32
% in cid2code.txt (Version 02/04/2012)
% for Adobe-Japan1-6
%
% Reference:
%   http://sourceforge.net/adobe/cmap/home/Home/
%   cmapresources_japan1-6.tar.z
%
% A newer CMap may be required for some code points.
%


Adobe-Japan1-0\\
𨳝櫛𥡴𨻶杓巽屠兔冕冤
𡨚𤏐爨🄀

Adobe-Japan1-4\\
🄐🄑🄒🄓🄔🄕🄖🄗🄘🄙
🄚🄛🄜🄝🄞🄟🄠🄡🄢🄣
🄤🄥🄦🄧🄨🄩🅐🅑🅒🅓
🅔🅕🅖🅗🅘🅙🅚🅛🅜🅝
🅞🅟🅠🅡🅢🅣🅤🅥🅦🅧
🅨🅩🄰🄱🄲🄳🄴🄵🄶🄷
🄸🄹🄺🄻🄼🄽🄾🄿🅀🅁
🅂🅃🅄🅅🅆🅇🅈🅉🈂🈷
🅰🅱🅲🅳🅴🅵🅶🅷🅸🅹
🅺🅻🅼🅽🅾🅿🆀🆁🆂🆃
🆄🆅🆆🆇🆈🆉眞𠤎𦥑𫟘
沿芽槪割𦈢𠮷𩵋卿𫞎憲
𠩤浩𫝆𫝷滋𠮟勺爵周将
𠀋城𩙿真𠆢𫝑成𧾷𣳾炭
𥫗彫潮𡈽冬𤴔姬𫞉諭輸
𥙿𦚰𠘨𠂊𠦄卉寃拔𦦙𣏌
杞𪧦𫞽絣𠔿𦉪𠂰𨦇𨸗𫠚
𤋮桒𣲾𠘑嶲你𣘺𣏾𢘉

Adobe-Japan1-5\\
𡌛𡑮𡢽𡚴𡸴𣇄𣗄𣜿𣝣𤟱
𥒎𥔎𥝱𥧄𥶡𦫿𦹀𧃴𧚄𨉷
𨏍𪆐𠂉𠂢𠂤𠈓𠌫𠎁𠍱𠏹
𠑊𠔉𠗖𠝏𠠇𠠺𠢹𠥼𠦝𠫓
𠬝𠵅𠷡𠺕𠹭𠹤𠽟𡈁𡉕𡉻
𡉴𡋤𡋗𡋽𡌶𡍄𡏄𡑭𡗗𦰩
𡙇𡜆𡝂𡧃𡱖𡴭𡵅𡵸𡵢𡶡
𡶜𡶒𡶷𡷠𡸳𡼞𡽶𡿺𢅻𢌞
𢎭𢛳𢡛𢢫𢦏𢪸𢭏𢭐𢭆𢰝
𢮦𢰤𢷡𣇃𣇵𣆶𣍲𣏓𣏒𣏐
𣏤𣏕𣏚𣏟𣑊𣑑𣑋𣑥𣓤𣕚
𣖔𣘹𣙇𣘸𣜜𣜌𣝤𣟿𣟧𣠤
𣠽𣪘𣱿𣴀𣵀𣷺𣷹𣷓𣽾𤂖
𤄃𤇆𤇾𤎼𤘩𤚥𤢖𤩍𤭖𤭯
𤰖𤸎𤸷𤹪𤺋𥁊𥁕𥄢𥆩𥇥
𥇍𥈞𥉌𥐮𥓙𥖧𥞩𥞴𥧔𥫤
𥫣𥫱𥮲𥱋𥱤𥸮𥹖𥹥𥹢𥻘
𥻂𥻨𥼣𥽜𥿠𥿔𦀌𥿻𦀗𦁠
𦃭𦉰𦊆𣴎𦐂𦙾𦜝𦣝𦣪𦥯
𦧝𦨞𦩘𦪌𦪷𦱳𦳝𦹥𦾔𦿸
𦿶𦿷𧄍𧄹𧏛𧏚𧏾𧐐𧑉𧘕
𧘔𧘱𧚓𧜎𧜣𧝒𧦅𧪄𧮳𧮾
𧯇𧲸𧶠𧸐𨂊𨂻𨊂𨋳𨐌𨑕
𨕫𨗈𨗉𨛗𨛺𨥉𨥆𨥫𨦈𨦺
𨦻𨨞𨨩𨩱𨩃𨪙𨫍𨫤𨫝𨯁
𨯯𨴐𨵱𨷻𨸟𨸶𨺉𨻫𨼲𨿸
𩊠𩊱𩒐𩗏𩛰𩜙𩝐𩣆𩩲𩷛
𩸕𩺊𩹉𩻄𩻩𩻛𩿎𩿗𪀯𪀚
𪃹𪂂𢈘𪎌𪐷𪗱𪘂𪚲𠃵𤸄
𤿲𧵳再善形慈栟軔𪊲𠅘
𠖱𠛬𫝓𠵘𫝚𫝜𥧌𫝶𢹂𫝼
𠟈𢿫𧦴𫞂𫞋𣟱𫞔𤁋𫞬𫞯
𫟉𫟏𫟒𦲞𧰼𫟰𫝥𫠍𫠗𦍌
𩸽𪘚

% end

\fi
%
% This file is generated from the data of UniJIS-UTF32
% in cid2code.txt (Version 07/30/2018)
% for Adobe-Japan1-7
%
% Reference:
%   https://github.com/adobe-type-tools/cmap-resources/
%   Adobe-Japan1-7/cid2code.txt
%
% A newer CMap may be required for some code points.
%


Adobe-Japan1-0\\
\UTF{28CDD}\UTF{2F8ED}\UTF{25874}\UTF{28EF6}\UTF{2F8DC}\UTF{2F884}\UTF{2F877}\UTF{2F80F}\UTF{2F8D3}\UTF{2F818}%
\UTF{21A1A}\UTF{243D0}\UTF{2F920}\UTF{1F100}

Adobe-Japan1-4\\
\UTF{1F110}\UTF{1F111}\UTF{1F112}\UTF{1F113}\UTF{1F114}\UTF{1F115}\UTF{1F116}\UTF{1F117}\UTF{1F118}\UTF{1F119}%
\UTF{1F11A}\UTF{1F11B}\UTF{1F11C}\UTF{1F11D}\UTF{1F11E}\UTF{1F11F}\UTF{1F120}\UTF{1F121}\UTF{1F122}\UTF{1F123}%
\UTF{1F124}\UTF{1F125}\UTF{1F126}\UTF{1F127}\UTF{1F128}\UTF{1F129}\UTF{1F150}\UTF{1F151}\UTF{1F152}\UTF{1F153}%
\UTF{1F154}\UTF{1F155}\UTF{1F156}\UTF{1F157}\UTF{1F158}\UTF{1F159}\UTF{1F15A}\UTF{1F15B}\UTF{1F15C}\UTF{1F15D}%
\UTF{1F15E}\UTF{1F15F}\UTF{1F160}\UTF{1F161}\UTF{1F162}\UTF{1F163}\UTF{1F164}\UTF{1F165}\UTF{1F166}\UTF{1F167}%
\UTF{1F168}\UTF{1F169}\UTF{1F130}\UTF{1F131}\UTF{1F132}\UTF{1F133}\UTF{1F134}\UTF{1F135}\UTF{1F136}\UTF{1F137}%
\UTF{1F138}\UTF{1F139}\UTF{1F13A}\UTF{1F13B}\UTF{1F13C}\UTF{1F13D}\UTF{1F13E}\UTF{1F13F}\UTF{1F140}\UTF{1F141}%
\UTF{1F142}\UTF{1F143}\UTF{1F144}\UTF{1F145}\UTF{1F146}\UTF{1F147}\UTF{1F148}\UTF{1F149}\UTF{1F202}\UTF{1F237}%
\UTF{1F170}\UTF{1F171}\UTF{1F172}\UTF{1F173}\UTF{1F174}\UTF{1F175}\UTF{1F176}\UTF{1F177}\UTF{1F178}\UTF{1F179}%
\UTF{1F17A}\UTF{1F17B}\UTF{1F17C}\UTF{1F17D}\UTF{1F17E}\UTF{1F17F}\UTF{1F180}\UTF{1F181}\UTF{1F182}\UTF{1F183}%
\UTF{1F184}\UTF{1F185}\UTF{1F186}\UTF{1F187}\UTF{1F188}\UTF{1F189}\UTF{2F945}\UTF{2090E}\UTF{26951}\UTF{2B7D8}%
\UTF{2F8FC}\UTF{2F995}\UTF{2F8EA}\UTF{2F822}\UTF{26222}\UTF{20BB7}\UTF{29D4B}\UTF{2F833}\UTF{2B78E}\UTF{2F8AC}%
\UTF{20A64}\UTF{2F903}\UTF{2B746}\UTF{2B777}\UTF{2F90B}\UTF{20B9F}\UTF{2F828}\UTF{2F921}\UTF{2F83F}\UTF{2F873}%
\UTF{2D544}\UTF{2000B}\UTF{2F852}\UTF{2967F}\UTF{2F947}\UTF{201A2}\UTF{2E569}\UTF{2B751}\UTF{2F8B2}\UTF{27FB7}%
\UTF{23CFE}\UTF{2F91A}\UTF{25AD7}\UTF{2F89A}\UTF{2F90F}\UTF{2123D}\UTF{2F81A}\UTF{24D14}\UTF{2F862}\UTF{2B789}%
\UTF{2F9D0}\UTF{2F9DF}\UTF{2567F}\UTF{266B0}\UTF{20628}\UTF{2008A}\UTF{20984}\UTF{2F82C}\UTF{2F86D}\UTF{2F8B6}%
\UTF{26999}\UTF{233CC}\UTF{2F8DB}\UTF{2A9E6}\UTF{2B7BD}\UTF{2F96C}\UTF{2E278}\UTF{2053F}\UTF{2626A}\UTF{200B0}%
\UTF{2E6EA}\UTF{28987}\UTF{28E17}\UTF{2B81A}\UTF{242EE}\UTF{2F8E1}\UTF{23CBE}\UTF{20611}\UTF{2F9F4}\UTF{2F804}%
\UTF{2363A}\UTF{233FE}\UTF{22609}

Adobe-Japan1-5\\
\UTF{2131B}\UTF{2146E}\UTF{218BD}\UTF{216B4}\UTF{21E34}\UTF{231C4}\UTF{235C4}\UTF{2373F}\UTF{23763}\UTF{247F1}%
\UTF{2548E}\UTF{2550E}\UTF{25771}\UTF{259C4}\UTF{25DA1}\UTF{26AFF}\UTF{26E40}\UTF{270F4}\UTF{27684}\UTF{28277}%
\UTF{283CD}\UTF{2A190}\UTF{20089}\UTF{200A2}\UTF{200A4}\UTF{20213}\UTF{2032B}\UTF{20381}\UTF{20371}\UTF{203F9}%
\UTF{2044A}\UTF{20509}\UTF{205D6}\UTF{2074F}\UTF{20807}\UTF{2083A}\UTF{208B9}\UTF{2097C}\UTF{2099D}\UTF{20AD3}%
\UTF{20B1D}\UTF{20D45}\UTF{20DE1}\UTF{20E95}\UTF{20E6D}\UTF{20E64}\UTF{20F5F}\UTF{21201}\UTF{21255}\UTF{2127B}%
\UTF{21274}\UTF{212E4}\UTF{212D7}\UTF{212FD}\UTF{21336}\UTF{21344}\UTF{213C4}\UTF{2146D}\UTF{215D7}\UTF{26C29}%
\UTF{21647}\UTF{21706}\UTF{21742}\UTF{219C3}\UTF{21C56}\UTF{21D2D}\UTF{21D45}\UTF{21D78}\UTF{21D62}\UTF{21DA1}%
\UTF{21D9C}\UTF{21D92}\UTF{21DB7}\UTF{21DE0}\UTF{21E33}\UTF{21F1E}\UTF{21F76}\UTF{21FFA}\UTF{2217B}\UTF{2231E}%
\UTF{223AD}\UTF{226F3}\UTF{2285B}\UTF{228AB}\UTF{2298F}\UTF{22AB8}\UTF{22B4F}\UTF{22B50}\UTF{22B46}\UTF{22C1D}%
\UTF{22BA6}\UTF{22C24}\UTF{22DE1}\UTF{231C3}\UTF{231F5}\UTF{231B6}\UTF{23372}\UTF{233D3}\UTF{233D2}\UTF{233D0}%
\UTF{233E4}\UTF{233D5}\UTF{233DA}\UTF{233DF}\UTF{2344A}\UTF{23451}\UTF{2344B}\UTF{23465}\UTF{234E4}\UTF{2355A}%
\UTF{23594}\UTF{23639}\UTF{23647}\UTF{23638}\UTF{2371C}\UTF{2370C}\UTF{23764}\UTF{237FF}\UTF{237E7}\UTF{23824}%
\UTF{2383D}\UTF{23A98}\UTF{23C7F}\UTF{23D00}\UTF{23D40}\UTF{23DFA}\UTF{23DF9}\UTF{23DD3}\UTF{23F7E}\UTF{24096}%
\UTF{24103}\UTF{241C6}\UTF{241FE}\UTF{243BC}\UTF{24629}\UTF{246A5}\UTF{24896}\UTF{24A4D}\UTF{24B56}\UTF{24B6F}%
\UTF{24C16}\UTF{24E0E}\UTF{24E37}\UTF{24E6A}\UTF{24E8B}\UTF{2504A}\UTF{25055}\UTF{25122}\UTF{251A9}\UTF{251E5}%
\UTF{251CD}\UTF{2521E}\UTF{2524C}\UTF{2542E}\UTF{254D9}\UTF{255A7}\UTF{257A9}\UTF{257B4}\UTF{259D4}\UTF{25AE4}%
\UTF{25AE3}\UTF{25AF1}\UTF{25BB2}\UTF{25C4B}\UTF{25C64}\UTF{25E2E}\UTF{25E56}\UTF{25E65}\UTF{25E62}\UTF{25ED8}%
\UTF{25EC2}\UTF{25EE8}\UTF{25F23}\UTF{25F5C}\UTF{25FE0}\UTF{25FD4}\UTF{2600C}\UTF{25FFB}\UTF{26017}\UTF{26060}%
\UTF{260ED}\UTF{26270}\UTF{26286}\UTF{23D0E}\UTF{26402}\UTF{2667E}\UTF{2671D}\UTF{268DD}\UTF{268EA}\UTF{2696F}%
\UTF{269DD}\UTF{26A1E}\UTF{26A58}\UTF{26A8C}\UTF{26AB7}\UTF{26C73}\UTF{26CDD}\UTF{26E65}\UTF{26F94}\UTF{26FF8}%
\UTF{26FF6}\UTF{26FF7}\UTF{2710D}\UTF{27139}\UTF{273DB}\UTF{273DA}\UTF{273FE}\UTF{27410}\UTF{27449}\UTF{27615}%
\UTF{27614}\UTF{27631}\UTF{27693}\UTF{2770E}\UTF{27723}\UTF{27752}\UTF{27985}\UTF{27A84}\UTF{27BB3}\UTF{27BBE}%
\UTF{27BC7}\UTF{27CB8}\UTF{27DA0}\UTF{27E10}\UTF{2808A}\UTF{280BB}\UTF{28282}\UTF{282F3}\UTF{2840C}\UTF{28455}%
\UTF{2856B}\UTF{285C8}\UTF{285C9}\UTF{286D7}\UTF{286FA}\UTF{28949}\UTF{28946}\UTF{2896B}\UTF{28988}\UTF{289BA}%
\UTF{289BB}\UTF{28A1E}\UTF{28A29}\UTF{28A71}\UTF{28A43}\UTF{28A99}\UTF{28ACD}\UTF{28AE4}\UTF{28ADD}\UTF{28BC1}%
\UTF{28BEF}\UTF{28D10}\UTF{28D71}\UTF{28DFB}\UTF{28E1F}\UTF{28E36}\UTF{28E89}\UTF{28EEB}\UTF{28F32}\UTF{28FF8}%
\UTF{292A0}\UTF{292B1}\UTF{29490}\UTF{295CF}\UTF{296F0}\UTF{29719}\UTF{29750}\UTF{298C6}\UTF{29A72}\UTF{29DDB}%
\UTF{29E15}\UTF{29E8A}\UTF{29E49}\UTF{29EC4}\UTF{29EE9}\UTF{29EDB}\UTF{29FCE}\UTF{29FD7}\UTF{2A02F}\UTF{2A01A}%
\UTF{2A0F9}\UTF{2A082}\UTF{22218}\UTF{2A38C}\UTF{2A437}\UTF{2A5F1}\UTF{2A602}\UTF{2A6B2}\UTF{200F5}\UTF{24E04}%
\UTF{24FF2}\UTF{27D73}\UTF{2F815}\UTF{2F846}\UTF{2F899}\UTF{2F8A6}\UTF{2F8E5}\UTF{2F9DE}\UTF{2A2B2}\UTF{20158}%
\UTF{205B1}\UTF{206EC}\UTF{2B753}\UTF{20D58}\UTF{2B75A}\UTF{2B75C}\UTF{259CC}\UTF{2B776}\UTF{22E42}\UTF{2B77C}%
\UTF{207C8}\UTF{22FEB}\UTF{279B4}\UTF{2B782}\UTF{2B78B}\UTF{237F1}\UTF{2B794}\UTF{2404B}\UTF{2B7AC}\UTF{2B7AF}%
\UTF{2B7C9}\UTF{2B7CF}\UTF{2B7D2}\UTF{26C9E}\UTF{27C3C}\UTF{2B7F0}\UTF{2B765}\UTF{2B80D}\UTF{2B817}\UTF{2634C}%
\UTF{29E3D}\UTF{2A61A}

% end


}

{\gtfamily
[gt/m]

\ifuptexmode
 %
% This file is generated from the data of UniJIS-UTF32
% in cid2code.txt (Version 02/04/2012)
% for Adobe-Japan1-6
%
% Reference:
%   http://sourceforge.net/adobe/cmap/home/Home/
%   cmapresources_japan1-6.tar.z
%
% A newer CMap may be required for some code points.
%


Adobe-Japan1-0\\
𨳝櫛𥡴𨻶杓巽屠兔冕冤
𡨚𤏐爨🄀

Adobe-Japan1-4\\
🄐🄑🄒🄓🄔🄕🄖🄗🄘🄙
🄚🄛🄜🄝🄞🄟🄠🄡🄢🄣
🄤🄥🄦🄧🄨🄩🅐🅑🅒🅓
🅔🅕🅖🅗🅘🅙🅚🅛🅜🅝
🅞🅟🅠🅡🅢🅣🅤🅥🅦🅧
🅨🅩🄰🄱🄲🄳🄴🄵🄶🄷
🄸🄹🄺🄻🄼🄽🄾🄿🅀🅁
🅂🅃🅄🅅🅆🅇🅈🅉🈂🈷
🅰🅱🅲🅳🅴🅵🅶🅷🅸🅹
🅺🅻🅼🅽🅾🅿🆀🆁🆂🆃
🆄🆅🆆🆇🆈🆉眞𠤎𦥑𫟘
沿芽槪割𦈢𠮷𩵋卿𫞎憲
𠩤浩𫝆𫝷滋𠮟勺爵周将
𠀋城𩙿真𠆢𫝑成𧾷𣳾炭
𥫗彫潮𡈽冬𤴔姬𫞉諭輸
𥙿𦚰𠘨𠂊𠦄卉寃拔𦦙𣏌
杞𪧦𫞽絣𠔿𦉪𠂰𨦇𨸗𫠚
𤋮桒𣲾𠘑嶲你𣘺𣏾𢘉

Adobe-Japan1-5\\
𡌛𡑮𡢽𡚴𡸴𣇄𣗄𣜿𣝣𤟱
𥒎𥔎𥝱𥧄𥶡𦫿𦹀𧃴𧚄𨉷
𨏍𪆐𠂉𠂢𠂤𠈓𠌫𠎁𠍱𠏹
𠑊𠔉𠗖𠝏𠠇𠠺𠢹𠥼𠦝𠫓
𠬝𠵅𠷡𠺕𠹭𠹤𠽟𡈁𡉕𡉻
𡉴𡋤𡋗𡋽𡌶𡍄𡏄𡑭𡗗𦰩
𡙇𡜆𡝂𡧃𡱖𡴭𡵅𡵸𡵢𡶡
𡶜𡶒𡶷𡷠𡸳𡼞𡽶𡿺𢅻𢌞
𢎭𢛳𢡛𢢫𢦏𢪸𢭏𢭐𢭆𢰝
𢮦𢰤𢷡𣇃𣇵𣆶𣍲𣏓𣏒𣏐
𣏤𣏕𣏚𣏟𣑊𣑑𣑋𣑥𣓤𣕚
𣖔𣘹𣙇𣘸𣜜𣜌𣝤𣟿𣟧𣠤
𣠽𣪘𣱿𣴀𣵀𣷺𣷹𣷓𣽾𤂖
𤄃𤇆𤇾𤎼𤘩𤚥𤢖𤩍𤭖𤭯
𤰖𤸎𤸷𤹪𤺋𥁊𥁕𥄢𥆩𥇥
𥇍𥈞𥉌𥐮𥓙𥖧𥞩𥞴𥧔𥫤
𥫣𥫱𥮲𥱋𥱤𥸮𥹖𥹥𥹢𥻘
𥻂𥻨𥼣𥽜𥿠𥿔𦀌𥿻𦀗𦁠
𦃭𦉰𦊆𣴎𦐂𦙾𦜝𦣝𦣪𦥯
𦧝𦨞𦩘𦪌𦪷𦱳𦳝𦹥𦾔𦿸
𦿶𦿷𧄍𧄹𧏛𧏚𧏾𧐐𧑉𧘕
𧘔𧘱𧚓𧜎𧜣𧝒𧦅𧪄𧮳𧮾
𧯇𧲸𧶠𧸐𨂊𨂻𨊂𨋳𨐌𨑕
𨕫𨗈𨗉𨛗𨛺𨥉𨥆𨥫𨦈𨦺
𨦻𨨞𨨩𨩱𨩃𨪙𨫍𨫤𨫝𨯁
𨯯𨴐𨵱𨷻𨸟𨸶𨺉𨻫𨼲𨿸
𩊠𩊱𩒐𩗏𩛰𩜙𩝐𩣆𩩲𩷛
𩸕𩺊𩹉𩻄𩻩𩻛𩿎𩿗𪀯𪀚
𪃹𪂂𢈘𪎌𪐷𪗱𪘂𪚲𠃵𤸄
𤿲𧵳再善形慈栟軔𪊲𠅘
𠖱𠛬𫝓𠵘𫝚𫝜𥧌𫝶𢹂𫝼
𠟈𢿫𧦴𫞂𫞋𣟱𫞔𤁋𫞬𫞯
𫟉𫟏𫟒𦲞𧰼𫟰𫝥𫠍𫠗𦍌
𩸽𪘚

% end

\fi
%
% This file is generated from the data of UniJIS-UTF32
% in cid2code.txt (Version 07/30/2018)
% for Adobe-Japan1-7
%
% Reference:
%   https://github.com/adobe-type-tools/cmap-resources/
%   Adobe-Japan1-7/cid2code.txt
%
% A newer CMap may be required for some code points.
%


Adobe-Japan1-0\\
\UTF{28CDD}\UTF{2F8ED}\UTF{25874}\UTF{28EF6}\UTF{2F8DC}\UTF{2F884}\UTF{2F877}\UTF{2F80F}\UTF{2F8D3}\UTF{2F818}%
\UTF{21A1A}\UTF{243D0}\UTF{2F920}\UTF{1F100}

Adobe-Japan1-4\\
\UTF{1F110}\UTF{1F111}\UTF{1F112}\UTF{1F113}\UTF{1F114}\UTF{1F115}\UTF{1F116}\UTF{1F117}\UTF{1F118}\UTF{1F119}%
\UTF{1F11A}\UTF{1F11B}\UTF{1F11C}\UTF{1F11D}\UTF{1F11E}\UTF{1F11F}\UTF{1F120}\UTF{1F121}\UTF{1F122}\UTF{1F123}%
\UTF{1F124}\UTF{1F125}\UTF{1F126}\UTF{1F127}\UTF{1F128}\UTF{1F129}\UTF{1F150}\UTF{1F151}\UTF{1F152}\UTF{1F153}%
\UTF{1F154}\UTF{1F155}\UTF{1F156}\UTF{1F157}\UTF{1F158}\UTF{1F159}\UTF{1F15A}\UTF{1F15B}\UTF{1F15C}\UTF{1F15D}%
\UTF{1F15E}\UTF{1F15F}\UTF{1F160}\UTF{1F161}\UTF{1F162}\UTF{1F163}\UTF{1F164}\UTF{1F165}\UTF{1F166}\UTF{1F167}%
\UTF{1F168}\UTF{1F169}\UTF{1F130}\UTF{1F131}\UTF{1F132}\UTF{1F133}\UTF{1F134}\UTF{1F135}\UTF{1F136}\UTF{1F137}%
\UTF{1F138}\UTF{1F139}\UTF{1F13A}\UTF{1F13B}\UTF{1F13C}\UTF{1F13D}\UTF{1F13E}\UTF{1F13F}\UTF{1F140}\UTF{1F141}%
\UTF{1F142}\UTF{1F143}\UTF{1F144}\UTF{1F145}\UTF{1F146}\UTF{1F147}\UTF{1F148}\UTF{1F149}\UTF{1F202}\UTF{1F237}%
\UTF{1F170}\UTF{1F171}\UTF{1F172}\UTF{1F173}\UTF{1F174}\UTF{1F175}\UTF{1F176}\UTF{1F177}\UTF{1F178}\UTF{1F179}%
\UTF{1F17A}\UTF{1F17B}\UTF{1F17C}\UTF{1F17D}\UTF{1F17E}\UTF{1F17F}\UTF{1F180}\UTF{1F181}\UTF{1F182}\UTF{1F183}%
\UTF{1F184}\UTF{1F185}\UTF{1F186}\UTF{1F187}\UTF{1F188}\UTF{1F189}\UTF{2F945}\UTF{2090E}\UTF{26951}\UTF{2B7D8}%
\UTF{2F8FC}\UTF{2F995}\UTF{2F8EA}\UTF{2F822}\UTF{26222}\UTF{20BB7}\UTF{29D4B}\UTF{2F833}\UTF{2B78E}\UTF{2F8AC}%
\UTF{20A64}\UTF{2F903}\UTF{2B746}\UTF{2B777}\UTF{2F90B}\UTF{20B9F}\UTF{2F828}\UTF{2F921}\UTF{2F83F}\UTF{2F873}%
\UTF{2D544}\UTF{2000B}\UTF{2F852}\UTF{2967F}\UTF{2F947}\UTF{201A2}\UTF{2E569}\UTF{2B751}\UTF{2F8B2}\UTF{27FB7}%
\UTF{23CFE}\UTF{2F91A}\UTF{25AD7}\UTF{2F89A}\UTF{2F90F}\UTF{2123D}\UTF{2F81A}\UTF{24D14}\UTF{2F862}\UTF{2B789}%
\UTF{2F9D0}\UTF{2F9DF}\UTF{2567F}\UTF{266B0}\UTF{20628}\UTF{2008A}\UTF{20984}\UTF{2F82C}\UTF{2F86D}\UTF{2F8B6}%
\UTF{26999}\UTF{233CC}\UTF{2F8DB}\UTF{2A9E6}\UTF{2B7BD}\UTF{2F96C}\UTF{2E278}\UTF{2053F}\UTF{2626A}\UTF{200B0}%
\UTF{2E6EA}\UTF{28987}\UTF{28E17}\UTF{2B81A}\UTF{242EE}\UTF{2F8E1}\UTF{23CBE}\UTF{20611}\UTF{2F9F4}\UTF{2F804}%
\UTF{2363A}\UTF{233FE}\UTF{22609}

Adobe-Japan1-5\\
\UTF{2131B}\UTF{2146E}\UTF{218BD}\UTF{216B4}\UTF{21E34}\UTF{231C4}\UTF{235C4}\UTF{2373F}\UTF{23763}\UTF{247F1}%
\UTF{2548E}\UTF{2550E}\UTF{25771}\UTF{259C4}\UTF{25DA1}\UTF{26AFF}\UTF{26E40}\UTF{270F4}\UTF{27684}\UTF{28277}%
\UTF{283CD}\UTF{2A190}\UTF{20089}\UTF{200A2}\UTF{200A4}\UTF{20213}\UTF{2032B}\UTF{20381}\UTF{20371}\UTF{203F9}%
\UTF{2044A}\UTF{20509}\UTF{205D6}\UTF{2074F}\UTF{20807}\UTF{2083A}\UTF{208B9}\UTF{2097C}\UTF{2099D}\UTF{20AD3}%
\UTF{20B1D}\UTF{20D45}\UTF{20DE1}\UTF{20E95}\UTF{20E6D}\UTF{20E64}\UTF{20F5F}\UTF{21201}\UTF{21255}\UTF{2127B}%
\UTF{21274}\UTF{212E4}\UTF{212D7}\UTF{212FD}\UTF{21336}\UTF{21344}\UTF{213C4}\UTF{2146D}\UTF{215D7}\UTF{26C29}%
\UTF{21647}\UTF{21706}\UTF{21742}\UTF{219C3}\UTF{21C56}\UTF{21D2D}\UTF{21D45}\UTF{21D78}\UTF{21D62}\UTF{21DA1}%
\UTF{21D9C}\UTF{21D92}\UTF{21DB7}\UTF{21DE0}\UTF{21E33}\UTF{21F1E}\UTF{21F76}\UTF{21FFA}\UTF{2217B}\UTF{2231E}%
\UTF{223AD}\UTF{226F3}\UTF{2285B}\UTF{228AB}\UTF{2298F}\UTF{22AB8}\UTF{22B4F}\UTF{22B50}\UTF{22B46}\UTF{22C1D}%
\UTF{22BA6}\UTF{22C24}\UTF{22DE1}\UTF{231C3}\UTF{231F5}\UTF{231B6}\UTF{23372}\UTF{233D3}\UTF{233D2}\UTF{233D0}%
\UTF{233E4}\UTF{233D5}\UTF{233DA}\UTF{233DF}\UTF{2344A}\UTF{23451}\UTF{2344B}\UTF{23465}\UTF{234E4}\UTF{2355A}%
\UTF{23594}\UTF{23639}\UTF{23647}\UTF{23638}\UTF{2371C}\UTF{2370C}\UTF{23764}\UTF{237FF}\UTF{237E7}\UTF{23824}%
\UTF{2383D}\UTF{23A98}\UTF{23C7F}\UTF{23D00}\UTF{23D40}\UTF{23DFA}\UTF{23DF9}\UTF{23DD3}\UTF{23F7E}\UTF{24096}%
\UTF{24103}\UTF{241C6}\UTF{241FE}\UTF{243BC}\UTF{24629}\UTF{246A5}\UTF{24896}\UTF{24A4D}\UTF{24B56}\UTF{24B6F}%
\UTF{24C16}\UTF{24E0E}\UTF{24E37}\UTF{24E6A}\UTF{24E8B}\UTF{2504A}\UTF{25055}\UTF{25122}\UTF{251A9}\UTF{251E5}%
\UTF{251CD}\UTF{2521E}\UTF{2524C}\UTF{2542E}\UTF{254D9}\UTF{255A7}\UTF{257A9}\UTF{257B4}\UTF{259D4}\UTF{25AE4}%
\UTF{25AE3}\UTF{25AF1}\UTF{25BB2}\UTF{25C4B}\UTF{25C64}\UTF{25E2E}\UTF{25E56}\UTF{25E65}\UTF{25E62}\UTF{25ED8}%
\UTF{25EC2}\UTF{25EE8}\UTF{25F23}\UTF{25F5C}\UTF{25FE0}\UTF{25FD4}\UTF{2600C}\UTF{25FFB}\UTF{26017}\UTF{26060}%
\UTF{260ED}\UTF{26270}\UTF{26286}\UTF{23D0E}\UTF{26402}\UTF{2667E}\UTF{2671D}\UTF{268DD}\UTF{268EA}\UTF{2696F}%
\UTF{269DD}\UTF{26A1E}\UTF{26A58}\UTF{26A8C}\UTF{26AB7}\UTF{26C73}\UTF{26CDD}\UTF{26E65}\UTF{26F94}\UTF{26FF8}%
\UTF{26FF6}\UTF{26FF7}\UTF{2710D}\UTF{27139}\UTF{273DB}\UTF{273DA}\UTF{273FE}\UTF{27410}\UTF{27449}\UTF{27615}%
\UTF{27614}\UTF{27631}\UTF{27693}\UTF{2770E}\UTF{27723}\UTF{27752}\UTF{27985}\UTF{27A84}\UTF{27BB3}\UTF{27BBE}%
\UTF{27BC7}\UTF{27CB8}\UTF{27DA0}\UTF{27E10}\UTF{2808A}\UTF{280BB}\UTF{28282}\UTF{282F3}\UTF{2840C}\UTF{28455}%
\UTF{2856B}\UTF{285C8}\UTF{285C9}\UTF{286D7}\UTF{286FA}\UTF{28949}\UTF{28946}\UTF{2896B}\UTF{28988}\UTF{289BA}%
\UTF{289BB}\UTF{28A1E}\UTF{28A29}\UTF{28A71}\UTF{28A43}\UTF{28A99}\UTF{28ACD}\UTF{28AE4}\UTF{28ADD}\UTF{28BC1}%
\UTF{28BEF}\UTF{28D10}\UTF{28D71}\UTF{28DFB}\UTF{28E1F}\UTF{28E36}\UTF{28E89}\UTF{28EEB}\UTF{28F32}\UTF{28FF8}%
\UTF{292A0}\UTF{292B1}\UTF{29490}\UTF{295CF}\UTF{296F0}\UTF{29719}\UTF{29750}\UTF{298C6}\UTF{29A72}\UTF{29DDB}%
\UTF{29E15}\UTF{29E8A}\UTF{29E49}\UTF{29EC4}\UTF{29EE9}\UTF{29EDB}\UTF{29FCE}\UTF{29FD7}\UTF{2A02F}\UTF{2A01A}%
\UTF{2A0F9}\UTF{2A082}\UTF{22218}\UTF{2A38C}\UTF{2A437}\UTF{2A5F1}\UTF{2A602}\UTF{2A6B2}\UTF{200F5}\UTF{24E04}%
\UTF{24FF2}\UTF{27D73}\UTF{2F815}\UTF{2F846}\UTF{2F899}\UTF{2F8A6}\UTF{2F8E5}\UTF{2F9DE}\UTF{2A2B2}\UTF{20158}%
\UTF{205B1}\UTF{206EC}\UTF{2B753}\UTF{20D58}\UTF{2B75A}\UTF{2B75C}\UTF{259CC}\UTF{2B776}\UTF{22E42}\UTF{2B77C}%
\UTF{207C8}\UTF{22FEB}\UTF{279B4}\UTF{2B782}\UTF{2B78B}\UTF{237F1}\UTF{2B794}\UTF{2404B}\UTF{2B7AC}\UTF{2B7AF}%
\UTF{2B7C9}\UTF{2B7CF}\UTF{2B7D2}\UTF{26C9E}\UTF{27C3C}\UTF{2B7F0}\UTF{2B765}\UTF{2B80D}\UTF{2B817}\UTF{2634C}%
\UTF{29E3D}\UTF{2A61A}

% end


{\bfseries%
[gt/bx]

\ifuptexmode
 %
% This file is generated from the data of UniJIS-UTF32
% in cid2code.txt (Version 02/04/2012)
% for Adobe-Japan1-6
%
% Reference:
%   http://sourceforge.net/adobe/cmap/home/Home/
%   cmapresources_japan1-6.tar.z
%
% A newer CMap may be required for some code points.
%


Adobe-Japan1-0\\
𨳝櫛𥡴𨻶杓巽屠兔冕冤
𡨚𤏐爨🄀

Adobe-Japan1-4\\
🄐🄑🄒🄓🄔🄕🄖🄗🄘🄙
🄚🄛🄜🄝🄞🄟🄠🄡🄢🄣
🄤🄥🄦🄧🄨🄩🅐🅑🅒🅓
🅔🅕🅖🅗🅘🅙🅚🅛🅜🅝
🅞🅟🅠🅡🅢🅣🅤🅥🅦🅧
🅨🅩🄰🄱🄲🄳🄴🄵🄶🄷
🄸🄹🄺🄻🄼🄽🄾🄿🅀🅁
🅂🅃🅄🅅🅆🅇🅈🅉🈂🈷
🅰🅱🅲🅳🅴🅵🅶🅷🅸🅹
🅺🅻🅼🅽🅾🅿🆀🆁🆂🆃
🆄🆅🆆🆇🆈🆉眞𠤎𦥑𫟘
沿芽槪割𦈢𠮷𩵋卿𫞎憲
𠩤浩𫝆𫝷滋𠮟勺爵周将
𠀋城𩙿真𠆢𫝑成𧾷𣳾炭
𥫗彫潮𡈽冬𤴔姬𫞉諭輸
𥙿𦚰𠘨𠂊𠦄卉寃拔𦦙𣏌
杞𪧦𫞽絣𠔿𦉪𠂰𨦇𨸗𫠚
𤋮桒𣲾𠘑嶲你𣘺𣏾𢘉

Adobe-Japan1-5\\
𡌛𡑮𡢽𡚴𡸴𣇄𣗄𣜿𣝣𤟱
𥒎𥔎𥝱𥧄𥶡𦫿𦹀𧃴𧚄𨉷
𨏍𪆐𠂉𠂢𠂤𠈓𠌫𠎁𠍱𠏹
𠑊𠔉𠗖𠝏𠠇𠠺𠢹𠥼𠦝𠫓
𠬝𠵅𠷡𠺕𠹭𠹤𠽟𡈁𡉕𡉻
𡉴𡋤𡋗𡋽𡌶𡍄𡏄𡑭𡗗𦰩
𡙇𡜆𡝂𡧃𡱖𡴭𡵅𡵸𡵢𡶡
𡶜𡶒𡶷𡷠𡸳𡼞𡽶𡿺𢅻𢌞
𢎭𢛳𢡛𢢫𢦏𢪸𢭏𢭐𢭆𢰝
𢮦𢰤𢷡𣇃𣇵𣆶𣍲𣏓𣏒𣏐
𣏤𣏕𣏚𣏟𣑊𣑑𣑋𣑥𣓤𣕚
𣖔𣘹𣙇𣘸𣜜𣜌𣝤𣟿𣟧𣠤
𣠽𣪘𣱿𣴀𣵀𣷺𣷹𣷓𣽾𤂖
𤄃𤇆𤇾𤎼𤘩𤚥𤢖𤩍𤭖𤭯
𤰖𤸎𤸷𤹪𤺋𥁊𥁕𥄢𥆩𥇥
𥇍𥈞𥉌𥐮𥓙𥖧𥞩𥞴𥧔𥫤
𥫣𥫱𥮲𥱋𥱤𥸮𥹖𥹥𥹢𥻘
𥻂𥻨𥼣𥽜𥿠𥿔𦀌𥿻𦀗𦁠
𦃭𦉰𦊆𣴎𦐂𦙾𦜝𦣝𦣪𦥯
𦧝𦨞𦩘𦪌𦪷𦱳𦳝𦹥𦾔𦿸
𦿶𦿷𧄍𧄹𧏛𧏚𧏾𧐐𧑉𧘕
𧘔𧘱𧚓𧜎𧜣𧝒𧦅𧪄𧮳𧮾
𧯇𧲸𧶠𧸐𨂊𨂻𨊂𨋳𨐌𨑕
𨕫𨗈𨗉𨛗𨛺𨥉𨥆𨥫𨦈𨦺
𨦻𨨞𨨩𨩱𨩃𨪙𨫍𨫤𨫝𨯁
𨯯𨴐𨵱𨷻𨸟𨸶𨺉𨻫𨼲𨿸
𩊠𩊱𩒐𩗏𩛰𩜙𩝐𩣆𩩲𩷛
𩸕𩺊𩹉𩻄𩻩𩻛𩿎𩿗𪀯𪀚
𪃹𪂂𢈘𪎌𪐷𪗱𪘂𪚲𠃵𤸄
𤿲𧵳再善形慈栟軔𪊲𠅘
𠖱𠛬𫝓𠵘𫝚𫝜𥧌𫝶𢹂𫝼
𠟈𢿫𧦴𫞂𫞋𣟱𫞔𤁋𫞬𫞯
𫟉𫟏𫟒𦲞𧰼𫟰𫝥𫠍𫠗𦍌
𩸽𪘚

% end

\fi
%
% This file is generated from the data of UniJIS-UTF32
% in cid2code.txt (Version 07/30/2018)
% for Adobe-Japan1-7
%
% Reference:
%   https://github.com/adobe-type-tools/cmap-resources/
%   Adobe-Japan1-7/cid2code.txt
%
% A newer CMap may be required for some code points.
%


Adobe-Japan1-0\\
\UTF{28CDD}\UTF{2F8ED}\UTF{25874}\UTF{28EF6}\UTF{2F8DC}\UTF{2F884}\UTF{2F877}\UTF{2F80F}\UTF{2F8D3}\UTF{2F818}%
\UTF{21A1A}\UTF{243D0}\UTF{2F920}\UTF{1F100}

Adobe-Japan1-4\\
\UTF{1F110}\UTF{1F111}\UTF{1F112}\UTF{1F113}\UTF{1F114}\UTF{1F115}\UTF{1F116}\UTF{1F117}\UTF{1F118}\UTF{1F119}%
\UTF{1F11A}\UTF{1F11B}\UTF{1F11C}\UTF{1F11D}\UTF{1F11E}\UTF{1F11F}\UTF{1F120}\UTF{1F121}\UTF{1F122}\UTF{1F123}%
\UTF{1F124}\UTF{1F125}\UTF{1F126}\UTF{1F127}\UTF{1F128}\UTF{1F129}\UTF{1F150}\UTF{1F151}\UTF{1F152}\UTF{1F153}%
\UTF{1F154}\UTF{1F155}\UTF{1F156}\UTF{1F157}\UTF{1F158}\UTF{1F159}\UTF{1F15A}\UTF{1F15B}\UTF{1F15C}\UTF{1F15D}%
\UTF{1F15E}\UTF{1F15F}\UTF{1F160}\UTF{1F161}\UTF{1F162}\UTF{1F163}\UTF{1F164}\UTF{1F165}\UTF{1F166}\UTF{1F167}%
\UTF{1F168}\UTF{1F169}\UTF{1F130}\UTF{1F131}\UTF{1F132}\UTF{1F133}\UTF{1F134}\UTF{1F135}\UTF{1F136}\UTF{1F137}%
\UTF{1F138}\UTF{1F139}\UTF{1F13A}\UTF{1F13B}\UTF{1F13C}\UTF{1F13D}\UTF{1F13E}\UTF{1F13F}\UTF{1F140}\UTF{1F141}%
\UTF{1F142}\UTF{1F143}\UTF{1F144}\UTF{1F145}\UTF{1F146}\UTF{1F147}\UTF{1F148}\UTF{1F149}\UTF{1F202}\UTF{1F237}%
\UTF{1F170}\UTF{1F171}\UTF{1F172}\UTF{1F173}\UTF{1F174}\UTF{1F175}\UTF{1F176}\UTF{1F177}\UTF{1F178}\UTF{1F179}%
\UTF{1F17A}\UTF{1F17B}\UTF{1F17C}\UTF{1F17D}\UTF{1F17E}\UTF{1F17F}\UTF{1F180}\UTF{1F181}\UTF{1F182}\UTF{1F183}%
\UTF{1F184}\UTF{1F185}\UTF{1F186}\UTF{1F187}\UTF{1F188}\UTF{1F189}\UTF{2F945}\UTF{2090E}\UTF{26951}\UTF{2B7D8}%
\UTF{2F8FC}\UTF{2F995}\UTF{2F8EA}\UTF{2F822}\UTF{26222}\UTF{20BB7}\UTF{29D4B}\UTF{2F833}\UTF{2B78E}\UTF{2F8AC}%
\UTF{20A64}\UTF{2F903}\UTF{2B746}\UTF{2B777}\UTF{2F90B}\UTF{20B9F}\UTF{2F828}\UTF{2F921}\UTF{2F83F}\UTF{2F873}%
\UTF{2D544}\UTF{2000B}\UTF{2F852}\UTF{2967F}\UTF{2F947}\UTF{201A2}\UTF{2E569}\UTF{2B751}\UTF{2F8B2}\UTF{27FB7}%
\UTF{23CFE}\UTF{2F91A}\UTF{25AD7}\UTF{2F89A}\UTF{2F90F}\UTF{2123D}\UTF{2F81A}\UTF{24D14}\UTF{2F862}\UTF{2B789}%
\UTF{2F9D0}\UTF{2F9DF}\UTF{2567F}\UTF{266B0}\UTF{20628}\UTF{2008A}\UTF{20984}\UTF{2F82C}\UTF{2F86D}\UTF{2F8B6}%
\UTF{26999}\UTF{233CC}\UTF{2F8DB}\UTF{2A9E6}\UTF{2B7BD}\UTF{2F96C}\UTF{2E278}\UTF{2053F}\UTF{2626A}\UTF{200B0}%
\UTF{2E6EA}\UTF{28987}\UTF{28E17}\UTF{2B81A}\UTF{242EE}\UTF{2F8E1}\UTF{23CBE}\UTF{20611}\UTF{2F9F4}\UTF{2F804}%
\UTF{2363A}\UTF{233FE}\UTF{22609}

Adobe-Japan1-5\\
\UTF{2131B}\UTF{2146E}\UTF{218BD}\UTF{216B4}\UTF{21E34}\UTF{231C4}\UTF{235C4}\UTF{2373F}\UTF{23763}\UTF{247F1}%
\UTF{2548E}\UTF{2550E}\UTF{25771}\UTF{259C4}\UTF{25DA1}\UTF{26AFF}\UTF{26E40}\UTF{270F4}\UTF{27684}\UTF{28277}%
\UTF{283CD}\UTF{2A190}\UTF{20089}\UTF{200A2}\UTF{200A4}\UTF{20213}\UTF{2032B}\UTF{20381}\UTF{20371}\UTF{203F9}%
\UTF{2044A}\UTF{20509}\UTF{205D6}\UTF{2074F}\UTF{20807}\UTF{2083A}\UTF{208B9}\UTF{2097C}\UTF{2099D}\UTF{20AD3}%
\UTF{20B1D}\UTF{20D45}\UTF{20DE1}\UTF{20E95}\UTF{20E6D}\UTF{20E64}\UTF{20F5F}\UTF{21201}\UTF{21255}\UTF{2127B}%
\UTF{21274}\UTF{212E4}\UTF{212D7}\UTF{212FD}\UTF{21336}\UTF{21344}\UTF{213C4}\UTF{2146D}\UTF{215D7}\UTF{26C29}%
\UTF{21647}\UTF{21706}\UTF{21742}\UTF{219C3}\UTF{21C56}\UTF{21D2D}\UTF{21D45}\UTF{21D78}\UTF{21D62}\UTF{21DA1}%
\UTF{21D9C}\UTF{21D92}\UTF{21DB7}\UTF{21DE0}\UTF{21E33}\UTF{21F1E}\UTF{21F76}\UTF{21FFA}\UTF{2217B}\UTF{2231E}%
\UTF{223AD}\UTF{226F3}\UTF{2285B}\UTF{228AB}\UTF{2298F}\UTF{22AB8}\UTF{22B4F}\UTF{22B50}\UTF{22B46}\UTF{22C1D}%
\UTF{22BA6}\UTF{22C24}\UTF{22DE1}\UTF{231C3}\UTF{231F5}\UTF{231B6}\UTF{23372}\UTF{233D3}\UTF{233D2}\UTF{233D0}%
\UTF{233E4}\UTF{233D5}\UTF{233DA}\UTF{233DF}\UTF{2344A}\UTF{23451}\UTF{2344B}\UTF{23465}\UTF{234E4}\UTF{2355A}%
\UTF{23594}\UTF{23639}\UTF{23647}\UTF{23638}\UTF{2371C}\UTF{2370C}\UTF{23764}\UTF{237FF}\UTF{237E7}\UTF{23824}%
\UTF{2383D}\UTF{23A98}\UTF{23C7F}\UTF{23D00}\UTF{23D40}\UTF{23DFA}\UTF{23DF9}\UTF{23DD3}\UTF{23F7E}\UTF{24096}%
\UTF{24103}\UTF{241C6}\UTF{241FE}\UTF{243BC}\UTF{24629}\UTF{246A5}\UTF{24896}\UTF{24A4D}\UTF{24B56}\UTF{24B6F}%
\UTF{24C16}\UTF{24E0E}\UTF{24E37}\UTF{24E6A}\UTF{24E8B}\UTF{2504A}\UTF{25055}\UTF{25122}\UTF{251A9}\UTF{251E5}%
\UTF{251CD}\UTF{2521E}\UTF{2524C}\UTF{2542E}\UTF{254D9}\UTF{255A7}\UTF{257A9}\UTF{257B4}\UTF{259D4}\UTF{25AE4}%
\UTF{25AE3}\UTF{25AF1}\UTF{25BB2}\UTF{25C4B}\UTF{25C64}\UTF{25E2E}\UTF{25E56}\UTF{25E65}\UTF{25E62}\UTF{25ED8}%
\UTF{25EC2}\UTF{25EE8}\UTF{25F23}\UTF{25F5C}\UTF{25FE0}\UTF{25FD4}\UTF{2600C}\UTF{25FFB}\UTF{26017}\UTF{26060}%
\UTF{260ED}\UTF{26270}\UTF{26286}\UTF{23D0E}\UTF{26402}\UTF{2667E}\UTF{2671D}\UTF{268DD}\UTF{268EA}\UTF{2696F}%
\UTF{269DD}\UTF{26A1E}\UTF{26A58}\UTF{26A8C}\UTF{26AB7}\UTF{26C73}\UTF{26CDD}\UTF{26E65}\UTF{26F94}\UTF{26FF8}%
\UTF{26FF6}\UTF{26FF7}\UTF{2710D}\UTF{27139}\UTF{273DB}\UTF{273DA}\UTF{273FE}\UTF{27410}\UTF{27449}\UTF{27615}%
\UTF{27614}\UTF{27631}\UTF{27693}\UTF{2770E}\UTF{27723}\UTF{27752}\UTF{27985}\UTF{27A84}\UTF{27BB3}\UTF{27BBE}%
\UTF{27BC7}\UTF{27CB8}\UTF{27DA0}\UTF{27E10}\UTF{2808A}\UTF{280BB}\UTF{28282}\UTF{282F3}\UTF{2840C}\UTF{28455}%
\UTF{2856B}\UTF{285C8}\UTF{285C9}\UTF{286D7}\UTF{286FA}\UTF{28949}\UTF{28946}\UTF{2896B}\UTF{28988}\UTF{289BA}%
\UTF{289BB}\UTF{28A1E}\UTF{28A29}\UTF{28A71}\UTF{28A43}\UTF{28A99}\UTF{28ACD}\UTF{28AE4}\UTF{28ADD}\UTF{28BC1}%
\UTF{28BEF}\UTF{28D10}\UTF{28D71}\UTF{28DFB}\UTF{28E1F}\UTF{28E36}\UTF{28E89}\UTF{28EEB}\UTF{28F32}\UTF{28FF8}%
\UTF{292A0}\UTF{292B1}\UTF{29490}\UTF{295CF}\UTF{296F0}\UTF{29719}\UTF{29750}\UTF{298C6}\UTF{29A72}\UTF{29DDB}%
\UTF{29E15}\UTF{29E8A}\UTF{29E49}\UTF{29EC4}\UTF{29EE9}\UTF{29EDB}\UTF{29FCE}\UTF{29FD7}\UTF{2A02F}\UTF{2A01A}%
\UTF{2A0F9}\UTF{2A082}\UTF{22218}\UTF{2A38C}\UTF{2A437}\UTF{2A5F1}\UTF{2A602}\UTF{2A6B2}\UTF{200F5}\UTF{24E04}%
\UTF{24FF2}\UTF{27D73}\UTF{2F815}\UTF{2F846}\UTF{2F899}\UTF{2F8A6}\UTF{2F8E5}\UTF{2F9DE}\UTF{2A2B2}\UTF{20158}%
\UTF{205B1}\UTF{206EC}\UTF{2B753}\UTF{20D58}\UTF{2B75A}\UTF{2B75C}\UTF{259CC}\UTF{2B776}\UTF{22E42}\UTF{2B77C}%
\UTF{207C8}\UTF{22FEB}\UTF{279B4}\UTF{2B782}\UTF{2B78B}\UTF{237F1}\UTF{2B794}\UTF{2404B}\UTF{2B7AC}\UTF{2B7AF}%
\UTF{2B7C9}\UTF{2B7CF}\UTF{2B7D2}\UTF{26C9E}\UTF{27C3C}\UTF{2B7F0}\UTF{2B765}\UTF{2B80D}\UTF{2B817}\UTF{2634C}%
\UTF{29E3D}\UTF{2A61A}

% end


}}

{\mgfamily
[mg/m]

\ifuptexmode
 %
% This file is generated from the data of UniJIS-UTF32
% in cid2code.txt (Version 02/04/2012)
% for Adobe-Japan1-6
%
% Reference:
%   http://sourceforge.net/adobe/cmap/home/Home/
%   cmapresources_japan1-6.tar.z
%
% A newer CMap may be required for some code points.
%


Adobe-Japan1-0\\
𨳝櫛𥡴𨻶杓巽屠兔冕冤
𡨚𤏐爨🄀

Adobe-Japan1-4\\
🄐🄑🄒🄓🄔🄕🄖🄗🄘🄙
🄚🄛🄜🄝🄞🄟🄠🄡🄢🄣
🄤🄥🄦🄧🄨🄩🅐🅑🅒🅓
🅔🅕🅖🅗🅘🅙🅚🅛🅜🅝
🅞🅟🅠🅡🅢🅣🅤🅥🅦🅧
🅨🅩🄰🄱🄲🄳🄴🄵🄶🄷
🄸🄹🄺🄻🄼🄽🄾🄿🅀🅁
🅂🅃🅄🅅🅆🅇🅈🅉🈂🈷
🅰🅱🅲🅳🅴🅵🅶🅷🅸🅹
🅺🅻🅼🅽🅾🅿🆀🆁🆂🆃
🆄🆅🆆🆇🆈🆉眞𠤎𦥑𫟘
沿芽槪割𦈢𠮷𩵋卿𫞎憲
𠩤浩𫝆𫝷滋𠮟勺爵周将
𠀋城𩙿真𠆢𫝑成𧾷𣳾炭
𥫗彫潮𡈽冬𤴔姬𫞉諭輸
𥙿𦚰𠘨𠂊𠦄卉寃拔𦦙𣏌
杞𪧦𫞽絣𠔿𦉪𠂰𨦇𨸗𫠚
𤋮桒𣲾𠘑嶲你𣘺𣏾𢘉

Adobe-Japan1-5\\
𡌛𡑮𡢽𡚴𡸴𣇄𣗄𣜿𣝣𤟱
𥒎𥔎𥝱𥧄𥶡𦫿𦹀𧃴𧚄𨉷
𨏍𪆐𠂉𠂢𠂤𠈓𠌫𠎁𠍱𠏹
𠑊𠔉𠗖𠝏𠠇𠠺𠢹𠥼𠦝𠫓
𠬝𠵅𠷡𠺕𠹭𠹤𠽟𡈁𡉕𡉻
𡉴𡋤𡋗𡋽𡌶𡍄𡏄𡑭𡗗𦰩
𡙇𡜆𡝂𡧃𡱖𡴭𡵅𡵸𡵢𡶡
𡶜𡶒𡶷𡷠𡸳𡼞𡽶𡿺𢅻𢌞
𢎭𢛳𢡛𢢫𢦏𢪸𢭏𢭐𢭆𢰝
𢮦𢰤𢷡𣇃𣇵𣆶𣍲𣏓𣏒𣏐
𣏤𣏕𣏚𣏟𣑊𣑑𣑋𣑥𣓤𣕚
𣖔𣘹𣙇𣘸𣜜𣜌𣝤𣟿𣟧𣠤
𣠽𣪘𣱿𣴀𣵀𣷺𣷹𣷓𣽾𤂖
𤄃𤇆𤇾𤎼𤘩𤚥𤢖𤩍𤭖𤭯
𤰖𤸎𤸷𤹪𤺋𥁊𥁕𥄢𥆩𥇥
𥇍𥈞𥉌𥐮𥓙𥖧𥞩𥞴𥧔𥫤
𥫣𥫱𥮲𥱋𥱤𥸮𥹖𥹥𥹢𥻘
𥻂𥻨𥼣𥽜𥿠𥿔𦀌𥿻𦀗𦁠
𦃭𦉰𦊆𣴎𦐂𦙾𦜝𦣝𦣪𦥯
𦧝𦨞𦩘𦪌𦪷𦱳𦳝𦹥𦾔𦿸
𦿶𦿷𧄍𧄹𧏛𧏚𧏾𧐐𧑉𧘕
𧘔𧘱𧚓𧜎𧜣𧝒𧦅𧪄𧮳𧮾
𧯇𧲸𧶠𧸐𨂊𨂻𨊂𨋳𨐌𨑕
𨕫𨗈𨗉𨛗𨛺𨥉𨥆𨥫𨦈𨦺
𨦻𨨞𨨩𨩱𨩃𨪙𨫍𨫤𨫝𨯁
𨯯𨴐𨵱𨷻𨸟𨸶𨺉𨻫𨼲𨿸
𩊠𩊱𩒐𩗏𩛰𩜙𩝐𩣆𩩲𩷛
𩸕𩺊𩹉𩻄𩻩𩻛𩿎𩿗𪀯𪀚
𪃹𪂂𢈘𪎌𪐷𪗱𪘂𪚲𠃵𤸄
𤿲𧵳再善形慈栟軔𪊲𠅘
𠖱𠛬𫝓𠵘𫝚𫝜𥧌𫝶𢹂𫝼
𠟈𢿫𧦴𫞂𫞋𣟱𫞔𤁋𫞬𫞯
𫟉𫟏𫟒𦲞𧰼𫟰𫝥𫠍𫠗𦍌
𩸽𪘚

% end

\fi
%
% This file is generated from the data of UniJIS-UTF32
% in cid2code.txt (Version 07/30/2018)
% for Adobe-Japan1-7
%
% Reference:
%   https://github.com/adobe-type-tools/cmap-resources/
%   Adobe-Japan1-7/cid2code.txt
%
% A newer CMap may be required for some code points.
%


Adobe-Japan1-0\\
\UTF{28CDD}\UTF{2F8ED}\UTF{25874}\UTF{28EF6}\UTF{2F8DC}\UTF{2F884}\UTF{2F877}\UTF{2F80F}\UTF{2F8D3}\UTF{2F818}%
\UTF{21A1A}\UTF{243D0}\UTF{2F920}\UTF{1F100}

Adobe-Japan1-4\\
\UTF{1F110}\UTF{1F111}\UTF{1F112}\UTF{1F113}\UTF{1F114}\UTF{1F115}\UTF{1F116}\UTF{1F117}\UTF{1F118}\UTF{1F119}%
\UTF{1F11A}\UTF{1F11B}\UTF{1F11C}\UTF{1F11D}\UTF{1F11E}\UTF{1F11F}\UTF{1F120}\UTF{1F121}\UTF{1F122}\UTF{1F123}%
\UTF{1F124}\UTF{1F125}\UTF{1F126}\UTF{1F127}\UTF{1F128}\UTF{1F129}\UTF{1F150}\UTF{1F151}\UTF{1F152}\UTF{1F153}%
\UTF{1F154}\UTF{1F155}\UTF{1F156}\UTF{1F157}\UTF{1F158}\UTF{1F159}\UTF{1F15A}\UTF{1F15B}\UTF{1F15C}\UTF{1F15D}%
\UTF{1F15E}\UTF{1F15F}\UTF{1F160}\UTF{1F161}\UTF{1F162}\UTF{1F163}\UTF{1F164}\UTF{1F165}\UTF{1F166}\UTF{1F167}%
\UTF{1F168}\UTF{1F169}\UTF{1F130}\UTF{1F131}\UTF{1F132}\UTF{1F133}\UTF{1F134}\UTF{1F135}\UTF{1F136}\UTF{1F137}%
\UTF{1F138}\UTF{1F139}\UTF{1F13A}\UTF{1F13B}\UTF{1F13C}\UTF{1F13D}\UTF{1F13E}\UTF{1F13F}\UTF{1F140}\UTF{1F141}%
\UTF{1F142}\UTF{1F143}\UTF{1F144}\UTF{1F145}\UTF{1F146}\UTF{1F147}\UTF{1F148}\UTF{1F149}\UTF{1F202}\UTF{1F237}%
\UTF{1F170}\UTF{1F171}\UTF{1F172}\UTF{1F173}\UTF{1F174}\UTF{1F175}\UTF{1F176}\UTF{1F177}\UTF{1F178}\UTF{1F179}%
\UTF{1F17A}\UTF{1F17B}\UTF{1F17C}\UTF{1F17D}\UTF{1F17E}\UTF{1F17F}\UTF{1F180}\UTF{1F181}\UTF{1F182}\UTF{1F183}%
\UTF{1F184}\UTF{1F185}\UTF{1F186}\UTF{1F187}\UTF{1F188}\UTF{1F189}\UTF{2F945}\UTF{2090E}\UTF{26951}\UTF{2B7D8}%
\UTF{2F8FC}\UTF{2F995}\UTF{2F8EA}\UTF{2F822}\UTF{26222}\UTF{20BB7}\UTF{29D4B}\UTF{2F833}\UTF{2B78E}\UTF{2F8AC}%
\UTF{20A64}\UTF{2F903}\UTF{2B746}\UTF{2B777}\UTF{2F90B}\UTF{20B9F}\UTF{2F828}\UTF{2F921}\UTF{2F83F}\UTF{2F873}%
\UTF{2D544}\UTF{2000B}\UTF{2F852}\UTF{2967F}\UTF{2F947}\UTF{201A2}\UTF{2E569}\UTF{2B751}\UTF{2F8B2}\UTF{27FB7}%
\UTF{23CFE}\UTF{2F91A}\UTF{25AD7}\UTF{2F89A}\UTF{2F90F}\UTF{2123D}\UTF{2F81A}\UTF{24D14}\UTF{2F862}\UTF{2B789}%
\UTF{2F9D0}\UTF{2F9DF}\UTF{2567F}\UTF{266B0}\UTF{20628}\UTF{2008A}\UTF{20984}\UTF{2F82C}\UTF{2F86D}\UTF{2F8B6}%
\UTF{26999}\UTF{233CC}\UTF{2F8DB}\UTF{2A9E6}\UTF{2B7BD}\UTF{2F96C}\UTF{2E278}\UTF{2053F}\UTF{2626A}\UTF{200B0}%
\UTF{2E6EA}\UTF{28987}\UTF{28E17}\UTF{2B81A}\UTF{242EE}\UTF{2F8E1}\UTF{23CBE}\UTF{20611}\UTF{2F9F4}\UTF{2F804}%
\UTF{2363A}\UTF{233FE}\UTF{22609}

Adobe-Japan1-5\\
\UTF{2131B}\UTF{2146E}\UTF{218BD}\UTF{216B4}\UTF{21E34}\UTF{231C4}\UTF{235C4}\UTF{2373F}\UTF{23763}\UTF{247F1}%
\UTF{2548E}\UTF{2550E}\UTF{25771}\UTF{259C4}\UTF{25DA1}\UTF{26AFF}\UTF{26E40}\UTF{270F4}\UTF{27684}\UTF{28277}%
\UTF{283CD}\UTF{2A190}\UTF{20089}\UTF{200A2}\UTF{200A4}\UTF{20213}\UTF{2032B}\UTF{20381}\UTF{20371}\UTF{203F9}%
\UTF{2044A}\UTF{20509}\UTF{205D6}\UTF{2074F}\UTF{20807}\UTF{2083A}\UTF{208B9}\UTF{2097C}\UTF{2099D}\UTF{20AD3}%
\UTF{20B1D}\UTF{20D45}\UTF{20DE1}\UTF{20E95}\UTF{20E6D}\UTF{20E64}\UTF{20F5F}\UTF{21201}\UTF{21255}\UTF{2127B}%
\UTF{21274}\UTF{212E4}\UTF{212D7}\UTF{212FD}\UTF{21336}\UTF{21344}\UTF{213C4}\UTF{2146D}\UTF{215D7}\UTF{26C29}%
\UTF{21647}\UTF{21706}\UTF{21742}\UTF{219C3}\UTF{21C56}\UTF{21D2D}\UTF{21D45}\UTF{21D78}\UTF{21D62}\UTF{21DA1}%
\UTF{21D9C}\UTF{21D92}\UTF{21DB7}\UTF{21DE0}\UTF{21E33}\UTF{21F1E}\UTF{21F76}\UTF{21FFA}\UTF{2217B}\UTF{2231E}%
\UTF{223AD}\UTF{226F3}\UTF{2285B}\UTF{228AB}\UTF{2298F}\UTF{22AB8}\UTF{22B4F}\UTF{22B50}\UTF{22B46}\UTF{22C1D}%
\UTF{22BA6}\UTF{22C24}\UTF{22DE1}\UTF{231C3}\UTF{231F5}\UTF{231B6}\UTF{23372}\UTF{233D3}\UTF{233D2}\UTF{233D0}%
\UTF{233E4}\UTF{233D5}\UTF{233DA}\UTF{233DF}\UTF{2344A}\UTF{23451}\UTF{2344B}\UTF{23465}\UTF{234E4}\UTF{2355A}%
\UTF{23594}\UTF{23639}\UTF{23647}\UTF{23638}\UTF{2371C}\UTF{2370C}\UTF{23764}\UTF{237FF}\UTF{237E7}\UTF{23824}%
\UTF{2383D}\UTF{23A98}\UTF{23C7F}\UTF{23D00}\UTF{23D40}\UTF{23DFA}\UTF{23DF9}\UTF{23DD3}\UTF{23F7E}\UTF{24096}%
\UTF{24103}\UTF{241C6}\UTF{241FE}\UTF{243BC}\UTF{24629}\UTF{246A5}\UTF{24896}\UTF{24A4D}\UTF{24B56}\UTF{24B6F}%
\UTF{24C16}\UTF{24E0E}\UTF{24E37}\UTF{24E6A}\UTF{24E8B}\UTF{2504A}\UTF{25055}\UTF{25122}\UTF{251A9}\UTF{251E5}%
\UTF{251CD}\UTF{2521E}\UTF{2524C}\UTF{2542E}\UTF{254D9}\UTF{255A7}\UTF{257A9}\UTF{257B4}\UTF{259D4}\UTF{25AE4}%
\UTF{25AE3}\UTF{25AF1}\UTF{25BB2}\UTF{25C4B}\UTF{25C64}\UTF{25E2E}\UTF{25E56}\UTF{25E65}\UTF{25E62}\UTF{25ED8}%
\UTF{25EC2}\UTF{25EE8}\UTF{25F23}\UTF{25F5C}\UTF{25FE0}\UTF{25FD4}\UTF{2600C}\UTF{25FFB}\UTF{26017}\UTF{26060}%
\UTF{260ED}\UTF{26270}\UTF{26286}\UTF{23D0E}\UTF{26402}\UTF{2667E}\UTF{2671D}\UTF{268DD}\UTF{268EA}\UTF{2696F}%
\UTF{269DD}\UTF{26A1E}\UTF{26A58}\UTF{26A8C}\UTF{26AB7}\UTF{26C73}\UTF{26CDD}\UTF{26E65}\UTF{26F94}\UTF{26FF8}%
\UTF{26FF6}\UTF{26FF7}\UTF{2710D}\UTF{27139}\UTF{273DB}\UTF{273DA}\UTF{273FE}\UTF{27410}\UTF{27449}\UTF{27615}%
\UTF{27614}\UTF{27631}\UTF{27693}\UTF{2770E}\UTF{27723}\UTF{27752}\UTF{27985}\UTF{27A84}\UTF{27BB3}\UTF{27BBE}%
\UTF{27BC7}\UTF{27CB8}\UTF{27DA0}\UTF{27E10}\UTF{2808A}\UTF{280BB}\UTF{28282}\UTF{282F3}\UTF{2840C}\UTF{28455}%
\UTF{2856B}\UTF{285C8}\UTF{285C9}\UTF{286D7}\UTF{286FA}\UTF{28949}\UTF{28946}\UTF{2896B}\UTF{28988}\UTF{289BA}%
\UTF{289BB}\UTF{28A1E}\UTF{28A29}\UTF{28A71}\UTF{28A43}\UTF{28A99}\UTF{28ACD}\UTF{28AE4}\UTF{28ADD}\UTF{28BC1}%
\UTF{28BEF}\UTF{28D10}\UTF{28D71}\UTF{28DFB}\UTF{28E1F}\UTF{28E36}\UTF{28E89}\UTF{28EEB}\UTF{28F32}\UTF{28FF8}%
\UTF{292A0}\UTF{292B1}\UTF{29490}\UTF{295CF}\UTF{296F0}\UTF{29719}\UTF{29750}\UTF{298C6}\UTF{29A72}\UTF{29DDB}%
\UTF{29E15}\UTF{29E8A}\UTF{29E49}\UTF{29EC4}\UTF{29EE9}\UTF{29EDB}\UTF{29FCE}\UTF{29FD7}\UTF{2A02F}\UTF{2A01A}%
\UTF{2A0F9}\UTF{2A082}\UTF{22218}\UTF{2A38C}\UTF{2A437}\UTF{2A5F1}\UTF{2A602}\UTF{2A6B2}\UTF{200F5}\UTF{24E04}%
\UTF{24FF2}\UTF{27D73}\UTF{2F815}\UTF{2F846}\UTF{2F899}\UTF{2F8A6}\UTF{2F8E5}\UTF{2F9DE}\UTF{2A2B2}\UTF{20158}%
\UTF{205B1}\UTF{206EC}\UTF{2B753}\UTF{20D58}\UTF{2B75A}\UTF{2B75C}\UTF{259CC}\UTF{2B776}\UTF{22E42}\UTF{2B77C}%
\UTF{207C8}\UTF{22FEB}\UTF{279B4}\UTF{2B782}\UTF{2B78B}\UTF{237F1}\UTF{2B794}\UTF{2404B}\UTF{2B7AC}\UTF{2B7AF}%
\UTF{2B7C9}\UTF{2B7CF}\UTF{2B7D2}\UTF{26C9E}\UTF{27C3C}\UTF{2B7F0}\UTF{2B765}\UTF{2B80D}\UTF{2B817}\UTF{2634C}%
\UTF{29E3D}\UTF{2A61A}

% end


}

\clearpage
[mc/m]

%
% This file is generated from the data of UniCNS-UTF32
% in cid2code.txt (Version 12/04/2015)
% for Adobe-CNS1-6
%
% Reference:
%   https://github.com/adobe-type-tools/cmap-resources/
%   cmapresources_cns1-6/cid2code.txt
%
% A newer CMap may be required for some code points.
%


Adobe-CNS1-0\\
\UTFT{200CC}\UTFT{2008A}\UTFT{27607}

Adobe-CNS1-1\\
\UTFT{23ED7}\UTFT{26ED3}\UTFT{257E0}\UTFT{28BE9}\UTFT{258E1}\UTFT{294D9}\UTFT{259AC}\UTFT{2648D}\UTFT{25C01}\UTFT{2530E}%
\UTFT{25CFE}\UTFT{25BB4}\UTFT{26C7F}\UTFT{25D20}\UTFT{25CC1}\UTFT{24882}\UTFT{24578}\UTFT{26E44}\UTFT{26ED6}\UTFT{24057}%
\UTFT{26029}\UTFT{217F9}\UTFT{2836D}\UTFT{26121}\UTFT{2615A}\UTFT{262D0}\UTFT{26351}\UTFT{21661}\UTFT{20068}\UTFT{23766}%
\UTFT{2833A}\UTFT{26489}\UTFT{2A087}\UTFT{26CC3}\UTFT{22714}\UTFT{26626}\UTFT{23DE3}\UTFT{266E8}\UTFT{28A48}\UTFT{226F6}%
\UTFT{26498}\UTFT{2148A}\UTFT{2185E}\UTFT{24A65}\UTFT{24A95}\UTFT{26A52}\UTFT{23D7E}\UTFT{214FD}\UTFT{2F98F}\UTFT{249A7}%
\UTFT{23530}\UTFT{21773}\UTFT{23DF8}\UTFT{2F994}\UTFT{20E16}\UTFT{217B4}\UTFT{2317D}\UTFT{2355A}\UTFT{23E8B}\UTFT{26DA3}%
\UTFT{26B05}\UTFT{26B97}\UTFT{235CE}\UTFT{26DA5}\UTFT{26ED4}\UTFT{26E42}\UTFT{25BE4}\UTFT{26B96}\UTFT{26E77}\UTFT{26E43}%
\UTFT{25C91}\UTFT{25CC0}\UTFT{28625}\UTFT{2863B}\UTFT{27088}\UTFT{21582}\UTFT{270CD}\UTFT{2F9B2}\UTFT{218A2}\UTFT{2739A}%
\UTFT{2A0F8}\UTFT{22C27}\UTFT{275E0}\UTFT{23DB9}\UTFT{275E4}\UTFT{2770F}\UTFT{28A25}\UTFT{27924}\UTFT{27ABD}\UTFT{27A59}%
\UTFT{27B3A}\UTFT{27B38}\UTFT{25430}\UTFT{25565}\UTFT{24A7A}\UTFT{216DF}\UTFT{27D54}\UTFT{27D8F}\UTFT{2F9D4}\UTFT{27D53}%
\UTFT{27D98}\UTFT{27DBD}\UTFT{21910}\UTFT{2F9D7}\UTFT{28002}\UTFT{21014}\UTFT{2498A}\UTFT{281BC}\UTFT{2710C}\UTFT{28365}%
\UTFT{28412}\UTFT{2A29F}\UTFT{20A50}\UTFT{289DE}\UTFT{2853D}\UTFT{23DBB}\UTFT{23262}\UTFT{22325}\UTFT{26ED7}\UTFT{2853C}%
\UTFT{27ABE}\UTFT{2856C}\UTFT{2860B}\UTFT{28713}\UTFT{286E6}\UTFT{28933}\UTFT{21E89}\UTFT{255B9}\UTFT{28AC6}\UTFT{23C9B}%
\UTFT{28B0C}\UTFT{255DB}\UTFT{20D31}\UTFT{28AE1}\UTFT{28BEB}\UTFT{28AE2}\UTFT{28AE5}\UTFT{28BEC}\UTFT{28C39}\UTFT{28BFF}%
\UTFT{286D8}\UTFT{2127C}\UTFT{23E2E}\UTFT{26ED5}\UTFT{28AE0}\UTFT{26CB8}\UTFT{20274}\UTFT{26410}\UTFT{290AF}\UTFT{290E5}%
\UTFT{24AD1}\UTFT{21915}\UTFT{2330A}\UTFT{24AE9}\UTFT{291D5}\UTFT{291EB}\UTFT{230B7}\UTFT{230BC}\UTFT{2546C}\UTFT{29433}%
\UTFT{2941D}\UTFT{2797A}\UTFT{27175}\UTFT{20630}\UTFT{2415C}\UTFT{25706}\UTFT{26D27}\UTFT{216D3}\UTFT{24A29}\UTFT{29857}%
\UTFT{29905}\UTFT{25725}\UTFT{290B1}\UTFT{29BD5}\UTFT{29B05}\UTFT{28600}\UTFT{2307D}\UTFT{29D3E}\UTFT{21863}\UTFT{29E68}%
\UTFT{29FB7}\UTFT{2A192}\UTFT{2A1AB}\UTFT{2A0E1}\UTFT{2A123}\UTFT{2A1DF}\UTFT{2A134}\UTFT{2A193}\UTFT{2A220}\UTFT{2193B}%
\UTFT{2A233}\UTFT{2A0B9}\UTFT{2A2B4}\UTFT{24364}\UTFT{28C2B}\UTFT{26DA2}\UTFT{2FA1B}\UTFT{2908B}\UTFT{24975}\UTFT{249BB}%
\UTFT{249F8}\UTFT{24348}\UTFT{24A51}\UTFT{28BDA}\UTFT{218FA}\UTFT{2897E}\UTFT{28E36}\UTFT{28A44}\UTFT{2896C}\UTFT{244B9}%
\UTFT{24473}\UTFT{243F8}\UTFT{217EF}\UTFT{218BE}\UTFT{23599}\UTFT{21885}\UTFT{2542F}\UTFT{217F8}\UTFT{216FB}\UTFT{21839}%
\UTFT{21774}\UTFT{218D1}\UTFT{25F4B}\UTFT{216C0}\UTFT{24A25}\UTFT{213FE}\UTFT{212A8}\UTFT{213C6}\UTFT{214B6}\UTFT{236A6}%
\UTFT{24994}\UTFT{27165}\UTFT{23E31}\UTFT{2555C}\UTFT{23EFB}\UTFT{27052}\UTFT{236EE}\UTFT{2999D}\UTFT{26F26}\UTFT{21922}%
\UTFT{2373F}\UTFT{240E1}\UTFT{2408B}\UTFT{2410F}\UTFT{26C21}\UTFT{266B1}\UTFT{20FDF}\UTFT{20BA8}\UTFT{20E0D}\UTFT{28B13}%
\UTFT{24436}\UTFT{20465}\UTFT{25651}\UTFT{201AB}\UTFT{203CB}\UTFT{2030A}\UTFT{20414}\UTFT{202C0}\UTFT{28EB3}\UTFT{20275}%
\UTFT{2020C}\UTFT{24A0E}\UTFT{23E8A}\UTFT{23595}\UTFT{23E39}\UTFT{23EBF}\UTFT{21884}\UTFT{23E89}\UTFT{205E0}\UTFT{204A3}%
\UTFT{20492}\UTFT{20491}\UTFT{28A9C}\UTFT{2070E}\UTFT{20873}\UTFT{2438C}\UTFT{20C20}\UTFT{249AC}\UTFT{210E4}\UTFT{20E1D}%
\UTFT{24ABC}\UTFT{2408D}\UTFT{240C9}\UTFT{20345}\UTFT{20BC6}\UTFT{28A46}\UTFT{216FA}\UTFT{2176F}\UTFT{21710}\UTFT{25946}%
\UTFT{219F3}\UTFT{21861}\UTFT{24295}\UTFT{25E83}\UTFT{28BD7}\UTFT{20413}\UTFT{21303}\UTFT{289FB}\UTFT{21996}\UTFT{2197C}%
\UTFT{23AEE}\UTFT{21903}\UTFT{21904}\UTFT{218A0}\UTFT{216FE}\UTFT{28A47}\UTFT{21DBA}\UTFT{23472}\UTFT{289A8}\UTFT{21927}%
\UTFT{217AB}\UTFT{2173B}\UTFT{275FD}\UTFT{22860}\UTFT{2262B}\UTFT{225AF}\UTFT{225BE}\UTFT{29088}\UTFT{26F73}\UTFT{2003E}%
\UTFT{20046}\UTFT{2261B}\UTFT{22C9B}\UTFT{22D07}\UTFT{246D4}\UTFT{2914D}\UTFT{24665}\UTFT{22B6A}\UTFT{22B22}\UTFT{23450}%
\UTFT{298EA}\UTFT{22E78}\UTFT{249E3}\UTFT{22D67}\UTFT{22CA1}\UTFT{2308E}\UTFT{232AD}\UTFT{24989}\UTFT{232AB}\UTFT{232E0}%
\UTFT{218D9}\UTFT{2943F}\UTFT{23289}\UTFT{231B3}\UTFT{25584}\UTFT{28B22}\UTFT{2558F}\UTFT{216FC}\UTFT{2555B}\UTFT{25425}%
\UTFT{23103}\UTFT{2182A}\UTFT{23234}\UTFT{2320F}\UTFT{23182}\UTFT{242C9}\UTFT{26D24}\UTFT{27870}\UTFT{21DEB}\UTFT{232D2}%
\UTFT{232E1}\UTFT{25872}\UTFT{2383A}\UTFT{237BC}\UTFT{237A2}\UTFT{233FE}\UTFT{2462A}\UTFT{237D5}\UTFT{24487}\UTFT{21912}%
\UTFT{23FC0}\UTFT{23C9A}\UTFT{28BEA}\UTFT{28ACB}\UTFT{2801E}\UTFT{289DC}\UTFT{23F7F}\UTFT{2403C}\UTFT{2431A}\UTFT{24276}%
\UTFT{2478F}\UTFT{24725}\UTFT{24AA4}\UTFT{205EB}\UTFT{23EF8}\UTFT{2365F}\UTFT{24A4A}\UTFT{24917}\UTFT{25FE1}\UTFT{24ADF}%
\UTFT{28C23}\UTFT{23F35}\UTFT{26DEA}\UTFT{24CD9}\UTFT{24D06}\UTFT{2A5C6}\UTFT{28ACC}\UTFT{249AB}\UTFT{2498E}\UTFT{24A4E}%
\UTFT{249C5}\UTFT{248F3}\UTFT{28AE3}\UTFT{21864}\UTFT{25221}\UTFT{251E7}\UTFT{23232}\UTFT{24697}\UTFT{23781}\UTFT{248F0}%
\UTFT{24ABA}\UTFT{24AC7}\UTFT{24A96}\UTFT{261AE}\UTFT{25581}\UTFT{27741}\UTFT{256E3}\UTFT{23EFA}\UTFT{216E6}\UTFT{20D4C}%
\UTFT{2498C}\UTFT{20299}\UTFT{23DBA}\UTFT{2176E}\UTFT{201D4}\UTFT{20C0D}\UTFT{226F5}\UTFT{25AAF}\UTFT{25A9C}\UTFT{2025B}%
\UTFT{25BC6}\UTFT{25BB3}\UTFT{25EBC}\UTFT{25EA6}\UTFT{249F9}\UTFT{217B0}\UTFT{26261}\UTFT{2615C}\UTFT{27B48}\UTFT{25E82}%
\UTFT{26B75}\UTFT{20916}\UTFT{2004E}\UTFT{235CF}\UTFT{26412}\UTFT{263F8}\UTFT{2082C}\UTFT{25AE9}\UTFT{25D43}\UTFT{25E0E}%
\UTFT{2343F}\UTFT{249F7}\UTFT{265AD}\UTFT{265A0}\UTFT{27127}\UTFT{26CD1}\UTFT{267B4}\UTFT{26A42}\UTFT{26A51}\UTFT{26DA7}%
\UTFT{2721B}\UTFT{21840}\UTFT{218A1}\UTFT{218D8}\UTFT{2F9BC}\UTFT{23D8F}\UTFT{27422}\UTFT{25683}\UTFT{27785}\UTFT{27784}%
\UTFT{28BF5}\UTFT{28BD9}\UTFT{28B9C}\UTFT{289F9}\UTFT{29448}\UTFT{24284}\UTFT{21845}\UTFT{27DDC}\UTFT{24C09}\UTFT{22321}%
\UTFT{217DA}\UTFT{2492F}\UTFT{28A4B}\UTFT{28AFC}\UTFT{28C1D}\UTFT{28C3B}\UTFT{28D34}\UTFT{248FF}\UTFT{24A42}\UTFT{243EA}%
\UTFT{23225}\UTFT{28EE7}\UTFT{28E66}\UTFT{28E65}\UTFT{249ED}\UTFT{24A78}\UTFT{23FEE}\UTFT{290B0}\UTFT{29093}\UTFT{257DF}%
\UTFT{28989}\UTFT{28C26}\UTFT{28B2F}\UTFT{263BE}\UTFT{2421B}\UTFT{20F26}\UTFT{28BC5}\UTFT{24AB2}\UTFT{294DA}\UTFT{295D7}%
\UTFT{28B50}\UTFT{24A67}\UTFT{28B64}\UTFT{28A45}\UTFT{27B06}\UTFT{28B65}\UTFT{258C8}\UTFT{298F1}\UTFT{29948}\UTFT{21302}%
\UTFT{249B8}\UTFT{214E8}\UTFT{2271F}\UTFT{23DB8}\UTFT{22781}\UTFT{2296B}\UTFT{29E2D}\UTFT{2A1F5}\UTFT{2A0FE}\UTFT{24104}%
\UTFT{2A1B4}\UTFT{2A0ED}\UTFT{2A0F3}\UTFT{2992F}\UTFT{26E12}\UTFT{26FDF}\UTFT{26B82}\UTFT{26DA4}\UTFT{26E84}\UTFT{26DF0}%
\UTFT{26E00}\UTFT{237D7}\UTFT{26064}\UTFT{2359C}\UTFT{23640}\UTFT{249DE}\UTFT{202BF}\UTFT{2555D}\UTFT{21757}\UTFT{231C9}%
\UTFT{24941}\UTFT{241B5}\UTFT{241AC}\UTFT{26C40}\UTFT{24F97}\UTFT{217B5}\UTFT{28A49}\UTFT{24488}\UTFT{289FC}\UTFT{218D6}%
\UTFT{20F1D}\UTFT{26CC0}\UTFT{21413}\UTFT{242FA}\UTFT{22C26}\UTFT{243C1}\UTFT{23DB7}\UTFT{26741}\UTFT{2615B}\UTFT{260A4}%
\UTFT{249B9}\UTFT{2498B}\UTFT{289FA}\UTFT{28B63}\UTFT{2189F}\UTFT{24AB3}\UTFT{24A3E}\UTFT{24A94}\UTFT{217D9}\UTFT{24A66}%
\UTFT{203A7}\UTFT{21424}\UTFT{249E5}\UTFT{24916}\UTFT{24976}\UTFT{204FE}\UTFT{28ACE}\UTFT{28A16}\UTFT{28BE7}\UTFT{255D5}%
\UTFT{28A82}\UTFT{24943}\UTFT{20CFF}\UTFT{2061A}\UTFT{20BEB}\UTFT{20CB8}\UTFT{217FA}\UTFT{216C2}\UTFT{24A50}\UTFT{21852}%
\UTFT{28AC0}\UTFT{249AD}\UTFT{218BF}\UTFT{21883}\UTFT{27484}\UTFT{23D5B}\UTFT{28A81}\UTFT{21862}\UTFT{20AB4}\UTFT{2139C}%
\UTFT{28218}\UTFT{290E4}\UTFT{27E4F}\UTFT{23FED}\UTFT{23E2D}\UTFT{203F5}\UTFT{28C1C}\UTFT{26BC0}\UTFT{21452}\UTFT{24362}%
\UTFT{24A71}\UTFT{22FE3}\UTFT{212B0}\UTFT{223BD}\UTFT{21398}\UTFT{234E5}\UTFT{27BF4}\UTFT{236DF}\UTFT{28A83}\UTFT{237D6}%
\UTFT{233FA}\UTFT{24C9F}\UTFT{236AD}\UTFT{26CB7}\UTFT{26D26}\UTFT{26D51}\UTFT{26C82}\UTFT{26FDE}\UTFT{2173A}\UTFT{26C80}%
\UTFT{27053}\UTFT{217DB}\UTFT{217B3}\UTFT{21905}\UTFT{241FC}\UTFT{2173C}\UTFT{242A5}\UTFT{24293}\UTFT{23EF9}\UTFT{27736}%
\UTFT{2445B}\UTFT{242CA}\UTFT{24259}\UTFT{289E1}\UTFT{26D28}\UTFT{244CE}\UTFT{27E4D}\UTFT{243BD}\UTFT{24256}\UTFT{21304}%
\UTFT{243E9}\UTFT{2F825}\UTFT{23300}\UTFT{27AF4}\UTFT{256F6}\UTFT{27B18}\UTFT{27A79}\UTFT{249BA}\UTFT{20346}\UTFT{27657}%
\UTFT{25FE2}\UTFT{275FE}\UTFT{2209A}\UTFT{28A9A}\UTFT{2403B}\UTFT{24A45}\UTFT{205CA}\UTFT{20611}\UTFT{21EA8}\UTFT{23CFF}%
\UTFT{285E8}\UTFT{299C9}\UTFT{221C3}\UTFT{28B4E}\UTFT{20C78}\UTFT{20779}\UTFT{23F4A}\UTFT{24AA7}\UTFT{26B52}\UTFT{27632}%
\UTFT{2493F}\UTFT{233CC}\UTFT{28948}\UTFT{21D90}\UTFT{27C12}\UTFT{24F9A}\UTFT{26BF7}\UTFT{2191C}\UTFT{249F6}\UTFT{23FEF}%
\UTFT{2271B}\UTFT{257E1}\UTFT{2F8CD}\UTFT{2F806}\UTFT{24521}\UTFT{24934}\UTFT{26CBD}\UTFT{26411}\UTFT{290C0}\UTFT{20A11}%
\UTFT{26469}\UTFT{20021}\UTFT{23519}\UTFT{2258D}\UTFT{2217A}\UTFT{249D0}\UTFT{20EF8}\UTFT{22926}\UTFT{28473}\UTFT{217B1}%
\UTFT{24A2A}\UTFT{21820}\UTFT{29CAD}\UTFT{298A4}\UTFT{2160A}\UTFT{2372F}\UTFT{280E8}\UTFT{213C5}\UTFT{291A8}\UTFT{270AF}%
\UTFT{289AB}\UTFT{2417A}\UTFT{2A2DF}\UTFT{28318}\UTFT{26E07}\UTFT{2816F}\UTFT{269B5}\UTFT{213ED}\UTFT{2322F}\UTFT{28C30}%
\UTFT{28949}\UTFT{24988}\UTFT{24AA5}\UTFT{23F81}\UTFT{21FA1}\UTFT{295E9}\UTFT{2789D}\UTFT{28024}\UTFT{27A3E}\UTFT{23CB7}%
\UTFT{26258}\UTFT{29D98}\UTFT{23D40}\UTFT{20E9D}\UTFT{282E2}\UTFT{20C41}\UTFT{20C96}\UTFT{20E76}\UTFT{22C62}\UTFT{20EA2}%
\UTFT{21075}\UTFT{22B43}\UTFT{22EB3}\UTFT{20DA7}\UTFT{2688A}\UTFT{20EF9}\UTFT{27FF9}\UTFT{247E0}\UTFT{29D7C}\UTFT{275A3}%
\UTFT{26048}\UTFT{24618}\UTFT{29EAC}\UTFT{29FDE}\UTFT{272B2}\UTFT{2048E}\UTFT{20EB6}\UTFT{27F2E}\UTFT{2A434}\UTFT{243F2}%
\UTFT{29E06}\UTFT{294D0}\UTFT{26335}\UTFT{20D28}\UTFT{20D71}\UTFT{21F0F}\UTFT{21DD1}\UTFT{2176D}\UTFT{2B473}\UTFT{28E97}%
\UTFT{25C21}\UTFT{20CD4}\UTFT{201F2}\UTFT{2A64A}\UTFT{2837D}\UTFT{2A2B2}\UTFT{24ABB}\UTFT{26E05}\UTFT{2AE67}\UTFT{2251B}%
\UTFT{28E39}\UTFT{20F3B}\UTFT{25F1A}\UTFT{27486}\UTFT{267CC}\UTFT{24011}\UTFT{2F922}\UTFT{20547}\UTFT{205DF}\UTFT{23FC5}%
\UTFT{24942}\UTFT{289E4}\UTFT{219DB}\UTFT{23CC8}\UTFT{24933}\UTFT{289AA}\UTFT{202A0}\UTFT{26BB3}\UTFT{21305}\UTFT{224ED}%
\UTFT{26D29}\UTFT{27A84}\UTFT{23600}\UTFT{24AB1}\UTFT{22513}\UTFT{2037E}\UTFT{20380}\UTFT{20347}\UTFT{2041F}\UTFT{249A4}%
\UTFT{20487}\UTFT{233B4}\UTFT{20BFF}\UTFT{220FC}\UTFT{202E5}\UTFT{22530}\UTFT{2058E}\UTFT{23233}\UTFT{21983}\UTFT{205B3}%
\UTFT{23C99}\UTFT{24AA6}\UTFT{2372D}\UTFT{26B13}\UTFT{2F829}\UTFT{28ADE}\UTFT{23F80}\UTFT{20954}\UTFT{23FEC}\UTFT{20BE2}%
\UTFT{21726}\UTFT{216E8}\UTFT{286AB}\UTFT{2F832}\UTFT{21596}\UTFT{21613}\UTFT{28A9B}\UTFT{25772}\UTFT{20B8F}\UTFT{23FEB}%
\UTFT{22DA3}\UTFT{20C77}\UTFT{26B53}\UTFT{20D74}\UTFT{2170D}\UTFT{20EDD}\UTFT{20D4D}\UTFT{289BC}\UTFT{22698}\UTFT{218D7}%
\UTFT{2403A}\UTFT{24435}\UTFT{210B4}\UTFT{2328A}\UTFT{28B66}\UTFT{2124F}\UTFT{241A5}\UTFT{26C7E}\UTFT{21416}\UTFT{21454}%
\UTFT{24363}\UTFT{24BF5}\UTFT{2123C}\UTFT{2A150}\UTFT{24278}\UTFT{2163E}\UTFT{21692}\UTFT{20D4E}\UTFT{26C81}\UTFT{26D2A}%
\UTFT{217DC}\UTFT{217FB}\UTFT{217B2}\UTFT{26DA6}\UTFT{21828}\UTFT{216D5}\UTFT{26E45}\UTFT{249A9}\UTFT{26FA1}\UTFT{22554}%
\UTFT{21911}\UTFT{216B8}\UTFT{27A0E}\UTFT{20204}\UTFT{21A34}\UTFT{259CC}\UTFT{205A5}\UTFT{21B44}\UTFT{21CA5}\UTFT{26B28}%
\UTFT{21DF9}\UTFT{21E37}\UTFT{21EA4}\UTFT{24901}\UTFT{22049}\UTFT{22173}\UTFT{244BC}\UTFT{20CD3}\UTFT{21771}\UTFT{28482}%
\UTFT{201C1}\UTFT{2F894}\UTFT{2133A}\UTFT{26888}\UTFT{223D0}\UTFT{22471}\UTFT{26E6E}\UTFT{28A36}\UTFT{25250}\UTFT{21F6A}%
\UTFT{270F8}\UTFT{22668}\UTFT{2029E}\UTFT{28A29}\UTFT{227B4}\UTFT{24982}\UTFT{2498F}\UTFT{27A53}\UTFT{2F8A6}\UTFT{26ED2}%
\UTFT{20656}\UTFT{23FB7}\UTFT{2285F}\UTFT{28B9D}\UTFT{2995D}\UTFT{22980}\UTFT{228C1}\UTFT{20118}\UTFT{21770}\UTFT{22E0D}%
\UTFT{249DF}\UTFT{2138E}\UTFT{217FC}\UTFT{22E36}\UTFT{2571D}\UTFT{24A28}\UTFT{24A23}\UTFT{24940}\UTFT{21829}\UTFT{23400}%
\UTFT{231F7}\UTFT{231F8}\UTFT{231A4}\UTFT{231A5}\UTFT{20E75}\UTFT{251E6}\UTFT{23231}\UTFT{285F4}\UTFT{231C8}\UTFT{25313}%
\UTFT{228F7}\UTFT{2439C}\UTFT{24A21}\UTFT{237C2}\UTFT{2F8DB}\UTFT{241CD}\UTFT{290ED}\UTFT{233E6}\UTFT{26DA0}\UTFT{2346F}%
\UTFT{28ADF}\UTFT{235CD}\UTFT{2363C}\UTFT{28A4A}\UTFT{203C9}\UTFT{23659}\UTFT{2212A}\UTFT{23703}\UTFT{2919C}\UTFT{20923}%
\UTFT{227CD}\UTFT{23ADB}\UTFT{21958}\UTFT{23B5A}\UTFT{23EFC}\UTFT{2248B}\UTFT{248F1}\UTFT{26B51}\UTFT{23DBC}\UTFT{23DBD}%
\UTFT{241A4}\UTFT{2490C}\UTFT{24900}\UTFT{23CC9}\UTFT{20D32}\UTFT{231F9}\UTFT{22491}\UTFT{26D25}\UTFT{26DA1}\UTFT{26DEB}%
\UTFT{2497F}\UTFT{24085}\UTFT{26E72}\UTFT{26F74}\UTFT{28B21}\UTFT{2F908}\UTFT{23E2F}\UTFT{23F82}\UTFT{2304B}\UTFT{23E30}%
\UTFT{21497}\UTFT{2403D}\UTFT{29170}\UTFT{24144}\UTFT{24091}\UTFT{24155}\UTFT{24039}\UTFT{23FF0}\UTFT{23FB4}\UTFT{2413F}%
\UTFT{24156}\UTFT{24157}\UTFT{24140}\UTFT{261DD}\UTFT{24277}\UTFT{24365}\UTFT{242C1}\UTFT{2445A}\UTFT{24A27}\UTFT{24A22}%
\UTFT{28BE8}\UTFT{25605}\UTFT{24974}\UTFT{23044}\UTFT{24823}\UTFT{2882B}\UTFT{28804}\UTFT{20C3A}\UTFT{26A2E}\UTFT{241E2}%
\UTFT{216E7}\UTFT{24A24}\UTFT{249B7}\UTFT{2498D}\UTFT{249FB}\UTFT{24A26}\UTFT{2F92F}\UTFT{228AD}\UTFT{28EB2}\UTFT{24A8C}%
\UTFT{2415F}\UTFT{24A79}\UTFT{28B8F}\UTFT{28C03}\UTFT{2189E}\UTFT{21988}\UTFT{28ED9}\UTFT{21A4B}\UTFT{28EAC}\UTFT{24F82}%
\UTFT{24D13}\UTFT{263F5}\UTFT{26911}\UTFT{2690E}\UTFT{26F9F}\UTFT{2509D}\UTFT{2517D}\UTFT{21E1C}\UTFT{25220}\UTFT{232AC}%
\UTFT{28964}\UTFT{28968}\UTFT{216C1}\UTFT{255E0}\UTFT{2760C}\UTFT{2261C}\UTFT{25857}\UTFT{27B39}\UTFT{27126}\UTFT{2910D}%
\UTFT{20C42}\UTFT{20D15}\UTFT{2512B}\UTFT{22CC6}\UTFT{20341}\UTFT{24DB8}\UTFT{294E5}\UTFT{280BE}\UTFT{22C38}\UTFT{2815D}%
\UTFT{269F2}\UTFT{24DEA}\UTFT{20D7C}\UTFT{20FB4}\UTFT{20CD5}\UTFT{2BAB3}\UTFT{20E96}\UTFT{20F64}\UTFT{22CA9}\UTFT{28256}%
\UTFT{244D3}\UTFT{20D46}\UTFT{29A4D}\UTFT{280E9}\UTFT{24EA7}\UTFT{22CC2}\UTFT{295F4}\UTFT{252C7}\UTFT{297D4}\UTFT{22D44}%
\UTFT{2BCD7}\UTFT{22BCA}\UTFT{2B977}\UTFT{266DA}\UTFT{26716}\UTFT{279A0}\UTFT{25052}\UTFT{20C43}\UTFT{28B4C}\UTFT{20731}%
\UTFT{201A9}\UTFT{22D8D}\UTFT{245C8}\UTFT{204FC}\UTFT{26097}\UTFT{20F4C}\UTFT{22A66}\UTFT{2109D}\UTFT{20D9C}\UTFT{22775}%
\UTFT{2A601}\UTFT{20E09}\UTFT{22ACF}\UTFT{2C5F8}\UTFT{210C8}\UTFT{239C2}\UTFT{2829B}\UTFT{25E49}\UTFT{220C7}\UTFT{22CB2}%
\UTFT{29720}\UTFT{24E3B}\UTFT{2C9A0}\UTFT{27574}\UTFT{22E8B}\UTFT{22208}\UTFT{2A65B}\UTFT{28CCD}\UTFT{20E7A}\UTFT{20C34}%
\UTFT{27639}\UTFT{22BCE}\UTFT{22C51}\UTFT{210C7}\UTFT{2A632}\UTFT{28CD2}\UTFT{28D99}\UTFT{28CCA}\UTFT{2775E}\UTFT{2F828}%
\UTFT{2107B}\UTFT{210D3}\UTFT{212FE}\UTFT{247EF}\UTFT{24EA5}\UTFT{24F5C}\UTFT{28189}\UTFT{2B42C}

Adobe-CNS1-3\\
\UTFT{2010C}\UTFT{200D1}\UTFT{200CD}\UTFT{200CB}\UTFT{21FE8}\UTFT{200CA}\UTFT{2010E}\UTFT{21BC1}\UTFT{2F878}\UTFT{20086}%
\UTFT{248E9}\UTFT{2626A}\UTFT{2634B}\UTFT{26612}\UTFT{26951}\UTFT{278B2}\UTFT{28E0F}\UTFT{29810}\UTFT{20087}\UTFT{2A3A9}%
\UTFT{21145}\UTFT{27735}\UTFT{209E7}\UTFT{29DF6}\UTFT{2700E}\UTFT{2A133}\UTFT{2846C}\UTFT{21DCA}\UTFT{205D0}\UTFT{22AE6}%
\UTFT{27D84}\UTFT{210F4}\UTFT{20C0B}\UTFT{278C8}\UTFT{260A5}\UTFT{22D4C}\UTFT{21077}\UTFT{2106F}\UTFT{221A1}\UTFT{20D96}%
\UTFT{22CC9}\UTFT{20F31}\UTFT{2681C}\UTFT{210CF}\UTFT{22803}\UTFT{22939}\UTFT{251E3}\UTFT{20E8C}\UTFT{20F8D}\UTFT{20EAA}%
\UTFT{20F30}\UTFT{20D47}\UTFT{2114F}\UTFT{20E4C}\UTFT{20EAB}\UTFT{20BA9}\UTFT{20D48}\UTFT{210C0}\UTFT{2113D}\UTFT{22696}%
\UTFT{20FAD}\UTFT{233F4}\UTFT{20D7E}\UTFT{20D7F}\UTFT{22C55}\UTFT{20E98}\UTFT{20F2E}\UTFT{26B50}\UTFT{29EC3}\UTFT{22DEE}%
\UTFT{26572}\UTFT{280BD}\UTFT{20EFA}\UTFT{20E0F}\UTFT{20E77}\UTFT{20EFB}\UTFT{24DEB}\UTFT{20CD6}\UTFT{227B5}\UTFT{210C9}%
\UTFT{20E10}\UTFT{20E78}\UTFT{21078}\UTFT{21148}\UTFT{28207}\UTFT{21455}\UTFT{20E79}\UTFT{24E50}\UTFT{22DA4}\UTFT{2101D}%
\UTFT{2101E}\UTFT{210F5}\UTFT{210F6}\UTFT{20E11}\UTFT{27694}\UTFT{282CD}\UTFT{20FB5}\UTFT{20E7B}\UTFT{2517E}\UTFT{20FB6}%
\UTFT{21180}\UTFT{252D8}\UTFT{2A2BD}\UTFT{249DA}\UTFT{2183A}\UTFT{24177}\UTFT{2827C}\UTFT{2573D}\UTFT{25B74}\UTFT{2313D}%
\UTFT{289C0}\UTFT{23F41}\UTFT{20325}\UTFT{20ED8}\UTFT{25C65}\UTFT{24FB8}\UTFT{20B0D}\UTFT{26B0A}\UTFT{22EEF}\UTFT{23CB5}%
\UTFT{26E99}\UTFT{23F8F}\UTFT{24CC9}\UTFT{2A014}\UTFT{286BC}\UTFT{28501}\UTFT{2267A}\UTFT{269A8}\UTFT{2424B}\UTFT{2215B}%
\UTFT{2037F}\UTFT{2A45B}\UTFT{249EC}\UTFT{24962}\UTFT{27109}\UTFT{24A4F}\UTFT{24A5D}\UTFT{217DF}\UTFT{23AFA}\UTFT{20214}%
\UTFT{208D5}\UTFT{20619}\UTFT{21F9E}\UTFT{2A2B6}\UTFT{2915B}\UTFT{28A59}\UTFT{29420}\UTFT{248F2}\UTFT{25535}\UTFT{20CCF}%
\UTFT{27967}\UTFT{21BC2}\UTFT{20094}\UTFT{202B7}\UTFT{203A0}\UTFT{204D7}\UTFT{205D5}\UTFT{20615}\UTFT{20676}\UTFT{216BA}%
\UTFT{20AC2}\UTFT{20ACD}\UTFT{20BBF}\UTFT{2F83B}\UTFT{20BCB}\UTFT{20BFB}\UTFT{20C3B}\UTFT{20C53}\UTFT{20C65}\UTFT{20C7C}%
\UTFT{20C8D}\UTFT{20CB5}\UTFT{20CDD}\UTFT{20CED}\UTFT{20D6F}\UTFT{20DB2}\UTFT{20DC8}\UTFT{20E04}\UTFT{20E0E}\UTFT{20ED7}%
\UTFT{20F90}\UTFT{20F2D}\UTFT{20E73}\UTFT{20FBC}\UTFT{2105C}\UTFT{2104F}\UTFT{21076}\UTFT{21088}\UTFT{21096}\UTFT{210BF}%
\UTFT{2112F}\UTFT{2113B}\UTFT{212E3}\UTFT{21375}\UTFT{21336}\UTFT{21577}\UTFT{21619}\UTFT{217C3}\UTFT{217C7}\UTFT{2182D}%
\UTFT{2196A}\UTFT{21A2D}\UTFT{21A45}\UTFT{21C2A}\UTFT{21C70}\UTFT{21CAC}\UTFT{21EC8}\UTFT{21ED5}\UTFT{21F15}\UTFT{22045}%
\UTFT{2227C}\UTFT{223D7}\UTFT{223FA}\UTFT{2272A}\UTFT{22871}\UTFT{2294F}\UTFT{22967}\UTFT{22993}\UTFT{22AD5}\UTFT{22AE8}%
\UTFT{22B0E}\UTFT{22B3F}\UTFT{22C4C}\UTFT{22C88}\UTFT{22CB7}\UTFT{25BE8}\UTFT{22D08}\UTFT{22D12}\UTFT{22DB7}\UTFT{22D95}%
\UTFT{22E42}\UTFT{22F74}\UTFT{22FCC}\UTFT{23033}\UTFT{23066}\UTFT{2331F}\UTFT{233DE}\UTFT{23567}\UTFT{235F3}\UTFT{2361A}%
\UTFT{23716}\UTFT{23AA7}\UTFT{23E11}\UTFT{23EB9}\UTFT{24119}\UTFT{242EE}\UTFT{2430D}\UTFT{24334}\UTFT{24396}\UTFT{24404}%
\UTFT{244D6}\UTFT{24674}\UTFT{2472F}\UTFT{24812}\UTFT{248FB}\UTFT{24A15}\UTFT{24AC0}\UTFT{24F86}\UTFT{2502C}\UTFT{25299}%
\UTFT{25419}\UTFT{25446}\UTFT{2546E}\UTFT{2553F}\UTFT{2555E}\UTFT{25562}\UTFT{25566}\UTFT{257C7}\UTFT{2585D}\UTFT{25903}%
\UTFT{25AAE}\UTFT{25B89}\UTFT{25C06}\UTFT{26102}\UTFT{261B2}\UTFT{26402}\UTFT{2644A}\UTFT{26484}\UTFT{26488}\UTFT{26512}%
\UTFT{265BF}\UTFT{266B5}\UTFT{266FC}\UTFT{26799}\UTFT{2686E}\UTFT{2685E}\UTFT{268C7}\UTFT{26926}\UTFT{26939}\UTFT{269FA}%
\UTFT{26A2D}\UTFT{26A34}\UTFT{26B5B}\UTFT{26B9D}\UTFT{26CA4}\UTFT{26DAE}\UTFT{2704B}\UTFT{271CD}\UTFT{27280}\UTFT{27285}%
\UTFT{2728B}\UTFT{272E6}\UTFT{27450}\UTFT{277CC}\UTFT{27858}\UTFT{279DD}\UTFT{279FD}\UTFT{27A0A}\UTFT{27B0B}\UTFT{27D66}%
\UTFT{28009}\UTFT{28023}\UTFT{28048}\UTFT{28083}\UTFT{28090}\UTFT{280F4}\UTFT{2812E}\UTFT{2814F}\UTFT{281AF}\UTFT{2821A}%
\UTFT{28306}\UTFT{2832F}\UTFT{2838A}\UTFT{28468}\UTFT{286AA}\UTFT{28956}\UTFT{289B8}\UTFT{289E7}\UTFT{289E8}\UTFT{28B46}%
\UTFT{28BD4}\UTFT{28C09}\UTFT{28FC5}\UTFT{290EC}\UTFT{29110}\UTFT{2913C}\UTFT{2915E}\UTFT{24ACA}\UTFT{294E7}\UTFT{295B0}%
\UTFT{295B8}\UTFT{29732}\UTFT{298D1}\UTFT{29949}\UTFT{2996A}\UTFT{299C3}\UTFT{29A28}\UTFT{29B0E}\UTFT{29D5A}\UTFT{29D9B}%
\UTFT{29EF8}\UTFT{29F23}\UTFT{2A293}\UTFT{2A2FF}\UTFT{2A5CB}\UTFT{20C9C}\UTFT{224B0}\UTFT{24A93}\UTFT{28B2C}\UTFT{217F5}%
\UTFT{28B6C}\UTFT{28B99}\UTFT{266AF}\UTFT{27655}\UTFT{25635}\UTFT{25956}\UTFT{25E81}\UTFT{20E6D}\UTFT{23E88}\UTFT{24C9E}%
\UTFT{217F6}\UTFT{2187B}\UTFT{25C4A}\UTFT{25311}\UTFT{25ED8}\UTFT{20FEA}\UTFT{20D49}\UTFT{236BA}\UTFT{25148}\UTFT{210C1}%
\UTFT{24706}\UTFT{26893}\UTFT{226F4}\UTFT{27D2F}\UTFT{241A3}\UTFT{27D73}\UTFT{26ED0}\UTFT{272B6}\UTFT{211D9}\UTFT{23CFC}%
\UTFT{2A6A9}\UTFT{20EAC}\UTFT{21CA2}\UTFT{24FC2}\UTFT{20FEB}\UTFT{22DA0}\UTFT{20FEC}\UTFT{20E0A}\UTFT{20FED}\UTFT{21187}%
\UTFT{24B6E}\UTFT{25A95}\UTFT{20979}\UTFT{22465}\UTFT{23CFE}\UTFT{29F30}\UTFT{24FA9}\UTFT{2959E}\UTFT{23DB6}\UTFT{267B3}%
\UTFT{23720}\UTFT{23EF7}\UTFT{23E2C}\UTFT{230DA}\UTFT{212A9}\UTFT{24963}\UTFT{270AE}\UTFT{2176C}\UTFT{27164}\UTFT{26D22}%
\UTFT{24AE2}\UTFT{2493E}\UTFT{26D23}\UTFT{203FC}\UTFT{23CFD}\UTFT{24919}\UTFT{24A77}\UTFT{28A5A}\UTFT{2F840}\UTFT{2183B}%
\UTFT{26159}\UTFT{233F5}\UTFT{28BC2}\UTFT{21D46}\UTFT{26ED1}\UTFT{28B2D}\UTFT{23CC7}\UTFT{25ED7}\UTFT{27656}\UTFT{25531}%
\UTFT{21944}\UTFT{29903}\UTFT{26DDC}\UTFT{270AD}\UTFT{261AD}\UTFT{28A0F}\UTFT{23677}\UTFT{200EE}\UTFT{26846}\UTFT{24F0E}%
\UTFT{2634C}\UTFT{2626B}\UTFT{21877}\UTFT{2408C}\UTFT{2307E}\UTFT{21E3D}\UTFT{203B5}\UTFT{205C3}\UTFT{21376}\UTFT{24A12}%
\UTFT{28B2B}\UTFT{26083}

Adobe-CNS1-4\\
\UTFT{29C73}\UTFT{2414E}\UTFT{251CD}\UTFT{25D30}\UTFT{28A32}\UTFT{23281}\UTFT{2A107}\UTFT{21980}\UTFT{2870F}\UTFT{2A2BA}%
\UTFT{29947}\UTFT{28AEA}\UTFT{2207E}\UTFT{289E3}\UTFT{21DB6}\UTFT{22712}\UTFT{233F9}\UTFT{23C63}\UTFT{24505}\UTFT{24A13}%
\UTFT{25CA4}\UTFT{25695}\UTFT{28DB9}\UTFT{2143F}\UTFT{2497B}\UTFT{2710D}\UTFT{26D74}\UTFT{26B15}\UTFT{26FBE}

Adobe-CNS1-5\\
\UTFT{27267}\UTFT{27CB1}\UTFT{27CC5}\UTFT{242BF}\UTFT{23617}\UTFT{27352}\UTFT{26E8B}\UTFT{270D2}\UTFT{2A351}\UTFT{27C6C}%
\UTFT{26B23}\UTFT{25A54}\UTFT{21A63}\UTFT{23E06}\UTFT{23F61}\UTFT{28BB9}\UTFT{27BEF}\UTFT{21D5E}\UTFT{29EB0}\UTFT{29945}%
\UTFT{20A6F}\UTFT{23256}\UTFT{22796}\UTFT{23B1A}\UTFT{23551}\UTFT{240EC}\UTFT{21E23}\UTFT{201A4}\UTFT{26C41}\UTFT{20239}%
\UTFT{298FA}\UTFT{20B9F}\UTFT{221C1}\UTFT{2896D}\UTFT{29079}\UTFT{2A1B5}\UTFT{26C46}\UTFT{286B2}\UTFT{273FF}\UTFT{2549A}%
\UTFT{24B0F}

Adobe-CNS1-6\\
\UTFT{21D53}\UTFT{2369E}\UTFT{26021}\UTFT{258DE}\UTFT{24161}\UTFT{2890D}\UTFT{231EA}\UTFT{20A8A}\UTFT{2325E}\UTFT{25DB9}%
\UTFT{2368E}\UTFT{27B65}\UTFT{26E88}\UTFT{25D99}\UTFT{224BC}\UTFT{224C1}\UTFT{224C9}\UTFT{224CC}\UTFT{235BB}\UTFT{2ADFF}%


% end


%
% This file is generated from the data of UniGB-UTF32
% in cid2code.txt (Version 12/05/2017)
% for Adobe-GB1-5
%
% Reference:
%   https://github.com/adobe-type-tools/cmap-resources/
%   Adobe-GB1-5/cid2code.txt
%
% A newer CMap may be required for some code points.
%


Adobe-GB1-2\\
\UTFC{20087}\UTFC{20089}\UTFC{200CC}\UTFC{215D7}\UTFC{2298F}\UTFC{20509}\UTFC{2099D}\UTFC{241FE}

% end


%\end{document}

{\bfseries%
[mc/bx]

%
% This file is generated from the data of UniCNS-UTF32
% in cid2code.txt (Version 12/04/2015)
% for Adobe-CNS1-6
%
% Reference:
%   https://github.com/adobe-type-tools/cmap-resources/
%   cmapresources_cns1-6/cid2code.txt
%
% A newer CMap may be required for some code points.
%


Adobe-CNS1-0\\
\UTFT{200CC}\UTFT{2008A}\UTFT{27607}

Adobe-CNS1-1\\
\UTFT{23ED7}\UTFT{26ED3}\UTFT{257E0}\UTFT{28BE9}\UTFT{258E1}\UTFT{294D9}\UTFT{259AC}\UTFT{2648D}\UTFT{25C01}\UTFT{2530E}%
\UTFT{25CFE}\UTFT{25BB4}\UTFT{26C7F}\UTFT{25D20}\UTFT{25CC1}\UTFT{24882}\UTFT{24578}\UTFT{26E44}\UTFT{26ED6}\UTFT{24057}%
\UTFT{26029}\UTFT{217F9}\UTFT{2836D}\UTFT{26121}\UTFT{2615A}\UTFT{262D0}\UTFT{26351}\UTFT{21661}\UTFT{20068}\UTFT{23766}%
\UTFT{2833A}\UTFT{26489}\UTFT{2A087}\UTFT{26CC3}\UTFT{22714}\UTFT{26626}\UTFT{23DE3}\UTFT{266E8}\UTFT{28A48}\UTFT{226F6}%
\UTFT{26498}\UTFT{2148A}\UTFT{2185E}\UTFT{24A65}\UTFT{24A95}\UTFT{26A52}\UTFT{23D7E}\UTFT{214FD}\UTFT{2F98F}\UTFT{249A7}%
\UTFT{23530}\UTFT{21773}\UTFT{23DF8}\UTFT{2F994}\UTFT{20E16}\UTFT{217B4}\UTFT{2317D}\UTFT{2355A}\UTFT{23E8B}\UTFT{26DA3}%
\UTFT{26B05}\UTFT{26B97}\UTFT{235CE}\UTFT{26DA5}\UTFT{26ED4}\UTFT{26E42}\UTFT{25BE4}\UTFT{26B96}\UTFT{26E77}\UTFT{26E43}%
\UTFT{25C91}\UTFT{25CC0}\UTFT{28625}\UTFT{2863B}\UTFT{27088}\UTFT{21582}\UTFT{270CD}\UTFT{2F9B2}\UTFT{218A2}\UTFT{2739A}%
\UTFT{2A0F8}\UTFT{22C27}\UTFT{275E0}\UTFT{23DB9}\UTFT{275E4}\UTFT{2770F}\UTFT{28A25}\UTFT{27924}\UTFT{27ABD}\UTFT{27A59}%
\UTFT{27B3A}\UTFT{27B38}\UTFT{25430}\UTFT{25565}\UTFT{24A7A}\UTFT{216DF}\UTFT{27D54}\UTFT{27D8F}\UTFT{2F9D4}\UTFT{27D53}%
\UTFT{27D98}\UTFT{27DBD}\UTFT{21910}\UTFT{2F9D7}\UTFT{28002}\UTFT{21014}\UTFT{2498A}\UTFT{281BC}\UTFT{2710C}\UTFT{28365}%
\UTFT{28412}\UTFT{2A29F}\UTFT{20A50}\UTFT{289DE}\UTFT{2853D}\UTFT{23DBB}\UTFT{23262}\UTFT{22325}\UTFT{26ED7}\UTFT{2853C}%
\UTFT{27ABE}\UTFT{2856C}\UTFT{2860B}\UTFT{28713}\UTFT{286E6}\UTFT{28933}\UTFT{21E89}\UTFT{255B9}\UTFT{28AC6}\UTFT{23C9B}%
\UTFT{28B0C}\UTFT{255DB}\UTFT{20D31}\UTFT{28AE1}\UTFT{28BEB}\UTFT{28AE2}\UTFT{28AE5}\UTFT{28BEC}\UTFT{28C39}\UTFT{28BFF}%
\UTFT{286D8}\UTFT{2127C}\UTFT{23E2E}\UTFT{26ED5}\UTFT{28AE0}\UTFT{26CB8}\UTFT{20274}\UTFT{26410}\UTFT{290AF}\UTFT{290E5}%
\UTFT{24AD1}\UTFT{21915}\UTFT{2330A}\UTFT{24AE9}\UTFT{291D5}\UTFT{291EB}\UTFT{230B7}\UTFT{230BC}\UTFT{2546C}\UTFT{29433}%
\UTFT{2941D}\UTFT{2797A}\UTFT{27175}\UTFT{20630}\UTFT{2415C}\UTFT{25706}\UTFT{26D27}\UTFT{216D3}\UTFT{24A29}\UTFT{29857}%
\UTFT{29905}\UTFT{25725}\UTFT{290B1}\UTFT{29BD5}\UTFT{29B05}\UTFT{28600}\UTFT{2307D}\UTFT{29D3E}\UTFT{21863}\UTFT{29E68}%
\UTFT{29FB7}\UTFT{2A192}\UTFT{2A1AB}\UTFT{2A0E1}\UTFT{2A123}\UTFT{2A1DF}\UTFT{2A134}\UTFT{2A193}\UTFT{2A220}\UTFT{2193B}%
\UTFT{2A233}\UTFT{2A0B9}\UTFT{2A2B4}\UTFT{24364}\UTFT{28C2B}\UTFT{26DA2}\UTFT{2FA1B}\UTFT{2908B}\UTFT{24975}\UTFT{249BB}%
\UTFT{249F8}\UTFT{24348}\UTFT{24A51}\UTFT{28BDA}\UTFT{218FA}\UTFT{2897E}\UTFT{28E36}\UTFT{28A44}\UTFT{2896C}\UTFT{244B9}%
\UTFT{24473}\UTFT{243F8}\UTFT{217EF}\UTFT{218BE}\UTFT{23599}\UTFT{21885}\UTFT{2542F}\UTFT{217F8}\UTFT{216FB}\UTFT{21839}%
\UTFT{21774}\UTFT{218D1}\UTFT{25F4B}\UTFT{216C0}\UTFT{24A25}\UTFT{213FE}\UTFT{212A8}\UTFT{213C6}\UTFT{214B6}\UTFT{236A6}%
\UTFT{24994}\UTFT{27165}\UTFT{23E31}\UTFT{2555C}\UTFT{23EFB}\UTFT{27052}\UTFT{236EE}\UTFT{2999D}\UTFT{26F26}\UTFT{21922}%
\UTFT{2373F}\UTFT{240E1}\UTFT{2408B}\UTFT{2410F}\UTFT{26C21}\UTFT{266B1}\UTFT{20FDF}\UTFT{20BA8}\UTFT{20E0D}\UTFT{28B13}%
\UTFT{24436}\UTFT{20465}\UTFT{25651}\UTFT{201AB}\UTFT{203CB}\UTFT{2030A}\UTFT{20414}\UTFT{202C0}\UTFT{28EB3}\UTFT{20275}%
\UTFT{2020C}\UTFT{24A0E}\UTFT{23E8A}\UTFT{23595}\UTFT{23E39}\UTFT{23EBF}\UTFT{21884}\UTFT{23E89}\UTFT{205E0}\UTFT{204A3}%
\UTFT{20492}\UTFT{20491}\UTFT{28A9C}\UTFT{2070E}\UTFT{20873}\UTFT{2438C}\UTFT{20C20}\UTFT{249AC}\UTFT{210E4}\UTFT{20E1D}%
\UTFT{24ABC}\UTFT{2408D}\UTFT{240C9}\UTFT{20345}\UTFT{20BC6}\UTFT{28A46}\UTFT{216FA}\UTFT{2176F}\UTFT{21710}\UTFT{25946}%
\UTFT{219F3}\UTFT{21861}\UTFT{24295}\UTFT{25E83}\UTFT{28BD7}\UTFT{20413}\UTFT{21303}\UTFT{289FB}\UTFT{21996}\UTFT{2197C}%
\UTFT{23AEE}\UTFT{21903}\UTFT{21904}\UTFT{218A0}\UTFT{216FE}\UTFT{28A47}\UTFT{21DBA}\UTFT{23472}\UTFT{289A8}\UTFT{21927}%
\UTFT{217AB}\UTFT{2173B}\UTFT{275FD}\UTFT{22860}\UTFT{2262B}\UTFT{225AF}\UTFT{225BE}\UTFT{29088}\UTFT{26F73}\UTFT{2003E}%
\UTFT{20046}\UTFT{2261B}\UTFT{22C9B}\UTFT{22D07}\UTFT{246D4}\UTFT{2914D}\UTFT{24665}\UTFT{22B6A}\UTFT{22B22}\UTFT{23450}%
\UTFT{298EA}\UTFT{22E78}\UTFT{249E3}\UTFT{22D67}\UTFT{22CA1}\UTFT{2308E}\UTFT{232AD}\UTFT{24989}\UTFT{232AB}\UTFT{232E0}%
\UTFT{218D9}\UTFT{2943F}\UTFT{23289}\UTFT{231B3}\UTFT{25584}\UTFT{28B22}\UTFT{2558F}\UTFT{216FC}\UTFT{2555B}\UTFT{25425}%
\UTFT{23103}\UTFT{2182A}\UTFT{23234}\UTFT{2320F}\UTFT{23182}\UTFT{242C9}\UTFT{26D24}\UTFT{27870}\UTFT{21DEB}\UTFT{232D2}%
\UTFT{232E1}\UTFT{25872}\UTFT{2383A}\UTFT{237BC}\UTFT{237A2}\UTFT{233FE}\UTFT{2462A}\UTFT{237D5}\UTFT{24487}\UTFT{21912}%
\UTFT{23FC0}\UTFT{23C9A}\UTFT{28BEA}\UTFT{28ACB}\UTFT{2801E}\UTFT{289DC}\UTFT{23F7F}\UTFT{2403C}\UTFT{2431A}\UTFT{24276}%
\UTFT{2478F}\UTFT{24725}\UTFT{24AA4}\UTFT{205EB}\UTFT{23EF8}\UTFT{2365F}\UTFT{24A4A}\UTFT{24917}\UTFT{25FE1}\UTFT{24ADF}%
\UTFT{28C23}\UTFT{23F35}\UTFT{26DEA}\UTFT{24CD9}\UTFT{24D06}\UTFT{2A5C6}\UTFT{28ACC}\UTFT{249AB}\UTFT{2498E}\UTFT{24A4E}%
\UTFT{249C5}\UTFT{248F3}\UTFT{28AE3}\UTFT{21864}\UTFT{25221}\UTFT{251E7}\UTFT{23232}\UTFT{24697}\UTFT{23781}\UTFT{248F0}%
\UTFT{24ABA}\UTFT{24AC7}\UTFT{24A96}\UTFT{261AE}\UTFT{25581}\UTFT{27741}\UTFT{256E3}\UTFT{23EFA}\UTFT{216E6}\UTFT{20D4C}%
\UTFT{2498C}\UTFT{20299}\UTFT{23DBA}\UTFT{2176E}\UTFT{201D4}\UTFT{20C0D}\UTFT{226F5}\UTFT{25AAF}\UTFT{25A9C}\UTFT{2025B}%
\UTFT{25BC6}\UTFT{25BB3}\UTFT{25EBC}\UTFT{25EA6}\UTFT{249F9}\UTFT{217B0}\UTFT{26261}\UTFT{2615C}\UTFT{27B48}\UTFT{25E82}%
\UTFT{26B75}\UTFT{20916}\UTFT{2004E}\UTFT{235CF}\UTFT{26412}\UTFT{263F8}\UTFT{2082C}\UTFT{25AE9}\UTFT{25D43}\UTFT{25E0E}%
\UTFT{2343F}\UTFT{249F7}\UTFT{265AD}\UTFT{265A0}\UTFT{27127}\UTFT{26CD1}\UTFT{267B4}\UTFT{26A42}\UTFT{26A51}\UTFT{26DA7}%
\UTFT{2721B}\UTFT{21840}\UTFT{218A1}\UTFT{218D8}\UTFT{2F9BC}\UTFT{23D8F}\UTFT{27422}\UTFT{25683}\UTFT{27785}\UTFT{27784}%
\UTFT{28BF5}\UTFT{28BD9}\UTFT{28B9C}\UTFT{289F9}\UTFT{29448}\UTFT{24284}\UTFT{21845}\UTFT{27DDC}\UTFT{24C09}\UTFT{22321}%
\UTFT{217DA}\UTFT{2492F}\UTFT{28A4B}\UTFT{28AFC}\UTFT{28C1D}\UTFT{28C3B}\UTFT{28D34}\UTFT{248FF}\UTFT{24A42}\UTFT{243EA}%
\UTFT{23225}\UTFT{28EE7}\UTFT{28E66}\UTFT{28E65}\UTFT{249ED}\UTFT{24A78}\UTFT{23FEE}\UTFT{290B0}\UTFT{29093}\UTFT{257DF}%
\UTFT{28989}\UTFT{28C26}\UTFT{28B2F}\UTFT{263BE}\UTFT{2421B}\UTFT{20F26}\UTFT{28BC5}\UTFT{24AB2}\UTFT{294DA}\UTFT{295D7}%
\UTFT{28B50}\UTFT{24A67}\UTFT{28B64}\UTFT{28A45}\UTFT{27B06}\UTFT{28B65}\UTFT{258C8}\UTFT{298F1}\UTFT{29948}\UTFT{21302}%
\UTFT{249B8}\UTFT{214E8}\UTFT{2271F}\UTFT{23DB8}\UTFT{22781}\UTFT{2296B}\UTFT{29E2D}\UTFT{2A1F5}\UTFT{2A0FE}\UTFT{24104}%
\UTFT{2A1B4}\UTFT{2A0ED}\UTFT{2A0F3}\UTFT{2992F}\UTFT{26E12}\UTFT{26FDF}\UTFT{26B82}\UTFT{26DA4}\UTFT{26E84}\UTFT{26DF0}%
\UTFT{26E00}\UTFT{237D7}\UTFT{26064}\UTFT{2359C}\UTFT{23640}\UTFT{249DE}\UTFT{202BF}\UTFT{2555D}\UTFT{21757}\UTFT{231C9}%
\UTFT{24941}\UTFT{241B5}\UTFT{241AC}\UTFT{26C40}\UTFT{24F97}\UTFT{217B5}\UTFT{28A49}\UTFT{24488}\UTFT{289FC}\UTFT{218D6}%
\UTFT{20F1D}\UTFT{26CC0}\UTFT{21413}\UTFT{242FA}\UTFT{22C26}\UTFT{243C1}\UTFT{23DB7}\UTFT{26741}\UTFT{2615B}\UTFT{260A4}%
\UTFT{249B9}\UTFT{2498B}\UTFT{289FA}\UTFT{28B63}\UTFT{2189F}\UTFT{24AB3}\UTFT{24A3E}\UTFT{24A94}\UTFT{217D9}\UTFT{24A66}%
\UTFT{203A7}\UTFT{21424}\UTFT{249E5}\UTFT{24916}\UTFT{24976}\UTFT{204FE}\UTFT{28ACE}\UTFT{28A16}\UTFT{28BE7}\UTFT{255D5}%
\UTFT{28A82}\UTFT{24943}\UTFT{20CFF}\UTFT{2061A}\UTFT{20BEB}\UTFT{20CB8}\UTFT{217FA}\UTFT{216C2}\UTFT{24A50}\UTFT{21852}%
\UTFT{28AC0}\UTFT{249AD}\UTFT{218BF}\UTFT{21883}\UTFT{27484}\UTFT{23D5B}\UTFT{28A81}\UTFT{21862}\UTFT{20AB4}\UTFT{2139C}%
\UTFT{28218}\UTFT{290E4}\UTFT{27E4F}\UTFT{23FED}\UTFT{23E2D}\UTFT{203F5}\UTFT{28C1C}\UTFT{26BC0}\UTFT{21452}\UTFT{24362}%
\UTFT{24A71}\UTFT{22FE3}\UTFT{212B0}\UTFT{223BD}\UTFT{21398}\UTFT{234E5}\UTFT{27BF4}\UTFT{236DF}\UTFT{28A83}\UTFT{237D6}%
\UTFT{233FA}\UTFT{24C9F}\UTFT{236AD}\UTFT{26CB7}\UTFT{26D26}\UTFT{26D51}\UTFT{26C82}\UTFT{26FDE}\UTFT{2173A}\UTFT{26C80}%
\UTFT{27053}\UTFT{217DB}\UTFT{217B3}\UTFT{21905}\UTFT{241FC}\UTFT{2173C}\UTFT{242A5}\UTFT{24293}\UTFT{23EF9}\UTFT{27736}%
\UTFT{2445B}\UTFT{242CA}\UTFT{24259}\UTFT{289E1}\UTFT{26D28}\UTFT{244CE}\UTFT{27E4D}\UTFT{243BD}\UTFT{24256}\UTFT{21304}%
\UTFT{243E9}\UTFT{2F825}\UTFT{23300}\UTFT{27AF4}\UTFT{256F6}\UTFT{27B18}\UTFT{27A79}\UTFT{249BA}\UTFT{20346}\UTFT{27657}%
\UTFT{25FE2}\UTFT{275FE}\UTFT{2209A}\UTFT{28A9A}\UTFT{2403B}\UTFT{24A45}\UTFT{205CA}\UTFT{20611}\UTFT{21EA8}\UTFT{23CFF}%
\UTFT{285E8}\UTFT{299C9}\UTFT{221C3}\UTFT{28B4E}\UTFT{20C78}\UTFT{20779}\UTFT{23F4A}\UTFT{24AA7}\UTFT{26B52}\UTFT{27632}%
\UTFT{2493F}\UTFT{233CC}\UTFT{28948}\UTFT{21D90}\UTFT{27C12}\UTFT{24F9A}\UTFT{26BF7}\UTFT{2191C}\UTFT{249F6}\UTFT{23FEF}%
\UTFT{2271B}\UTFT{257E1}\UTFT{2F8CD}\UTFT{2F806}\UTFT{24521}\UTFT{24934}\UTFT{26CBD}\UTFT{26411}\UTFT{290C0}\UTFT{20A11}%
\UTFT{26469}\UTFT{20021}\UTFT{23519}\UTFT{2258D}\UTFT{2217A}\UTFT{249D0}\UTFT{20EF8}\UTFT{22926}\UTFT{28473}\UTFT{217B1}%
\UTFT{24A2A}\UTFT{21820}\UTFT{29CAD}\UTFT{298A4}\UTFT{2160A}\UTFT{2372F}\UTFT{280E8}\UTFT{213C5}\UTFT{291A8}\UTFT{270AF}%
\UTFT{289AB}\UTFT{2417A}\UTFT{2A2DF}\UTFT{28318}\UTFT{26E07}\UTFT{2816F}\UTFT{269B5}\UTFT{213ED}\UTFT{2322F}\UTFT{28C30}%
\UTFT{28949}\UTFT{24988}\UTFT{24AA5}\UTFT{23F81}\UTFT{21FA1}\UTFT{295E9}\UTFT{2789D}\UTFT{28024}\UTFT{27A3E}\UTFT{23CB7}%
\UTFT{26258}\UTFT{29D98}\UTFT{23D40}\UTFT{20E9D}\UTFT{282E2}\UTFT{20C41}\UTFT{20C96}\UTFT{20E76}\UTFT{22C62}\UTFT{20EA2}%
\UTFT{21075}\UTFT{22B43}\UTFT{22EB3}\UTFT{20DA7}\UTFT{2688A}\UTFT{20EF9}\UTFT{27FF9}\UTFT{247E0}\UTFT{29D7C}\UTFT{275A3}%
\UTFT{26048}\UTFT{24618}\UTFT{29EAC}\UTFT{29FDE}\UTFT{272B2}\UTFT{2048E}\UTFT{20EB6}\UTFT{27F2E}\UTFT{2A434}\UTFT{243F2}%
\UTFT{29E06}\UTFT{294D0}\UTFT{26335}\UTFT{20D28}\UTFT{20D71}\UTFT{21F0F}\UTFT{21DD1}\UTFT{2176D}\UTFT{2B473}\UTFT{28E97}%
\UTFT{25C21}\UTFT{20CD4}\UTFT{201F2}\UTFT{2A64A}\UTFT{2837D}\UTFT{2A2B2}\UTFT{24ABB}\UTFT{26E05}\UTFT{2AE67}\UTFT{2251B}%
\UTFT{28E39}\UTFT{20F3B}\UTFT{25F1A}\UTFT{27486}\UTFT{267CC}\UTFT{24011}\UTFT{2F922}\UTFT{20547}\UTFT{205DF}\UTFT{23FC5}%
\UTFT{24942}\UTFT{289E4}\UTFT{219DB}\UTFT{23CC8}\UTFT{24933}\UTFT{289AA}\UTFT{202A0}\UTFT{26BB3}\UTFT{21305}\UTFT{224ED}%
\UTFT{26D29}\UTFT{27A84}\UTFT{23600}\UTFT{24AB1}\UTFT{22513}\UTFT{2037E}\UTFT{20380}\UTFT{20347}\UTFT{2041F}\UTFT{249A4}%
\UTFT{20487}\UTFT{233B4}\UTFT{20BFF}\UTFT{220FC}\UTFT{202E5}\UTFT{22530}\UTFT{2058E}\UTFT{23233}\UTFT{21983}\UTFT{205B3}%
\UTFT{23C99}\UTFT{24AA6}\UTFT{2372D}\UTFT{26B13}\UTFT{2F829}\UTFT{28ADE}\UTFT{23F80}\UTFT{20954}\UTFT{23FEC}\UTFT{20BE2}%
\UTFT{21726}\UTFT{216E8}\UTFT{286AB}\UTFT{2F832}\UTFT{21596}\UTFT{21613}\UTFT{28A9B}\UTFT{25772}\UTFT{20B8F}\UTFT{23FEB}%
\UTFT{22DA3}\UTFT{20C77}\UTFT{26B53}\UTFT{20D74}\UTFT{2170D}\UTFT{20EDD}\UTFT{20D4D}\UTFT{289BC}\UTFT{22698}\UTFT{218D7}%
\UTFT{2403A}\UTFT{24435}\UTFT{210B4}\UTFT{2328A}\UTFT{28B66}\UTFT{2124F}\UTFT{241A5}\UTFT{26C7E}\UTFT{21416}\UTFT{21454}%
\UTFT{24363}\UTFT{24BF5}\UTFT{2123C}\UTFT{2A150}\UTFT{24278}\UTFT{2163E}\UTFT{21692}\UTFT{20D4E}\UTFT{26C81}\UTFT{26D2A}%
\UTFT{217DC}\UTFT{217FB}\UTFT{217B2}\UTFT{26DA6}\UTFT{21828}\UTFT{216D5}\UTFT{26E45}\UTFT{249A9}\UTFT{26FA1}\UTFT{22554}%
\UTFT{21911}\UTFT{216B8}\UTFT{27A0E}\UTFT{20204}\UTFT{21A34}\UTFT{259CC}\UTFT{205A5}\UTFT{21B44}\UTFT{21CA5}\UTFT{26B28}%
\UTFT{21DF9}\UTFT{21E37}\UTFT{21EA4}\UTFT{24901}\UTFT{22049}\UTFT{22173}\UTFT{244BC}\UTFT{20CD3}\UTFT{21771}\UTFT{28482}%
\UTFT{201C1}\UTFT{2F894}\UTFT{2133A}\UTFT{26888}\UTFT{223D0}\UTFT{22471}\UTFT{26E6E}\UTFT{28A36}\UTFT{25250}\UTFT{21F6A}%
\UTFT{270F8}\UTFT{22668}\UTFT{2029E}\UTFT{28A29}\UTFT{227B4}\UTFT{24982}\UTFT{2498F}\UTFT{27A53}\UTFT{2F8A6}\UTFT{26ED2}%
\UTFT{20656}\UTFT{23FB7}\UTFT{2285F}\UTFT{28B9D}\UTFT{2995D}\UTFT{22980}\UTFT{228C1}\UTFT{20118}\UTFT{21770}\UTFT{22E0D}%
\UTFT{249DF}\UTFT{2138E}\UTFT{217FC}\UTFT{22E36}\UTFT{2571D}\UTFT{24A28}\UTFT{24A23}\UTFT{24940}\UTFT{21829}\UTFT{23400}%
\UTFT{231F7}\UTFT{231F8}\UTFT{231A4}\UTFT{231A5}\UTFT{20E75}\UTFT{251E6}\UTFT{23231}\UTFT{285F4}\UTFT{231C8}\UTFT{25313}%
\UTFT{228F7}\UTFT{2439C}\UTFT{24A21}\UTFT{237C2}\UTFT{2F8DB}\UTFT{241CD}\UTFT{290ED}\UTFT{233E6}\UTFT{26DA0}\UTFT{2346F}%
\UTFT{28ADF}\UTFT{235CD}\UTFT{2363C}\UTFT{28A4A}\UTFT{203C9}\UTFT{23659}\UTFT{2212A}\UTFT{23703}\UTFT{2919C}\UTFT{20923}%
\UTFT{227CD}\UTFT{23ADB}\UTFT{21958}\UTFT{23B5A}\UTFT{23EFC}\UTFT{2248B}\UTFT{248F1}\UTFT{26B51}\UTFT{23DBC}\UTFT{23DBD}%
\UTFT{241A4}\UTFT{2490C}\UTFT{24900}\UTFT{23CC9}\UTFT{20D32}\UTFT{231F9}\UTFT{22491}\UTFT{26D25}\UTFT{26DA1}\UTFT{26DEB}%
\UTFT{2497F}\UTFT{24085}\UTFT{26E72}\UTFT{26F74}\UTFT{28B21}\UTFT{2F908}\UTFT{23E2F}\UTFT{23F82}\UTFT{2304B}\UTFT{23E30}%
\UTFT{21497}\UTFT{2403D}\UTFT{29170}\UTFT{24144}\UTFT{24091}\UTFT{24155}\UTFT{24039}\UTFT{23FF0}\UTFT{23FB4}\UTFT{2413F}%
\UTFT{24156}\UTFT{24157}\UTFT{24140}\UTFT{261DD}\UTFT{24277}\UTFT{24365}\UTFT{242C1}\UTFT{2445A}\UTFT{24A27}\UTFT{24A22}%
\UTFT{28BE8}\UTFT{25605}\UTFT{24974}\UTFT{23044}\UTFT{24823}\UTFT{2882B}\UTFT{28804}\UTFT{20C3A}\UTFT{26A2E}\UTFT{241E2}%
\UTFT{216E7}\UTFT{24A24}\UTFT{249B7}\UTFT{2498D}\UTFT{249FB}\UTFT{24A26}\UTFT{2F92F}\UTFT{228AD}\UTFT{28EB2}\UTFT{24A8C}%
\UTFT{2415F}\UTFT{24A79}\UTFT{28B8F}\UTFT{28C03}\UTFT{2189E}\UTFT{21988}\UTFT{28ED9}\UTFT{21A4B}\UTFT{28EAC}\UTFT{24F82}%
\UTFT{24D13}\UTFT{263F5}\UTFT{26911}\UTFT{2690E}\UTFT{26F9F}\UTFT{2509D}\UTFT{2517D}\UTFT{21E1C}\UTFT{25220}\UTFT{232AC}%
\UTFT{28964}\UTFT{28968}\UTFT{216C1}\UTFT{255E0}\UTFT{2760C}\UTFT{2261C}\UTFT{25857}\UTFT{27B39}\UTFT{27126}\UTFT{2910D}%
\UTFT{20C42}\UTFT{20D15}\UTFT{2512B}\UTFT{22CC6}\UTFT{20341}\UTFT{24DB8}\UTFT{294E5}\UTFT{280BE}\UTFT{22C38}\UTFT{2815D}%
\UTFT{269F2}\UTFT{24DEA}\UTFT{20D7C}\UTFT{20FB4}\UTFT{20CD5}\UTFT{2BAB3}\UTFT{20E96}\UTFT{20F64}\UTFT{22CA9}\UTFT{28256}%
\UTFT{244D3}\UTFT{20D46}\UTFT{29A4D}\UTFT{280E9}\UTFT{24EA7}\UTFT{22CC2}\UTFT{295F4}\UTFT{252C7}\UTFT{297D4}\UTFT{22D44}%
\UTFT{2BCD7}\UTFT{22BCA}\UTFT{2B977}\UTFT{266DA}\UTFT{26716}\UTFT{279A0}\UTFT{25052}\UTFT{20C43}\UTFT{28B4C}\UTFT{20731}%
\UTFT{201A9}\UTFT{22D8D}\UTFT{245C8}\UTFT{204FC}\UTFT{26097}\UTFT{20F4C}\UTFT{22A66}\UTFT{2109D}\UTFT{20D9C}\UTFT{22775}%
\UTFT{2A601}\UTFT{20E09}\UTFT{22ACF}\UTFT{2C5F8}\UTFT{210C8}\UTFT{239C2}\UTFT{2829B}\UTFT{25E49}\UTFT{220C7}\UTFT{22CB2}%
\UTFT{29720}\UTFT{24E3B}\UTFT{2C9A0}\UTFT{27574}\UTFT{22E8B}\UTFT{22208}\UTFT{2A65B}\UTFT{28CCD}\UTFT{20E7A}\UTFT{20C34}%
\UTFT{27639}\UTFT{22BCE}\UTFT{22C51}\UTFT{210C7}\UTFT{2A632}\UTFT{28CD2}\UTFT{28D99}\UTFT{28CCA}\UTFT{2775E}\UTFT{2F828}%
\UTFT{2107B}\UTFT{210D3}\UTFT{212FE}\UTFT{247EF}\UTFT{24EA5}\UTFT{24F5C}\UTFT{28189}\UTFT{2B42C}

Adobe-CNS1-3\\
\UTFT{2010C}\UTFT{200D1}\UTFT{200CD}\UTFT{200CB}\UTFT{21FE8}\UTFT{200CA}\UTFT{2010E}\UTFT{21BC1}\UTFT{2F878}\UTFT{20086}%
\UTFT{248E9}\UTFT{2626A}\UTFT{2634B}\UTFT{26612}\UTFT{26951}\UTFT{278B2}\UTFT{28E0F}\UTFT{29810}\UTFT{20087}\UTFT{2A3A9}%
\UTFT{21145}\UTFT{27735}\UTFT{209E7}\UTFT{29DF6}\UTFT{2700E}\UTFT{2A133}\UTFT{2846C}\UTFT{21DCA}\UTFT{205D0}\UTFT{22AE6}%
\UTFT{27D84}\UTFT{210F4}\UTFT{20C0B}\UTFT{278C8}\UTFT{260A5}\UTFT{22D4C}\UTFT{21077}\UTFT{2106F}\UTFT{221A1}\UTFT{20D96}%
\UTFT{22CC9}\UTFT{20F31}\UTFT{2681C}\UTFT{210CF}\UTFT{22803}\UTFT{22939}\UTFT{251E3}\UTFT{20E8C}\UTFT{20F8D}\UTFT{20EAA}%
\UTFT{20F30}\UTFT{20D47}\UTFT{2114F}\UTFT{20E4C}\UTFT{20EAB}\UTFT{20BA9}\UTFT{20D48}\UTFT{210C0}\UTFT{2113D}\UTFT{22696}%
\UTFT{20FAD}\UTFT{233F4}\UTFT{20D7E}\UTFT{20D7F}\UTFT{22C55}\UTFT{20E98}\UTFT{20F2E}\UTFT{26B50}\UTFT{29EC3}\UTFT{22DEE}%
\UTFT{26572}\UTFT{280BD}\UTFT{20EFA}\UTFT{20E0F}\UTFT{20E77}\UTFT{20EFB}\UTFT{24DEB}\UTFT{20CD6}\UTFT{227B5}\UTFT{210C9}%
\UTFT{20E10}\UTFT{20E78}\UTFT{21078}\UTFT{21148}\UTFT{28207}\UTFT{21455}\UTFT{20E79}\UTFT{24E50}\UTFT{22DA4}\UTFT{2101D}%
\UTFT{2101E}\UTFT{210F5}\UTFT{210F6}\UTFT{20E11}\UTFT{27694}\UTFT{282CD}\UTFT{20FB5}\UTFT{20E7B}\UTFT{2517E}\UTFT{20FB6}%
\UTFT{21180}\UTFT{252D8}\UTFT{2A2BD}\UTFT{249DA}\UTFT{2183A}\UTFT{24177}\UTFT{2827C}\UTFT{2573D}\UTFT{25B74}\UTFT{2313D}%
\UTFT{289C0}\UTFT{23F41}\UTFT{20325}\UTFT{20ED8}\UTFT{25C65}\UTFT{24FB8}\UTFT{20B0D}\UTFT{26B0A}\UTFT{22EEF}\UTFT{23CB5}%
\UTFT{26E99}\UTFT{23F8F}\UTFT{24CC9}\UTFT{2A014}\UTFT{286BC}\UTFT{28501}\UTFT{2267A}\UTFT{269A8}\UTFT{2424B}\UTFT{2215B}%
\UTFT{2037F}\UTFT{2A45B}\UTFT{249EC}\UTFT{24962}\UTFT{27109}\UTFT{24A4F}\UTFT{24A5D}\UTFT{217DF}\UTFT{23AFA}\UTFT{20214}%
\UTFT{208D5}\UTFT{20619}\UTFT{21F9E}\UTFT{2A2B6}\UTFT{2915B}\UTFT{28A59}\UTFT{29420}\UTFT{248F2}\UTFT{25535}\UTFT{20CCF}%
\UTFT{27967}\UTFT{21BC2}\UTFT{20094}\UTFT{202B7}\UTFT{203A0}\UTFT{204D7}\UTFT{205D5}\UTFT{20615}\UTFT{20676}\UTFT{216BA}%
\UTFT{20AC2}\UTFT{20ACD}\UTFT{20BBF}\UTFT{2F83B}\UTFT{20BCB}\UTFT{20BFB}\UTFT{20C3B}\UTFT{20C53}\UTFT{20C65}\UTFT{20C7C}%
\UTFT{20C8D}\UTFT{20CB5}\UTFT{20CDD}\UTFT{20CED}\UTFT{20D6F}\UTFT{20DB2}\UTFT{20DC8}\UTFT{20E04}\UTFT{20E0E}\UTFT{20ED7}%
\UTFT{20F90}\UTFT{20F2D}\UTFT{20E73}\UTFT{20FBC}\UTFT{2105C}\UTFT{2104F}\UTFT{21076}\UTFT{21088}\UTFT{21096}\UTFT{210BF}%
\UTFT{2112F}\UTFT{2113B}\UTFT{212E3}\UTFT{21375}\UTFT{21336}\UTFT{21577}\UTFT{21619}\UTFT{217C3}\UTFT{217C7}\UTFT{2182D}%
\UTFT{2196A}\UTFT{21A2D}\UTFT{21A45}\UTFT{21C2A}\UTFT{21C70}\UTFT{21CAC}\UTFT{21EC8}\UTFT{21ED5}\UTFT{21F15}\UTFT{22045}%
\UTFT{2227C}\UTFT{223D7}\UTFT{223FA}\UTFT{2272A}\UTFT{22871}\UTFT{2294F}\UTFT{22967}\UTFT{22993}\UTFT{22AD5}\UTFT{22AE8}%
\UTFT{22B0E}\UTFT{22B3F}\UTFT{22C4C}\UTFT{22C88}\UTFT{22CB7}\UTFT{25BE8}\UTFT{22D08}\UTFT{22D12}\UTFT{22DB7}\UTFT{22D95}%
\UTFT{22E42}\UTFT{22F74}\UTFT{22FCC}\UTFT{23033}\UTFT{23066}\UTFT{2331F}\UTFT{233DE}\UTFT{23567}\UTFT{235F3}\UTFT{2361A}%
\UTFT{23716}\UTFT{23AA7}\UTFT{23E11}\UTFT{23EB9}\UTFT{24119}\UTFT{242EE}\UTFT{2430D}\UTFT{24334}\UTFT{24396}\UTFT{24404}%
\UTFT{244D6}\UTFT{24674}\UTFT{2472F}\UTFT{24812}\UTFT{248FB}\UTFT{24A15}\UTFT{24AC0}\UTFT{24F86}\UTFT{2502C}\UTFT{25299}%
\UTFT{25419}\UTFT{25446}\UTFT{2546E}\UTFT{2553F}\UTFT{2555E}\UTFT{25562}\UTFT{25566}\UTFT{257C7}\UTFT{2585D}\UTFT{25903}%
\UTFT{25AAE}\UTFT{25B89}\UTFT{25C06}\UTFT{26102}\UTFT{261B2}\UTFT{26402}\UTFT{2644A}\UTFT{26484}\UTFT{26488}\UTFT{26512}%
\UTFT{265BF}\UTFT{266B5}\UTFT{266FC}\UTFT{26799}\UTFT{2686E}\UTFT{2685E}\UTFT{268C7}\UTFT{26926}\UTFT{26939}\UTFT{269FA}%
\UTFT{26A2D}\UTFT{26A34}\UTFT{26B5B}\UTFT{26B9D}\UTFT{26CA4}\UTFT{26DAE}\UTFT{2704B}\UTFT{271CD}\UTFT{27280}\UTFT{27285}%
\UTFT{2728B}\UTFT{272E6}\UTFT{27450}\UTFT{277CC}\UTFT{27858}\UTFT{279DD}\UTFT{279FD}\UTFT{27A0A}\UTFT{27B0B}\UTFT{27D66}%
\UTFT{28009}\UTFT{28023}\UTFT{28048}\UTFT{28083}\UTFT{28090}\UTFT{280F4}\UTFT{2812E}\UTFT{2814F}\UTFT{281AF}\UTFT{2821A}%
\UTFT{28306}\UTFT{2832F}\UTFT{2838A}\UTFT{28468}\UTFT{286AA}\UTFT{28956}\UTFT{289B8}\UTFT{289E7}\UTFT{289E8}\UTFT{28B46}%
\UTFT{28BD4}\UTFT{28C09}\UTFT{28FC5}\UTFT{290EC}\UTFT{29110}\UTFT{2913C}\UTFT{2915E}\UTFT{24ACA}\UTFT{294E7}\UTFT{295B0}%
\UTFT{295B8}\UTFT{29732}\UTFT{298D1}\UTFT{29949}\UTFT{2996A}\UTFT{299C3}\UTFT{29A28}\UTFT{29B0E}\UTFT{29D5A}\UTFT{29D9B}%
\UTFT{29EF8}\UTFT{29F23}\UTFT{2A293}\UTFT{2A2FF}\UTFT{2A5CB}\UTFT{20C9C}\UTFT{224B0}\UTFT{24A93}\UTFT{28B2C}\UTFT{217F5}%
\UTFT{28B6C}\UTFT{28B99}\UTFT{266AF}\UTFT{27655}\UTFT{25635}\UTFT{25956}\UTFT{25E81}\UTFT{20E6D}\UTFT{23E88}\UTFT{24C9E}%
\UTFT{217F6}\UTFT{2187B}\UTFT{25C4A}\UTFT{25311}\UTFT{25ED8}\UTFT{20FEA}\UTFT{20D49}\UTFT{236BA}\UTFT{25148}\UTFT{210C1}%
\UTFT{24706}\UTFT{26893}\UTFT{226F4}\UTFT{27D2F}\UTFT{241A3}\UTFT{27D73}\UTFT{26ED0}\UTFT{272B6}\UTFT{211D9}\UTFT{23CFC}%
\UTFT{2A6A9}\UTFT{20EAC}\UTFT{21CA2}\UTFT{24FC2}\UTFT{20FEB}\UTFT{22DA0}\UTFT{20FEC}\UTFT{20E0A}\UTFT{20FED}\UTFT{21187}%
\UTFT{24B6E}\UTFT{25A95}\UTFT{20979}\UTFT{22465}\UTFT{23CFE}\UTFT{29F30}\UTFT{24FA9}\UTFT{2959E}\UTFT{23DB6}\UTFT{267B3}%
\UTFT{23720}\UTFT{23EF7}\UTFT{23E2C}\UTFT{230DA}\UTFT{212A9}\UTFT{24963}\UTFT{270AE}\UTFT{2176C}\UTFT{27164}\UTFT{26D22}%
\UTFT{24AE2}\UTFT{2493E}\UTFT{26D23}\UTFT{203FC}\UTFT{23CFD}\UTFT{24919}\UTFT{24A77}\UTFT{28A5A}\UTFT{2F840}\UTFT{2183B}%
\UTFT{26159}\UTFT{233F5}\UTFT{28BC2}\UTFT{21D46}\UTFT{26ED1}\UTFT{28B2D}\UTFT{23CC7}\UTFT{25ED7}\UTFT{27656}\UTFT{25531}%
\UTFT{21944}\UTFT{29903}\UTFT{26DDC}\UTFT{270AD}\UTFT{261AD}\UTFT{28A0F}\UTFT{23677}\UTFT{200EE}\UTFT{26846}\UTFT{24F0E}%
\UTFT{2634C}\UTFT{2626B}\UTFT{21877}\UTFT{2408C}\UTFT{2307E}\UTFT{21E3D}\UTFT{203B5}\UTFT{205C3}\UTFT{21376}\UTFT{24A12}%
\UTFT{28B2B}\UTFT{26083}

Adobe-CNS1-4\\
\UTFT{29C73}\UTFT{2414E}\UTFT{251CD}\UTFT{25D30}\UTFT{28A32}\UTFT{23281}\UTFT{2A107}\UTFT{21980}\UTFT{2870F}\UTFT{2A2BA}%
\UTFT{29947}\UTFT{28AEA}\UTFT{2207E}\UTFT{289E3}\UTFT{21DB6}\UTFT{22712}\UTFT{233F9}\UTFT{23C63}\UTFT{24505}\UTFT{24A13}%
\UTFT{25CA4}\UTFT{25695}\UTFT{28DB9}\UTFT{2143F}\UTFT{2497B}\UTFT{2710D}\UTFT{26D74}\UTFT{26B15}\UTFT{26FBE}

Adobe-CNS1-5\\
\UTFT{27267}\UTFT{27CB1}\UTFT{27CC5}\UTFT{242BF}\UTFT{23617}\UTFT{27352}\UTFT{26E8B}\UTFT{270D2}\UTFT{2A351}\UTFT{27C6C}%
\UTFT{26B23}\UTFT{25A54}\UTFT{21A63}\UTFT{23E06}\UTFT{23F61}\UTFT{28BB9}\UTFT{27BEF}\UTFT{21D5E}\UTFT{29EB0}\UTFT{29945}%
\UTFT{20A6F}\UTFT{23256}\UTFT{22796}\UTFT{23B1A}\UTFT{23551}\UTFT{240EC}\UTFT{21E23}\UTFT{201A4}\UTFT{26C41}\UTFT{20239}%
\UTFT{298FA}\UTFT{20B9F}\UTFT{221C1}\UTFT{2896D}\UTFT{29079}\UTFT{2A1B5}\UTFT{26C46}\UTFT{286B2}\UTFT{273FF}\UTFT{2549A}%
\UTFT{24B0F}

Adobe-CNS1-6\\
\UTFT{21D53}\UTFT{2369E}\UTFT{26021}\UTFT{258DE}\UTFT{24161}\UTFT{2890D}\UTFT{231EA}\UTFT{20A8A}\UTFT{2325E}\UTFT{25DB9}%
\UTFT{2368E}\UTFT{27B65}\UTFT{26E88}\UTFT{25D99}\UTFT{224BC}\UTFT{224C1}\UTFT{224C9}\UTFT{224CC}\UTFT{235BB}\UTFT{2ADFF}%


% end


%
% This file is generated from the data of UniGB-UTF32
% in cid2code.txt (Version 12/05/2017)
% for Adobe-GB1-5
%
% Reference:
%   https://github.com/adobe-type-tools/cmap-resources/
%   Adobe-GB1-5/cid2code.txt
%
% A newer CMap may be required for some code points.
%


Adobe-GB1-2\\
\UTFC{20087}\UTFC{20089}\UTFC{200CC}\UTFC{215D7}\UTFC{2298F}\UTFC{20509}\UTFC{2099D}\UTFC{241FE}

% end


}

\end{document}


{\gtfamily
[gt/m]

%
% This file is generated from the data of UniCNS-UTF32
% in cid2code.txt (Version 12/04/2015)
% for Adobe-CNS1-6
%
% Reference:
%   https://github.com/adobe-type-tools/cmap-resources/
%   cmapresources_cns1-6/cid2code.txt
%
% A newer CMap may be required for some code points.
%


Adobe-CNS1-0\\
\UTFT{200CC}\UTFT{2008A}\UTFT{27607}

Adobe-CNS1-1\\
\UTFT{23ED7}\UTFT{26ED3}\UTFT{257E0}\UTFT{28BE9}\UTFT{258E1}\UTFT{294D9}\UTFT{259AC}\UTFT{2648D}\UTFT{25C01}\UTFT{2530E}%
\UTFT{25CFE}\UTFT{25BB4}\UTFT{26C7F}\UTFT{25D20}\UTFT{25CC1}\UTFT{24882}\UTFT{24578}\UTFT{26E44}\UTFT{26ED6}\UTFT{24057}%
\UTFT{26029}\UTFT{217F9}\UTFT{2836D}\UTFT{26121}\UTFT{2615A}\UTFT{262D0}\UTFT{26351}\UTFT{21661}\UTFT{20068}\UTFT{23766}%
\UTFT{2833A}\UTFT{26489}\UTFT{2A087}\UTFT{26CC3}\UTFT{22714}\UTFT{26626}\UTFT{23DE3}\UTFT{266E8}\UTFT{28A48}\UTFT{226F6}%
\UTFT{26498}\UTFT{2148A}\UTFT{2185E}\UTFT{24A65}\UTFT{24A95}\UTFT{26A52}\UTFT{23D7E}\UTFT{214FD}\UTFT{2F98F}\UTFT{249A7}%
\UTFT{23530}\UTFT{21773}\UTFT{23DF8}\UTFT{2F994}\UTFT{20E16}\UTFT{217B4}\UTFT{2317D}\UTFT{2355A}\UTFT{23E8B}\UTFT{26DA3}%
\UTFT{26B05}\UTFT{26B97}\UTFT{235CE}\UTFT{26DA5}\UTFT{26ED4}\UTFT{26E42}\UTFT{25BE4}\UTFT{26B96}\UTFT{26E77}\UTFT{26E43}%
\UTFT{25C91}\UTFT{25CC0}\UTFT{28625}\UTFT{2863B}\UTFT{27088}\UTFT{21582}\UTFT{270CD}\UTFT{2F9B2}\UTFT{218A2}\UTFT{2739A}%
\UTFT{2A0F8}\UTFT{22C27}\UTFT{275E0}\UTFT{23DB9}\UTFT{275E4}\UTFT{2770F}\UTFT{28A25}\UTFT{27924}\UTFT{27ABD}\UTFT{27A59}%
\UTFT{27B3A}\UTFT{27B38}\UTFT{25430}\UTFT{25565}\UTFT{24A7A}\UTFT{216DF}\UTFT{27D54}\UTFT{27D8F}\UTFT{2F9D4}\UTFT{27D53}%
\UTFT{27D98}\UTFT{27DBD}\UTFT{21910}\UTFT{2F9D7}\UTFT{28002}\UTFT{21014}\UTFT{2498A}\UTFT{281BC}\UTFT{2710C}\UTFT{28365}%
\UTFT{28412}\UTFT{2A29F}\UTFT{20A50}\UTFT{289DE}\UTFT{2853D}\UTFT{23DBB}\UTFT{23262}\UTFT{22325}\UTFT{26ED7}\UTFT{2853C}%
\UTFT{27ABE}\UTFT{2856C}\UTFT{2860B}\UTFT{28713}\UTFT{286E6}\UTFT{28933}\UTFT{21E89}\UTFT{255B9}\UTFT{28AC6}\UTFT{23C9B}%
\UTFT{28B0C}\UTFT{255DB}\UTFT{20D31}\UTFT{28AE1}\UTFT{28BEB}\UTFT{28AE2}\UTFT{28AE5}\UTFT{28BEC}\UTFT{28C39}\UTFT{28BFF}%
\UTFT{286D8}\UTFT{2127C}\UTFT{23E2E}\UTFT{26ED5}\UTFT{28AE0}\UTFT{26CB8}\UTFT{20274}\UTFT{26410}\UTFT{290AF}\UTFT{290E5}%
\UTFT{24AD1}\UTFT{21915}\UTFT{2330A}\UTFT{24AE9}\UTFT{291D5}\UTFT{291EB}\UTFT{230B7}\UTFT{230BC}\UTFT{2546C}\UTFT{29433}%
\UTFT{2941D}\UTFT{2797A}\UTFT{27175}\UTFT{20630}\UTFT{2415C}\UTFT{25706}\UTFT{26D27}\UTFT{216D3}\UTFT{24A29}\UTFT{29857}%
\UTFT{29905}\UTFT{25725}\UTFT{290B1}\UTFT{29BD5}\UTFT{29B05}\UTFT{28600}\UTFT{2307D}\UTFT{29D3E}\UTFT{21863}\UTFT{29E68}%
\UTFT{29FB7}\UTFT{2A192}\UTFT{2A1AB}\UTFT{2A0E1}\UTFT{2A123}\UTFT{2A1DF}\UTFT{2A134}\UTFT{2A193}\UTFT{2A220}\UTFT{2193B}%
\UTFT{2A233}\UTFT{2A0B9}\UTFT{2A2B4}\UTFT{24364}\UTFT{28C2B}\UTFT{26DA2}\UTFT{2FA1B}\UTFT{2908B}\UTFT{24975}\UTFT{249BB}%
\UTFT{249F8}\UTFT{24348}\UTFT{24A51}\UTFT{28BDA}\UTFT{218FA}\UTFT{2897E}\UTFT{28E36}\UTFT{28A44}\UTFT{2896C}\UTFT{244B9}%
\UTFT{24473}\UTFT{243F8}\UTFT{217EF}\UTFT{218BE}\UTFT{23599}\UTFT{21885}\UTFT{2542F}\UTFT{217F8}\UTFT{216FB}\UTFT{21839}%
\UTFT{21774}\UTFT{218D1}\UTFT{25F4B}\UTFT{216C0}\UTFT{24A25}\UTFT{213FE}\UTFT{212A8}\UTFT{213C6}\UTFT{214B6}\UTFT{236A6}%
\UTFT{24994}\UTFT{27165}\UTFT{23E31}\UTFT{2555C}\UTFT{23EFB}\UTFT{27052}\UTFT{236EE}\UTFT{2999D}\UTFT{26F26}\UTFT{21922}%
\UTFT{2373F}\UTFT{240E1}\UTFT{2408B}\UTFT{2410F}\UTFT{26C21}\UTFT{266B1}\UTFT{20FDF}\UTFT{20BA8}\UTFT{20E0D}\UTFT{28B13}%
\UTFT{24436}\UTFT{20465}\UTFT{25651}\UTFT{201AB}\UTFT{203CB}\UTFT{2030A}\UTFT{20414}\UTFT{202C0}\UTFT{28EB3}\UTFT{20275}%
\UTFT{2020C}\UTFT{24A0E}\UTFT{23E8A}\UTFT{23595}\UTFT{23E39}\UTFT{23EBF}\UTFT{21884}\UTFT{23E89}\UTFT{205E0}\UTFT{204A3}%
\UTFT{20492}\UTFT{20491}\UTFT{28A9C}\UTFT{2070E}\UTFT{20873}\UTFT{2438C}\UTFT{20C20}\UTFT{249AC}\UTFT{210E4}\UTFT{20E1D}%
\UTFT{24ABC}\UTFT{2408D}\UTFT{240C9}\UTFT{20345}\UTFT{20BC6}\UTFT{28A46}\UTFT{216FA}\UTFT{2176F}\UTFT{21710}\UTFT{25946}%
\UTFT{219F3}\UTFT{21861}\UTFT{24295}\UTFT{25E83}\UTFT{28BD7}\UTFT{20413}\UTFT{21303}\UTFT{289FB}\UTFT{21996}\UTFT{2197C}%
\UTFT{23AEE}\UTFT{21903}\UTFT{21904}\UTFT{218A0}\UTFT{216FE}\UTFT{28A47}\UTFT{21DBA}\UTFT{23472}\UTFT{289A8}\UTFT{21927}%
\UTFT{217AB}\UTFT{2173B}\UTFT{275FD}\UTFT{22860}\UTFT{2262B}\UTFT{225AF}\UTFT{225BE}\UTFT{29088}\UTFT{26F73}\UTFT{2003E}%
\UTFT{20046}\UTFT{2261B}\UTFT{22C9B}\UTFT{22D07}\UTFT{246D4}\UTFT{2914D}\UTFT{24665}\UTFT{22B6A}\UTFT{22B22}\UTFT{23450}%
\UTFT{298EA}\UTFT{22E78}\UTFT{249E3}\UTFT{22D67}\UTFT{22CA1}\UTFT{2308E}\UTFT{232AD}\UTFT{24989}\UTFT{232AB}\UTFT{232E0}%
\UTFT{218D9}\UTFT{2943F}\UTFT{23289}\UTFT{231B3}\UTFT{25584}\UTFT{28B22}\UTFT{2558F}\UTFT{216FC}\UTFT{2555B}\UTFT{25425}%
\UTFT{23103}\UTFT{2182A}\UTFT{23234}\UTFT{2320F}\UTFT{23182}\UTFT{242C9}\UTFT{26D24}\UTFT{27870}\UTFT{21DEB}\UTFT{232D2}%
\UTFT{232E1}\UTFT{25872}\UTFT{2383A}\UTFT{237BC}\UTFT{237A2}\UTFT{233FE}\UTFT{2462A}\UTFT{237D5}\UTFT{24487}\UTFT{21912}%
\UTFT{23FC0}\UTFT{23C9A}\UTFT{28BEA}\UTFT{28ACB}\UTFT{2801E}\UTFT{289DC}\UTFT{23F7F}\UTFT{2403C}\UTFT{2431A}\UTFT{24276}%
\UTFT{2478F}\UTFT{24725}\UTFT{24AA4}\UTFT{205EB}\UTFT{23EF8}\UTFT{2365F}\UTFT{24A4A}\UTFT{24917}\UTFT{25FE1}\UTFT{24ADF}%
\UTFT{28C23}\UTFT{23F35}\UTFT{26DEA}\UTFT{24CD9}\UTFT{24D06}\UTFT{2A5C6}\UTFT{28ACC}\UTFT{249AB}\UTFT{2498E}\UTFT{24A4E}%
\UTFT{249C5}\UTFT{248F3}\UTFT{28AE3}\UTFT{21864}\UTFT{25221}\UTFT{251E7}\UTFT{23232}\UTFT{24697}\UTFT{23781}\UTFT{248F0}%
\UTFT{24ABA}\UTFT{24AC7}\UTFT{24A96}\UTFT{261AE}\UTFT{25581}\UTFT{27741}\UTFT{256E3}\UTFT{23EFA}\UTFT{216E6}\UTFT{20D4C}%
\UTFT{2498C}\UTFT{20299}\UTFT{23DBA}\UTFT{2176E}\UTFT{201D4}\UTFT{20C0D}\UTFT{226F5}\UTFT{25AAF}\UTFT{25A9C}\UTFT{2025B}%
\UTFT{25BC6}\UTFT{25BB3}\UTFT{25EBC}\UTFT{25EA6}\UTFT{249F9}\UTFT{217B0}\UTFT{26261}\UTFT{2615C}\UTFT{27B48}\UTFT{25E82}%
\UTFT{26B75}\UTFT{20916}\UTFT{2004E}\UTFT{235CF}\UTFT{26412}\UTFT{263F8}\UTFT{2082C}\UTFT{25AE9}\UTFT{25D43}\UTFT{25E0E}%
\UTFT{2343F}\UTFT{249F7}\UTFT{265AD}\UTFT{265A0}\UTFT{27127}\UTFT{26CD1}\UTFT{267B4}\UTFT{26A42}\UTFT{26A51}\UTFT{26DA7}%
\UTFT{2721B}\UTFT{21840}\UTFT{218A1}\UTFT{218D8}\UTFT{2F9BC}\UTFT{23D8F}\UTFT{27422}\UTFT{25683}\UTFT{27785}\UTFT{27784}%
\UTFT{28BF5}\UTFT{28BD9}\UTFT{28B9C}\UTFT{289F9}\UTFT{29448}\UTFT{24284}\UTFT{21845}\UTFT{27DDC}\UTFT{24C09}\UTFT{22321}%
\UTFT{217DA}\UTFT{2492F}\UTFT{28A4B}\UTFT{28AFC}\UTFT{28C1D}\UTFT{28C3B}\UTFT{28D34}\UTFT{248FF}\UTFT{24A42}\UTFT{243EA}%
\UTFT{23225}\UTFT{28EE7}\UTFT{28E66}\UTFT{28E65}\UTFT{249ED}\UTFT{24A78}\UTFT{23FEE}\UTFT{290B0}\UTFT{29093}\UTFT{257DF}%
\UTFT{28989}\UTFT{28C26}\UTFT{28B2F}\UTFT{263BE}\UTFT{2421B}\UTFT{20F26}\UTFT{28BC5}\UTFT{24AB2}\UTFT{294DA}\UTFT{295D7}%
\UTFT{28B50}\UTFT{24A67}\UTFT{28B64}\UTFT{28A45}\UTFT{27B06}\UTFT{28B65}\UTFT{258C8}\UTFT{298F1}\UTFT{29948}\UTFT{21302}%
\UTFT{249B8}\UTFT{214E8}\UTFT{2271F}\UTFT{23DB8}\UTFT{22781}\UTFT{2296B}\UTFT{29E2D}\UTFT{2A1F5}\UTFT{2A0FE}\UTFT{24104}%
\UTFT{2A1B4}\UTFT{2A0ED}\UTFT{2A0F3}\UTFT{2992F}\UTFT{26E12}\UTFT{26FDF}\UTFT{26B82}\UTFT{26DA4}\UTFT{26E84}\UTFT{26DF0}%
\UTFT{26E00}\UTFT{237D7}\UTFT{26064}\UTFT{2359C}\UTFT{23640}\UTFT{249DE}\UTFT{202BF}\UTFT{2555D}\UTFT{21757}\UTFT{231C9}%
\UTFT{24941}\UTFT{241B5}\UTFT{241AC}\UTFT{26C40}\UTFT{24F97}\UTFT{217B5}\UTFT{28A49}\UTFT{24488}\UTFT{289FC}\UTFT{218D6}%
\UTFT{20F1D}\UTFT{26CC0}\UTFT{21413}\UTFT{242FA}\UTFT{22C26}\UTFT{243C1}\UTFT{23DB7}\UTFT{26741}\UTFT{2615B}\UTFT{260A4}%
\UTFT{249B9}\UTFT{2498B}\UTFT{289FA}\UTFT{28B63}\UTFT{2189F}\UTFT{24AB3}\UTFT{24A3E}\UTFT{24A94}\UTFT{217D9}\UTFT{24A66}%
\UTFT{203A7}\UTFT{21424}\UTFT{249E5}\UTFT{24916}\UTFT{24976}\UTFT{204FE}\UTFT{28ACE}\UTFT{28A16}\UTFT{28BE7}\UTFT{255D5}%
\UTFT{28A82}\UTFT{24943}\UTFT{20CFF}\UTFT{2061A}\UTFT{20BEB}\UTFT{20CB8}\UTFT{217FA}\UTFT{216C2}\UTFT{24A50}\UTFT{21852}%
\UTFT{28AC0}\UTFT{249AD}\UTFT{218BF}\UTFT{21883}\UTFT{27484}\UTFT{23D5B}\UTFT{28A81}\UTFT{21862}\UTFT{20AB4}\UTFT{2139C}%
\UTFT{28218}\UTFT{290E4}\UTFT{27E4F}\UTFT{23FED}\UTFT{23E2D}\UTFT{203F5}\UTFT{28C1C}\UTFT{26BC0}\UTFT{21452}\UTFT{24362}%
\UTFT{24A71}\UTFT{22FE3}\UTFT{212B0}\UTFT{223BD}\UTFT{21398}\UTFT{234E5}\UTFT{27BF4}\UTFT{236DF}\UTFT{28A83}\UTFT{237D6}%
\UTFT{233FA}\UTFT{24C9F}\UTFT{236AD}\UTFT{26CB7}\UTFT{26D26}\UTFT{26D51}\UTFT{26C82}\UTFT{26FDE}\UTFT{2173A}\UTFT{26C80}%
\UTFT{27053}\UTFT{217DB}\UTFT{217B3}\UTFT{21905}\UTFT{241FC}\UTFT{2173C}\UTFT{242A5}\UTFT{24293}\UTFT{23EF9}\UTFT{27736}%
\UTFT{2445B}\UTFT{242CA}\UTFT{24259}\UTFT{289E1}\UTFT{26D28}\UTFT{244CE}\UTFT{27E4D}\UTFT{243BD}\UTFT{24256}\UTFT{21304}%
\UTFT{243E9}\UTFT{2F825}\UTFT{23300}\UTFT{27AF4}\UTFT{256F6}\UTFT{27B18}\UTFT{27A79}\UTFT{249BA}\UTFT{20346}\UTFT{27657}%
\UTFT{25FE2}\UTFT{275FE}\UTFT{2209A}\UTFT{28A9A}\UTFT{2403B}\UTFT{24A45}\UTFT{205CA}\UTFT{20611}\UTFT{21EA8}\UTFT{23CFF}%
\UTFT{285E8}\UTFT{299C9}\UTFT{221C3}\UTFT{28B4E}\UTFT{20C78}\UTFT{20779}\UTFT{23F4A}\UTFT{24AA7}\UTFT{26B52}\UTFT{27632}%
\UTFT{2493F}\UTFT{233CC}\UTFT{28948}\UTFT{21D90}\UTFT{27C12}\UTFT{24F9A}\UTFT{26BF7}\UTFT{2191C}\UTFT{249F6}\UTFT{23FEF}%
\UTFT{2271B}\UTFT{257E1}\UTFT{2F8CD}\UTFT{2F806}\UTFT{24521}\UTFT{24934}\UTFT{26CBD}\UTFT{26411}\UTFT{290C0}\UTFT{20A11}%
\UTFT{26469}\UTFT{20021}\UTFT{23519}\UTFT{2258D}\UTFT{2217A}\UTFT{249D0}\UTFT{20EF8}\UTFT{22926}\UTFT{28473}\UTFT{217B1}%
\UTFT{24A2A}\UTFT{21820}\UTFT{29CAD}\UTFT{298A4}\UTFT{2160A}\UTFT{2372F}\UTFT{280E8}\UTFT{213C5}\UTFT{291A8}\UTFT{270AF}%
\UTFT{289AB}\UTFT{2417A}\UTFT{2A2DF}\UTFT{28318}\UTFT{26E07}\UTFT{2816F}\UTFT{269B5}\UTFT{213ED}\UTFT{2322F}\UTFT{28C30}%
\UTFT{28949}\UTFT{24988}\UTFT{24AA5}\UTFT{23F81}\UTFT{21FA1}\UTFT{295E9}\UTFT{2789D}\UTFT{28024}\UTFT{27A3E}\UTFT{23CB7}%
\UTFT{26258}\UTFT{29D98}\UTFT{23D40}\UTFT{20E9D}\UTFT{282E2}\UTFT{20C41}\UTFT{20C96}\UTFT{20E76}\UTFT{22C62}\UTFT{20EA2}%
\UTFT{21075}\UTFT{22B43}\UTFT{22EB3}\UTFT{20DA7}\UTFT{2688A}\UTFT{20EF9}\UTFT{27FF9}\UTFT{247E0}\UTFT{29D7C}\UTFT{275A3}%
\UTFT{26048}\UTFT{24618}\UTFT{29EAC}\UTFT{29FDE}\UTFT{272B2}\UTFT{2048E}\UTFT{20EB6}\UTFT{27F2E}\UTFT{2A434}\UTFT{243F2}%
\UTFT{29E06}\UTFT{294D0}\UTFT{26335}\UTFT{20D28}\UTFT{20D71}\UTFT{21F0F}\UTFT{21DD1}\UTFT{2176D}\UTFT{2B473}\UTFT{28E97}%
\UTFT{25C21}\UTFT{20CD4}\UTFT{201F2}\UTFT{2A64A}\UTFT{2837D}\UTFT{2A2B2}\UTFT{24ABB}\UTFT{26E05}\UTFT{2AE67}\UTFT{2251B}%
\UTFT{28E39}\UTFT{20F3B}\UTFT{25F1A}\UTFT{27486}\UTFT{267CC}\UTFT{24011}\UTFT{2F922}\UTFT{20547}\UTFT{205DF}\UTFT{23FC5}%
\UTFT{24942}\UTFT{289E4}\UTFT{219DB}\UTFT{23CC8}\UTFT{24933}\UTFT{289AA}\UTFT{202A0}\UTFT{26BB3}\UTFT{21305}\UTFT{224ED}%
\UTFT{26D29}\UTFT{27A84}\UTFT{23600}\UTFT{24AB1}\UTFT{22513}\UTFT{2037E}\UTFT{20380}\UTFT{20347}\UTFT{2041F}\UTFT{249A4}%
\UTFT{20487}\UTFT{233B4}\UTFT{20BFF}\UTFT{220FC}\UTFT{202E5}\UTFT{22530}\UTFT{2058E}\UTFT{23233}\UTFT{21983}\UTFT{205B3}%
\UTFT{23C99}\UTFT{24AA6}\UTFT{2372D}\UTFT{26B13}\UTFT{2F829}\UTFT{28ADE}\UTFT{23F80}\UTFT{20954}\UTFT{23FEC}\UTFT{20BE2}%
\UTFT{21726}\UTFT{216E8}\UTFT{286AB}\UTFT{2F832}\UTFT{21596}\UTFT{21613}\UTFT{28A9B}\UTFT{25772}\UTFT{20B8F}\UTFT{23FEB}%
\UTFT{22DA3}\UTFT{20C77}\UTFT{26B53}\UTFT{20D74}\UTFT{2170D}\UTFT{20EDD}\UTFT{20D4D}\UTFT{289BC}\UTFT{22698}\UTFT{218D7}%
\UTFT{2403A}\UTFT{24435}\UTFT{210B4}\UTFT{2328A}\UTFT{28B66}\UTFT{2124F}\UTFT{241A5}\UTFT{26C7E}\UTFT{21416}\UTFT{21454}%
\UTFT{24363}\UTFT{24BF5}\UTFT{2123C}\UTFT{2A150}\UTFT{24278}\UTFT{2163E}\UTFT{21692}\UTFT{20D4E}\UTFT{26C81}\UTFT{26D2A}%
\UTFT{217DC}\UTFT{217FB}\UTFT{217B2}\UTFT{26DA6}\UTFT{21828}\UTFT{216D5}\UTFT{26E45}\UTFT{249A9}\UTFT{26FA1}\UTFT{22554}%
\UTFT{21911}\UTFT{216B8}\UTFT{27A0E}\UTFT{20204}\UTFT{21A34}\UTFT{259CC}\UTFT{205A5}\UTFT{21B44}\UTFT{21CA5}\UTFT{26B28}%
\UTFT{21DF9}\UTFT{21E37}\UTFT{21EA4}\UTFT{24901}\UTFT{22049}\UTFT{22173}\UTFT{244BC}\UTFT{20CD3}\UTFT{21771}\UTFT{28482}%
\UTFT{201C1}\UTFT{2F894}\UTFT{2133A}\UTFT{26888}\UTFT{223D0}\UTFT{22471}\UTFT{26E6E}\UTFT{28A36}\UTFT{25250}\UTFT{21F6A}%
\UTFT{270F8}\UTFT{22668}\UTFT{2029E}\UTFT{28A29}\UTFT{227B4}\UTFT{24982}\UTFT{2498F}\UTFT{27A53}\UTFT{2F8A6}\UTFT{26ED2}%
\UTFT{20656}\UTFT{23FB7}\UTFT{2285F}\UTFT{28B9D}\UTFT{2995D}\UTFT{22980}\UTFT{228C1}\UTFT{20118}\UTFT{21770}\UTFT{22E0D}%
\UTFT{249DF}\UTFT{2138E}\UTFT{217FC}\UTFT{22E36}\UTFT{2571D}\UTFT{24A28}\UTFT{24A23}\UTFT{24940}\UTFT{21829}\UTFT{23400}%
\UTFT{231F7}\UTFT{231F8}\UTFT{231A4}\UTFT{231A5}\UTFT{20E75}\UTFT{251E6}\UTFT{23231}\UTFT{285F4}\UTFT{231C8}\UTFT{25313}%
\UTFT{228F7}\UTFT{2439C}\UTFT{24A21}\UTFT{237C2}\UTFT{2F8DB}\UTFT{241CD}\UTFT{290ED}\UTFT{233E6}\UTFT{26DA0}\UTFT{2346F}%
\UTFT{28ADF}\UTFT{235CD}\UTFT{2363C}\UTFT{28A4A}\UTFT{203C9}\UTFT{23659}\UTFT{2212A}\UTFT{23703}\UTFT{2919C}\UTFT{20923}%
\UTFT{227CD}\UTFT{23ADB}\UTFT{21958}\UTFT{23B5A}\UTFT{23EFC}\UTFT{2248B}\UTFT{248F1}\UTFT{26B51}\UTFT{23DBC}\UTFT{23DBD}%
\UTFT{241A4}\UTFT{2490C}\UTFT{24900}\UTFT{23CC9}\UTFT{20D32}\UTFT{231F9}\UTFT{22491}\UTFT{26D25}\UTFT{26DA1}\UTFT{26DEB}%
\UTFT{2497F}\UTFT{24085}\UTFT{26E72}\UTFT{26F74}\UTFT{28B21}\UTFT{2F908}\UTFT{23E2F}\UTFT{23F82}\UTFT{2304B}\UTFT{23E30}%
\UTFT{21497}\UTFT{2403D}\UTFT{29170}\UTFT{24144}\UTFT{24091}\UTFT{24155}\UTFT{24039}\UTFT{23FF0}\UTFT{23FB4}\UTFT{2413F}%
\UTFT{24156}\UTFT{24157}\UTFT{24140}\UTFT{261DD}\UTFT{24277}\UTFT{24365}\UTFT{242C1}\UTFT{2445A}\UTFT{24A27}\UTFT{24A22}%
\UTFT{28BE8}\UTFT{25605}\UTFT{24974}\UTFT{23044}\UTFT{24823}\UTFT{2882B}\UTFT{28804}\UTFT{20C3A}\UTFT{26A2E}\UTFT{241E2}%
\UTFT{216E7}\UTFT{24A24}\UTFT{249B7}\UTFT{2498D}\UTFT{249FB}\UTFT{24A26}\UTFT{2F92F}\UTFT{228AD}\UTFT{28EB2}\UTFT{24A8C}%
\UTFT{2415F}\UTFT{24A79}\UTFT{28B8F}\UTFT{28C03}\UTFT{2189E}\UTFT{21988}\UTFT{28ED9}\UTFT{21A4B}\UTFT{28EAC}\UTFT{24F82}%
\UTFT{24D13}\UTFT{263F5}\UTFT{26911}\UTFT{2690E}\UTFT{26F9F}\UTFT{2509D}\UTFT{2517D}\UTFT{21E1C}\UTFT{25220}\UTFT{232AC}%
\UTFT{28964}\UTFT{28968}\UTFT{216C1}\UTFT{255E0}\UTFT{2760C}\UTFT{2261C}\UTFT{25857}\UTFT{27B39}\UTFT{27126}\UTFT{2910D}%
\UTFT{20C42}\UTFT{20D15}\UTFT{2512B}\UTFT{22CC6}\UTFT{20341}\UTFT{24DB8}\UTFT{294E5}\UTFT{280BE}\UTFT{22C38}\UTFT{2815D}%
\UTFT{269F2}\UTFT{24DEA}\UTFT{20D7C}\UTFT{20FB4}\UTFT{20CD5}\UTFT{2BAB3}\UTFT{20E96}\UTFT{20F64}\UTFT{22CA9}\UTFT{28256}%
\UTFT{244D3}\UTFT{20D46}\UTFT{29A4D}\UTFT{280E9}\UTFT{24EA7}\UTFT{22CC2}\UTFT{295F4}\UTFT{252C7}\UTFT{297D4}\UTFT{22D44}%
\UTFT{2BCD7}\UTFT{22BCA}\UTFT{2B977}\UTFT{266DA}\UTFT{26716}\UTFT{279A0}\UTFT{25052}\UTFT{20C43}\UTFT{28B4C}\UTFT{20731}%
\UTFT{201A9}\UTFT{22D8D}\UTFT{245C8}\UTFT{204FC}\UTFT{26097}\UTFT{20F4C}\UTFT{22A66}\UTFT{2109D}\UTFT{20D9C}\UTFT{22775}%
\UTFT{2A601}\UTFT{20E09}\UTFT{22ACF}\UTFT{2C5F8}\UTFT{210C8}\UTFT{239C2}\UTFT{2829B}\UTFT{25E49}\UTFT{220C7}\UTFT{22CB2}%
\UTFT{29720}\UTFT{24E3B}\UTFT{2C9A0}\UTFT{27574}\UTFT{22E8B}\UTFT{22208}\UTFT{2A65B}\UTFT{28CCD}\UTFT{20E7A}\UTFT{20C34}%
\UTFT{27639}\UTFT{22BCE}\UTFT{22C51}\UTFT{210C7}\UTFT{2A632}\UTFT{28CD2}\UTFT{28D99}\UTFT{28CCA}\UTFT{2775E}\UTFT{2F828}%
\UTFT{2107B}\UTFT{210D3}\UTFT{212FE}\UTFT{247EF}\UTFT{24EA5}\UTFT{24F5C}\UTFT{28189}\UTFT{2B42C}

Adobe-CNS1-3\\
\UTFT{2010C}\UTFT{200D1}\UTFT{200CD}\UTFT{200CB}\UTFT{21FE8}\UTFT{200CA}\UTFT{2010E}\UTFT{21BC1}\UTFT{2F878}\UTFT{20086}%
\UTFT{248E9}\UTFT{2626A}\UTFT{2634B}\UTFT{26612}\UTFT{26951}\UTFT{278B2}\UTFT{28E0F}\UTFT{29810}\UTFT{20087}\UTFT{2A3A9}%
\UTFT{21145}\UTFT{27735}\UTFT{209E7}\UTFT{29DF6}\UTFT{2700E}\UTFT{2A133}\UTFT{2846C}\UTFT{21DCA}\UTFT{205D0}\UTFT{22AE6}%
\UTFT{27D84}\UTFT{210F4}\UTFT{20C0B}\UTFT{278C8}\UTFT{260A5}\UTFT{22D4C}\UTFT{21077}\UTFT{2106F}\UTFT{221A1}\UTFT{20D96}%
\UTFT{22CC9}\UTFT{20F31}\UTFT{2681C}\UTFT{210CF}\UTFT{22803}\UTFT{22939}\UTFT{251E3}\UTFT{20E8C}\UTFT{20F8D}\UTFT{20EAA}%
\UTFT{20F30}\UTFT{20D47}\UTFT{2114F}\UTFT{20E4C}\UTFT{20EAB}\UTFT{20BA9}\UTFT{20D48}\UTFT{210C0}\UTFT{2113D}\UTFT{22696}%
\UTFT{20FAD}\UTFT{233F4}\UTFT{20D7E}\UTFT{20D7F}\UTFT{22C55}\UTFT{20E98}\UTFT{20F2E}\UTFT{26B50}\UTFT{29EC3}\UTFT{22DEE}%
\UTFT{26572}\UTFT{280BD}\UTFT{20EFA}\UTFT{20E0F}\UTFT{20E77}\UTFT{20EFB}\UTFT{24DEB}\UTFT{20CD6}\UTFT{227B5}\UTFT{210C9}%
\UTFT{20E10}\UTFT{20E78}\UTFT{21078}\UTFT{21148}\UTFT{28207}\UTFT{21455}\UTFT{20E79}\UTFT{24E50}\UTFT{22DA4}\UTFT{2101D}%
\UTFT{2101E}\UTFT{210F5}\UTFT{210F6}\UTFT{20E11}\UTFT{27694}\UTFT{282CD}\UTFT{20FB5}\UTFT{20E7B}\UTFT{2517E}\UTFT{20FB6}%
\UTFT{21180}\UTFT{252D8}\UTFT{2A2BD}\UTFT{249DA}\UTFT{2183A}\UTFT{24177}\UTFT{2827C}\UTFT{2573D}\UTFT{25B74}\UTFT{2313D}%
\UTFT{289C0}\UTFT{23F41}\UTFT{20325}\UTFT{20ED8}\UTFT{25C65}\UTFT{24FB8}\UTFT{20B0D}\UTFT{26B0A}\UTFT{22EEF}\UTFT{23CB5}%
\UTFT{26E99}\UTFT{23F8F}\UTFT{24CC9}\UTFT{2A014}\UTFT{286BC}\UTFT{28501}\UTFT{2267A}\UTFT{269A8}\UTFT{2424B}\UTFT{2215B}%
\UTFT{2037F}\UTFT{2A45B}\UTFT{249EC}\UTFT{24962}\UTFT{27109}\UTFT{24A4F}\UTFT{24A5D}\UTFT{217DF}\UTFT{23AFA}\UTFT{20214}%
\UTFT{208D5}\UTFT{20619}\UTFT{21F9E}\UTFT{2A2B6}\UTFT{2915B}\UTFT{28A59}\UTFT{29420}\UTFT{248F2}\UTFT{25535}\UTFT{20CCF}%
\UTFT{27967}\UTFT{21BC2}\UTFT{20094}\UTFT{202B7}\UTFT{203A0}\UTFT{204D7}\UTFT{205D5}\UTFT{20615}\UTFT{20676}\UTFT{216BA}%
\UTFT{20AC2}\UTFT{20ACD}\UTFT{20BBF}\UTFT{2F83B}\UTFT{20BCB}\UTFT{20BFB}\UTFT{20C3B}\UTFT{20C53}\UTFT{20C65}\UTFT{20C7C}%
\UTFT{20C8D}\UTFT{20CB5}\UTFT{20CDD}\UTFT{20CED}\UTFT{20D6F}\UTFT{20DB2}\UTFT{20DC8}\UTFT{20E04}\UTFT{20E0E}\UTFT{20ED7}%
\UTFT{20F90}\UTFT{20F2D}\UTFT{20E73}\UTFT{20FBC}\UTFT{2105C}\UTFT{2104F}\UTFT{21076}\UTFT{21088}\UTFT{21096}\UTFT{210BF}%
\UTFT{2112F}\UTFT{2113B}\UTFT{212E3}\UTFT{21375}\UTFT{21336}\UTFT{21577}\UTFT{21619}\UTFT{217C3}\UTFT{217C7}\UTFT{2182D}%
\UTFT{2196A}\UTFT{21A2D}\UTFT{21A45}\UTFT{21C2A}\UTFT{21C70}\UTFT{21CAC}\UTFT{21EC8}\UTFT{21ED5}\UTFT{21F15}\UTFT{22045}%
\UTFT{2227C}\UTFT{223D7}\UTFT{223FA}\UTFT{2272A}\UTFT{22871}\UTFT{2294F}\UTFT{22967}\UTFT{22993}\UTFT{22AD5}\UTFT{22AE8}%
\UTFT{22B0E}\UTFT{22B3F}\UTFT{22C4C}\UTFT{22C88}\UTFT{22CB7}\UTFT{25BE8}\UTFT{22D08}\UTFT{22D12}\UTFT{22DB7}\UTFT{22D95}%
\UTFT{22E42}\UTFT{22F74}\UTFT{22FCC}\UTFT{23033}\UTFT{23066}\UTFT{2331F}\UTFT{233DE}\UTFT{23567}\UTFT{235F3}\UTFT{2361A}%
\UTFT{23716}\UTFT{23AA7}\UTFT{23E11}\UTFT{23EB9}\UTFT{24119}\UTFT{242EE}\UTFT{2430D}\UTFT{24334}\UTFT{24396}\UTFT{24404}%
\UTFT{244D6}\UTFT{24674}\UTFT{2472F}\UTFT{24812}\UTFT{248FB}\UTFT{24A15}\UTFT{24AC0}\UTFT{24F86}\UTFT{2502C}\UTFT{25299}%
\UTFT{25419}\UTFT{25446}\UTFT{2546E}\UTFT{2553F}\UTFT{2555E}\UTFT{25562}\UTFT{25566}\UTFT{257C7}\UTFT{2585D}\UTFT{25903}%
\UTFT{25AAE}\UTFT{25B89}\UTFT{25C06}\UTFT{26102}\UTFT{261B2}\UTFT{26402}\UTFT{2644A}\UTFT{26484}\UTFT{26488}\UTFT{26512}%
\UTFT{265BF}\UTFT{266B5}\UTFT{266FC}\UTFT{26799}\UTFT{2686E}\UTFT{2685E}\UTFT{268C7}\UTFT{26926}\UTFT{26939}\UTFT{269FA}%
\UTFT{26A2D}\UTFT{26A34}\UTFT{26B5B}\UTFT{26B9D}\UTFT{26CA4}\UTFT{26DAE}\UTFT{2704B}\UTFT{271CD}\UTFT{27280}\UTFT{27285}%
\UTFT{2728B}\UTFT{272E6}\UTFT{27450}\UTFT{277CC}\UTFT{27858}\UTFT{279DD}\UTFT{279FD}\UTFT{27A0A}\UTFT{27B0B}\UTFT{27D66}%
\UTFT{28009}\UTFT{28023}\UTFT{28048}\UTFT{28083}\UTFT{28090}\UTFT{280F4}\UTFT{2812E}\UTFT{2814F}\UTFT{281AF}\UTFT{2821A}%
\UTFT{28306}\UTFT{2832F}\UTFT{2838A}\UTFT{28468}\UTFT{286AA}\UTFT{28956}\UTFT{289B8}\UTFT{289E7}\UTFT{289E8}\UTFT{28B46}%
\UTFT{28BD4}\UTFT{28C09}\UTFT{28FC5}\UTFT{290EC}\UTFT{29110}\UTFT{2913C}\UTFT{2915E}\UTFT{24ACA}\UTFT{294E7}\UTFT{295B0}%
\UTFT{295B8}\UTFT{29732}\UTFT{298D1}\UTFT{29949}\UTFT{2996A}\UTFT{299C3}\UTFT{29A28}\UTFT{29B0E}\UTFT{29D5A}\UTFT{29D9B}%
\UTFT{29EF8}\UTFT{29F23}\UTFT{2A293}\UTFT{2A2FF}\UTFT{2A5CB}\UTFT{20C9C}\UTFT{224B0}\UTFT{24A93}\UTFT{28B2C}\UTFT{217F5}%
\UTFT{28B6C}\UTFT{28B99}\UTFT{266AF}\UTFT{27655}\UTFT{25635}\UTFT{25956}\UTFT{25E81}\UTFT{20E6D}\UTFT{23E88}\UTFT{24C9E}%
\UTFT{217F6}\UTFT{2187B}\UTFT{25C4A}\UTFT{25311}\UTFT{25ED8}\UTFT{20FEA}\UTFT{20D49}\UTFT{236BA}\UTFT{25148}\UTFT{210C1}%
\UTFT{24706}\UTFT{26893}\UTFT{226F4}\UTFT{27D2F}\UTFT{241A3}\UTFT{27D73}\UTFT{26ED0}\UTFT{272B6}\UTFT{211D9}\UTFT{23CFC}%
\UTFT{2A6A9}\UTFT{20EAC}\UTFT{21CA2}\UTFT{24FC2}\UTFT{20FEB}\UTFT{22DA0}\UTFT{20FEC}\UTFT{20E0A}\UTFT{20FED}\UTFT{21187}%
\UTFT{24B6E}\UTFT{25A95}\UTFT{20979}\UTFT{22465}\UTFT{23CFE}\UTFT{29F30}\UTFT{24FA9}\UTFT{2959E}\UTFT{23DB6}\UTFT{267B3}%
\UTFT{23720}\UTFT{23EF7}\UTFT{23E2C}\UTFT{230DA}\UTFT{212A9}\UTFT{24963}\UTFT{270AE}\UTFT{2176C}\UTFT{27164}\UTFT{26D22}%
\UTFT{24AE2}\UTFT{2493E}\UTFT{26D23}\UTFT{203FC}\UTFT{23CFD}\UTFT{24919}\UTFT{24A77}\UTFT{28A5A}\UTFT{2F840}\UTFT{2183B}%
\UTFT{26159}\UTFT{233F5}\UTFT{28BC2}\UTFT{21D46}\UTFT{26ED1}\UTFT{28B2D}\UTFT{23CC7}\UTFT{25ED7}\UTFT{27656}\UTFT{25531}%
\UTFT{21944}\UTFT{29903}\UTFT{26DDC}\UTFT{270AD}\UTFT{261AD}\UTFT{28A0F}\UTFT{23677}\UTFT{200EE}\UTFT{26846}\UTFT{24F0E}%
\UTFT{2634C}\UTFT{2626B}\UTFT{21877}\UTFT{2408C}\UTFT{2307E}\UTFT{21E3D}\UTFT{203B5}\UTFT{205C3}\UTFT{21376}\UTFT{24A12}%
\UTFT{28B2B}\UTFT{26083}

Adobe-CNS1-4\\
\UTFT{29C73}\UTFT{2414E}\UTFT{251CD}\UTFT{25D30}\UTFT{28A32}\UTFT{23281}\UTFT{2A107}\UTFT{21980}\UTFT{2870F}\UTFT{2A2BA}%
\UTFT{29947}\UTFT{28AEA}\UTFT{2207E}\UTFT{289E3}\UTFT{21DB6}\UTFT{22712}\UTFT{233F9}\UTFT{23C63}\UTFT{24505}\UTFT{24A13}%
\UTFT{25CA4}\UTFT{25695}\UTFT{28DB9}\UTFT{2143F}\UTFT{2497B}\UTFT{2710D}\UTFT{26D74}\UTFT{26B15}\UTFT{26FBE}

Adobe-CNS1-5\\
\UTFT{27267}\UTFT{27CB1}\UTFT{27CC5}\UTFT{242BF}\UTFT{23617}\UTFT{27352}\UTFT{26E8B}\UTFT{270D2}\UTFT{2A351}\UTFT{27C6C}%
\UTFT{26B23}\UTFT{25A54}\UTFT{21A63}\UTFT{23E06}\UTFT{23F61}\UTFT{28BB9}\UTFT{27BEF}\UTFT{21D5E}\UTFT{29EB0}\UTFT{29945}%
\UTFT{20A6F}\UTFT{23256}\UTFT{22796}\UTFT{23B1A}\UTFT{23551}\UTFT{240EC}\UTFT{21E23}\UTFT{201A4}\UTFT{26C41}\UTFT{20239}%
\UTFT{298FA}\UTFT{20B9F}\UTFT{221C1}\UTFT{2896D}\UTFT{29079}\UTFT{2A1B5}\UTFT{26C46}\UTFT{286B2}\UTFT{273FF}\UTFT{2549A}%
\UTFT{24B0F}

Adobe-CNS1-6\\
\UTFT{21D53}\UTFT{2369E}\UTFT{26021}\UTFT{258DE}\UTFT{24161}\UTFT{2890D}\UTFT{231EA}\UTFT{20A8A}\UTFT{2325E}\UTFT{25DB9}%
\UTFT{2368E}\UTFT{27B65}\UTFT{26E88}\UTFT{25D99}\UTFT{224BC}\UTFT{224C1}\UTFT{224C9}\UTFT{224CC}\UTFT{235BB}\UTFT{2ADFF}%


% end


%
% This file is generated from the data of UniGB-UTF32
% in cid2code.txt (Version 12/05/2017)
% for Adobe-GB1-5
%
% Reference:
%   https://github.com/adobe-type-tools/cmap-resources/
%   Adobe-GB1-5/cid2code.txt
%
% A newer CMap may be required for some code points.
%


Adobe-GB1-2\\
\UTFC{20087}\UTFC{20089}\UTFC{200CC}\UTFC{215D7}\UTFC{2298F}\UTFC{20509}\UTFC{2099D}\UTFC{241FE}

% end


{\bfseries%
[gt/bx]

%
% This file is generated from the data of UniCNS-UTF32
% in cid2code.txt (Version 12/04/2015)
% for Adobe-CNS1-6
%
% Reference:
%   https://github.com/adobe-type-tools/cmap-resources/
%   cmapresources_cns1-6/cid2code.txt
%
% A newer CMap may be required for some code points.
%


Adobe-CNS1-0\\
\UTFT{200CC}\UTFT{2008A}\UTFT{27607}

Adobe-CNS1-1\\
\UTFT{23ED7}\UTFT{26ED3}\UTFT{257E0}\UTFT{28BE9}\UTFT{258E1}\UTFT{294D9}\UTFT{259AC}\UTFT{2648D}\UTFT{25C01}\UTFT{2530E}%
\UTFT{25CFE}\UTFT{25BB4}\UTFT{26C7F}\UTFT{25D20}\UTFT{25CC1}\UTFT{24882}\UTFT{24578}\UTFT{26E44}\UTFT{26ED6}\UTFT{24057}%
\UTFT{26029}\UTFT{217F9}\UTFT{2836D}\UTFT{26121}\UTFT{2615A}\UTFT{262D0}\UTFT{26351}\UTFT{21661}\UTFT{20068}\UTFT{23766}%
\UTFT{2833A}\UTFT{26489}\UTFT{2A087}\UTFT{26CC3}\UTFT{22714}\UTFT{26626}\UTFT{23DE3}\UTFT{266E8}\UTFT{28A48}\UTFT{226F6}%
\UTFT{26498}\UTFT{2148A}\UTFT{2185E}\UTFT{24A65}\UTFT{24A95}\UTFT{26A52}\UTFT{23D7E}\UTFT{214FD}\UTFT{2F98F}\UTFT{249A7}%
\UTFT{23530}\UTFT{21773}\UTFT{23DF8}\UTFT{2F994}\UTFT{20E16}\UTFT{217B4}\UTFT{2317D}\UTFT{2355A}\UTFT{23E8B}\UTFT{26DA3}%
\UTFT{26B05}\UTFT{26B97}\UTFT{235CE}\UTFT{26DA5}\UTFT{26ED4}\UTFT{26E42}\UTFT{25BE4}\UTFT{26B96}\UTFT{26E77}\UTFT{26E43}%
\UTFT{25C91}\UTFT{25CC0}\UTFT{28625}\UTFT{2863B}\UTFT{27088}\UTFT{21582}\UTFT{270CD}\UTFT{2F9B2}\UTFT{218A2}\UTFT{2739A}%
\UTFT{2A0F8}\UTFT{22C27}\UTFT{275E0}\UTFT{23DB9}\UTFT{275E4}\UTFT{2770F}\UTFT{28A25}\UTFT{27924}\UTFT{27ABD}\UTFT{27A59}%
\UTFT{27B3A}\UTFT{27B38}\UTFT{25430}\UTFT{25565}\UTFT{24A7A}\UTFT{216DF}\UTFT{27D54}\UTFT{27D8F}\UTFT{2F9D4}\UTFT{27D53}%
\UTFT{27D98}\UTFT{27DBD}\UTFT{21910}\UTFT{2F9D7}\UTFT{28002}\UTFT{21014}\UTFT{2498A}\UTFT{281BC}\UTFT{2710C}\UTFT{28365}%
\UTFT{28412}\UTFT{2A29F}\UTFT{20A50}\UTFT{289DE}\UTFT{2853D}\UTFT{23DBB}\UTFT{23262}\UTFT{22325}\UTFT{26ED7}\UTFT{2853C}%
\UTFT{27ABE}\UTFT{2856C}\UTFT{2860B}\UTFT{28713}\UTFT{286E6}\UTFT{28933}\UTFT{21E89}\UTFT{255B9}\UTFT{28AC6}\UTFT{23C9B}%
\UTFT{28B0C}\UTFT{255DB}\UTFT{20D31}\UTFT{28AE1}\UTFT{28BEB}\UTFT{28AE2}\UTFT{28AE5}\UTFT{28BEC}\UTFT{28C39}\UTFT{28BFF}%
\UTFT{286D8}\UTFT{2127C}\UTFT{23E2E}\UTFT{26ED5}\UTFT{28AE0}\UTFT{26CB8}\UTFT{20274}\UTFT{26410}\UTFT{290AF}\UTFT{290E5}%
\UTFT{24AD1}\UTFT{21915}\UTFT{2330A}\UTFT{24AE9}\UTFT{291D5}\UTFT{291EB}\UTFT{230B7}\UTFT{230BC}\UTFT{2546C}\UTFT{29433}%
\UTFT{2941D}\UTFT{2797A}\UTFT{27175}\UTFT{20630}\UTFT{2415C}\UTFT{25706}\UTFT{26D27}\UTFT{216D3}\UTFT{24A29}\UTFT{29857}%
\UTFT{29905}\UTFT{25725}\UTFT{290B1}\UTFT{29BD5}\UTFT{29B05}\UTFT{28600}\UTFT{2307D}\UTFT{29D3E}\UTFT{21863}\UTFT{29E68}%
\UTFT{29FB7}\UTFT{2A192}\UTFT{2A1AB}\UTFT{2A0E1}\UTFT{2A123}\UTFT{2A1DF}\UTFT{2A134}\UTFT{2A193}\UTFT{2A220}\UTFT{2193B}%
\UTFT{2A233}\UTFT{2A0B9}\UTFT{2A2B4}\UTFT{24364}\UTFT{28C2B}\UTFT{26DA2}\UTFT{2FA1B}\UTFT{2908B}\UTFT{24975}\UTFT{249BB}%
\UTFT{249F8}\UTFT{24348}\UTFT{24A51}\UTFT{28BDA}\UTFT{218FA}\UTFT{2897E}\UTFT{28E36}\UTFT{28A44}\UTFT{2896C}\UTFT{244B9}%
\UTFT{24473}\UTFT{243F8}\UTFT{217EF}\UTFT{218BE}\UTFT{23599}\UTFT{21885}\UTFT{2542F}\UTFT{217F8}\UTFT{216FB}\UTFT{21839}%
\UTFT{21774}\UTFT{218D1}\UTFT{25F4B}\UTFT{216C0}\UTFT{24A25}\UTFT{213FE}\UTFT{212A8}\UTFT{213C6}\UTFT{214B6}\UTFT{236A6}%
\UTFT{24994}\UTFT{27165}\UTFT{23E31}\UTFT{2555C}\UTFT{23EFB}\UTFT{27052}\UTFT{236EE}\UTFT{2999D}\UTFT{26F26}\UTFT{21922}%
\UTFT{2373F}\UTFT{240E1}\UTFT{2408B}\UTFT{2410F}\UTFT{26C21}\UTFT{266B1}\UTFT{20FDF}\UTFT{20BA8}\UTFT{20E0D}\UTFT{28B13}%
\UTFT{24436}\UTFT{20465}\UTFT{25651}\UTFT{201AB}\UTFT{203CB}\UTFT{2030A}\UTFT{20414}\UTFT{202C0}\UTFT{28EB3}\UTFT{20275}%
\UTFT{2020C}\UTFT{24A0E}\UTFT{23E8A}\UTFT{23595}\UTFT{23E39}\UTFT{23EBF}\UTFT{21884}\UTFT{23E89}\UTFT{205E0}\UTFT{204A3}%
\UTFT{20492}\UTFT{20491}\UTFT{28A9C}\UTFT{2070E}\UTFT{20873}\UTFT{2438C}\UTFT{20C20}\UTFT{249AC}\UTFT{210E4}\UTFT{20E1D}%
\UTFT{24ABC}\UTFT{2408D}\UTFT{240C9}\UTFT{20345}\UTFT{20BC6}\UTFT{28A46}\UTFT{216FA}\UTFT{2176F}\UTFT{21710}\UTFT{25946}%
\UTFT{219F3}\UTFT{21861}\UTFT{24295}\UTFT{25E83}\UTFT{28BD7}\UTFT{20413}\UTFT{21303}\UTFT{289FB}\UTFT{21996}\UTFT{2197C}%
\UTFT{23AEE}\UTFT{21903}\UTFT{21904}\UTFT{218A0}\UTFT{216FE}\UTFT{28A47}\UTFT{21DBA}\UTFT{23472}\UTFT{289A8}\UTFT{21927}%
\UTFT{217AB}\UTFT{2173B}\UTFT{275FD}\UTFT{22860}\UTFT{2262B}\UTFT{225AF}\UTFT{225BE}\UTFT{29088}\UTFT{26F73}\UTFT{2003E}%
\UTFT{20046}\UTFT{2261B}\UTFT{22C9B}\UTFT{22D07}\UTFT{246D4}\UTFT{2914D}\UTFT{24665}\UTFT{22B6A}\UTFT{22B22}\UTFT{23450}%
\UTFT{298EA}\UTFT{22E78}\UTFT{249E3}\UTFT{22D67}\UTFT{22CA1}\UTFT{2308E}\UTFT{232AD}\UTFT{24989}\UTFT{232AB}\UTFT{232E0}%
\UTFT{218D9}\UTFT{2943F}\UTFT{23289}\UTFT{231B3}\UTFT{25584}\UTFT{28B22}\UTFT{2558F}\UTFT{216FC}\UTFT{2555B}\UTFT{25425}%
\UTFT{23103}\UTFT{2182A}\UTFT{23234}\UTFT{2320F}\UTFT{23182}\UTFT{242C9}\UTFT{26D24}\UTFT{27870}\UTFT{21DEB}\UTFT{232D2}%
\UTFT{232E1}\UTFT{25872}\UTFT{2383A}\UTFT{237BC}\UTFT{237A2}\UTFT{233FE}\UTFT{2462A}\UTFT{237D5}\UTFT{24487}\UTFT{21912}%
\UTFT{23FC0}\UTFT{23C9A}\UTFT{28BEA}\UTFT{28ACB}\UTFT{2801E}\UTFT{289DC}\UTFT{23F7F}\UTFT{2403C}\UTFT{2431A}\UTFT{24276}%
\UTFT{2478F}\UTFT{24725}\UTFT{24AA4}\UTFT{205EB}\UTFT{23EF8}\UTFT{2365F}\UTFT{24A4A}\UTFT{24917}\UTFT{25FE1}\UTFT{24ADF}%
\UTFT{28C23}\UTFT{23F35}\UTFT{26DEA}\UTFT{24CD9}\UTFT{24D06}\UTFT{2A5C6}\UTFT{28ACC}\UTFT{249AB}\UTFT{2498E}\UTFT{24A4E}%
\UTFT{249C5}\UTFT{248F3}\UTFT{28AE3}\UTFT{21864}\UTFT{25221}\UTFT{251E7}\UTFT{23232}\UTFT{24697}\UTFT{23781}\UTFT{248F0}%
\UTFT{24ABA}\UTFT{24AC7}\UTFT{24A96}\UTFT{261AE}\UTFT{25581}\UTFT{27741}\UTFT{256E3}\UTFT{23EFA}\UTFT{216E6}\UTFT{20D4C}%
\UTFT{2498C}\UTFT{20299}\UTFT{23DBA}\UTFT{2176E}\UTFT{201D4}\UTFT{20C0D}\UTFT{226F5}\UTFT{25AAF}\UTFT{25A9C}\UTFT{2025B}%
\UTFT{25BC6}\UTFT{25BB3}\UTFT{25EBC}\UTFT{25EA6}\UTFT{249F9}\UTFT{217B0}\UTFT{26261}\UTFT{2615C}\UTFT{27B48}\UTFT{25E82}%
\UTFT{26B75}\UTFT{20916}\UTFT{2004E}\UTFT{235CF}\UTFT{26412}\UTFT{263F8}\UTFT{2082C}\UTFT{25AE9}\UTFT{25D43}\UTFT{25E0E}%
\UTFT{2343F}\UTFT{249F7}\UTFT{265AD}\UTFT{265A0}\UTFT{27127}\UTFT{26CD1}\UTFT{267B4}\UTFT{26A42}\UTFT{26A51}\UTFT{26DA7}%
\UTFT{2721B}\UTFT{21840}\UTFT{218A1}\UTFT{218D8}\UTFT{2F9BC}\UTFT{23D8F}\UTFT{27422}\UTFT{25683}\UTFT{27785}\UTFT{27784}%
\UTFT{28BF5}\UTFT{28BD9}\UTFT{28B9C}\UTFT{289F9}\UTFT{29448}\UTFT{24284}\UTFT{21845}\UTFT{27DDC}\UTFT{24C09}\UTFT{22321}%
\UTFT{217DA}\UTFT{2492F}\UTFT{28A4B}\UTFT{28AFC}\UTFT{28C1D}\UTFT{28C3B}\UTFT{28D34}\UTFT{248FF}\UTFT{24A42}\UTFT{243EA}%
\UTFT{23225}\UTFT{28EE7}\UTFT{28E66}\UTFT{28E65}\UTFT{249ED}\UTFT{24A78}\UTFT{23FEE}\UTFT{290B0}\UTFT{29093}\UTFT{257DF}%
\UTFT{28989}\UTFT{28C26}\UTFT{28B2F}\UTFT{263BE}\UTFT{2421B}\UTFT{20F26}\UTFT{28BC5}\UTFT{24AB2}\UTFT{294DA}\UTFT{295D7}%
\UTFT{28B50}\UTFT{24A67}\UTFT{28B64}\UTFT{28A45}\UTFT{27B06}\UTFT{28B65}\UTFT{258C8}\UTFT{298F1}\UTFT{29948}\UTFT{21302}%
\UTFT{249B8}\UTFT{214E8}\UTFT{2271F}\UTFT{23DB8}\UTFT{22781}\UTFT{2296B}\UTFT{29E2D}\UTFT{2A1F5}\UTFT{2A0FE}\UTFT{24104}%
\UTFT{2A1B4}\UTFT{2A0ED}\UTFT{2A0F3}\UTFT{2992F}\UTFT{26E12}\UTFT{26FDF}\UTFT{26B82}\UTFT{26DA4}\UTFT{26E84}\UTFT{26DF0}%
\UTFT{26E00}\UTFT{237D7}\UTFT{26064}\UTFT{2359C}\UTFT{23640}\UTFT{249DE}\UTFT{202BF}\UTFT{2555D}\UTFT{21757}\UTFT{231C9}%
\UTFT{24941}\UTFT{241B5}\UTFT{241AC}\UTFT{26C40}\UTFT{24F97}\UTFT{217B5}\UTFT{28A49}\UTFT{24488}\UTFT{289FC}\UTFT{218D6}%
\UTFT{20F1D}\UTFT{26CC0}\UTFT{21413}\UTFT{242FA}\UTFT{22C26}\UTFT{243C1}\UTFT{23DB7}\UTFT{26741}\UTFT{2615B}\UTFT{260A4}%
\UTFT{249B9}\UTFT{2498B}\UTFT{289FA}\UTFT{28B63}\UTFT{2189F}\UTFT{24AB3}\UTFT{24A3E}\UTFT{24A94}\UTFT{217D9}\UTFT{24A66}%
\UTFT{203A7}\UTFT{21424}\UTFT{249E5}\UTFT{24916}\UTFT{24976}\UTFT{204FE}\UTFT{28ACE}\UTFT{28A16}\UTFT{28BE7}\UTFT{255D5}%
\UTFT{28A82}\UTFT{24943}\UTFT{20CFF}\UTFT{2061A}\UTFT{20BEB}\UTFT{20CB8}\UTFT{217FA}\UTFT{216C2}\UTFT{24A50}\UTFT{21852}%
\UTFT{28AC0}\UTFT{249AD}\UTFT{218BF}\UTFT{21883}\UTFT{27484}\UTFT{23D5B}\UTFT{28A81}\UTFT{21862}\UTFT{20AB4}\UTFT{2139C}%
\UTFT{28218}\UTFT{290E4}\UTFT{27E4F}\UTFT{23FED}\UTFT{23E2D}\UTFT{203F5}\UTFT{28C1C}\UTFT{26BC0}\UTFT{21452}\UTFT{24362}%
\UTFT{24A71}\UTFT{22FE3}\UTFT{212B0}\UTFT{223BD}\UTFT{21398}\UTFT{234E5}\UTFT{27BF4}\UTFT{236DF}\UTFT{28A83}\UTFT{237D6}%
\UTFT{233FA}\UTFT{24C9F}\UTFT{236AD}\UTFT{26CB7}\UTFT{26D26}\UTFT{26D51}\UTFT{26C82}\UTFT{26FDE}\UTFT{2173A}\UTFT{26C80}%
\UTFT{27053}\UTFT{217DB}\UTFT{217B3}\UTFT{21905}\UTFT{241FC}\UTFT{2173C}\UTFT{242A5}\UTFT{24293}\UTFT{23EF9}\UTFT{27736}%
\UTFT{2445B}\UTFT{242CA}\UTFT{24259}\UTFT{289E1}\UTFT{26D28}\UTFT{244CE}\UTFT{27E4D}\UTFT{243BD}\UTFT{24256}\UTFT{21304}%
\UTFT{243E9}\UTFT{2F825}\UTFT{23300}\UTFT{27AF4}\UTFT{256F6}\UTFT{27B18}\UTFT{27A79}\UTFT{249BA}\UTFT{20346}\UTFT{27657}%
\UTFT{25FE2}\UTFT{275FE}\UTFT{2209A}\UTFT{28A9A}\UTFT{2403B}\UTFT{24A45}\UTFT{205CA}\UTFT{20611}\UTFT{21EA8}\UTFT{23CFF}%
\UTFT{285E8}\UTFT{299C9}\UTFT{221C3}\UTFT{28B4E}\UTFT{20C78}\UTFT{20779}\UTFT{23F4A}\UTFT{24AA7}\UTFT{26B52}\UTFT{27632}%
\UTFT{2493F}\UTFT{233CC}\UTFT{28948}\UTFT{21D90}\UTFT{27C12}\UTFT{24F9A}\UTFT{26BF7}\UTFT{2191C}\UTFT{249F6}\UTFT{23FEF}%
\UTFT{2271B}\UTFT{257E1}\UTFT{2F8CD}\UTFT{2F806}\UTFT{24521}\UTFT{24934}\UTFT{26CBD}\UTFT{26411}\UTFT{290C0}\UTFT{20A11}%
\UTFT{26469}\UTFT{20021}\UTFT{23519}\UTFT{2258D}\UTFT{2217A}\UTFT{249D0}\UTFT{20EF8}\UTFT{22926}\UTFT{28473}\UTFT{217B1}%
\UTFT{24A2A}\UTFT{21820}\UTFT{29CAD}\UTFT{298A4}\UTFT{2160A}\UTFT{2372F}\UTFT{280E8}\UTFT{213C5}\UTFT{291A8}\UTFT{270AF}%
\UTFT{289AB}\UTFT{2417A}\UTFT{2A2DF}\UTFT{28318}\UTFT{26E07}\UTFT{2816F}\UTFT{269B5}\UTFT{213ED}\UTFT{2322F}\UTFT{28C30}%
\UTFT{28949}\UTFT{24988}\UTFT{24AA5}\UTFT{23F81}\UTFT{21FA1}\UTFT{295E9}\UTFT{2789D}\UTFT{28024}\UTFT{27A3E}\UTFT{23CB7}%
\UTFT{26258}\UTFT{29D98}\UTFT{23D40}\UTFT{20E9D}\UTFT{282E2}\UTFT{20C41}\UTFT{20C96}\UTFT{20E76}\UTFT{22C62}\UTFT{20EA2}%
\UTFT{21075}\UTFT{22B43}\UTFT{22EB3}\UTFT{20DA7}\UTFT{2688A}\UTFT{20EF9}\UTFT{27FF9}\UTFT{247E0}\UTFT{29D7C}\UTFT{275A3}%
\UTFT{26048}\UTFT{24618}\UTFT{29EAC}\UTFT{29FDE}\UTFT{272B2}\UTFT{2048E}\UTFT{20EB6}\UTFT{27F2E}\UTFT{2A434}\UTFT{243F2}%
\UTFT{29E06}\UTFT{294D0}\UTFT{26335}\UTFT{20D28}\UTFT{20D71}\UTFT{21F0F}\UTFT{21DD1}\UTFT{2176D}\UTFT{2B473}\UTFT{28E97}%
\UTFT{25C21}\UTFT{20CD4}\UTFT{201F2}\UTFT{2A64A}\UTFT{2837D}\UTFT{2A2B2}\UTFT{24ABB}\UTFT{26E05}\UTFT{2AE67}\UTFT{2251B}%
\UTFT{28E39}\UTFT{20F3B}\UTFT{25F1A}\UTFT{27486}\UTFT{267CC}\UTFT{24011}\UTFT{2F922}\UTFT{20547}\UTFT{205DF}\UTFT{23FC5}%
\UTFT{24942}\UTFT{289E4}\UTFT{219DB}\UTFT{23CC8}\UTFT{24933}\UTFT{289AA}\UTFT{202A0}\UTFT{26BB3}\UTFT{21305}\UTFT{224ED}%
\UTFT{26D29}\UTFT{27A84}\UTFT{23600}\UTFT{24AB1}\UTFT{22513}\UTFT{2037E}\UTFT{20380}\UTFT{20347}\UTFT{2041F}\UTFT{249A4}%
\UTFT{20487}\UTFT{233B4}\UTFT{20BFF}\UTFT{220FC}\UTFT{202E5}\UTFT{22530}\UTFT{2058E}\UTFT{23233}\UTFT{21983}\UTFT{205B3}%
\UTFT{23C99}\UTFT{24AA6}\UTFT{2372D}\UTFT{26B13}\UTFT{2F829}\UTFT{28ADE}\UTFT{23F80}\UTFT{20954}\UTFT{23FEC}\UTFT{20BE2}%
\UTFT{21726}\UTFT{216E8}\UTFT{286AB}\UTFT{2F832}\UTFT{21596}\UTFT{21613}\UTFT{28A9B}\UTFT{25772}\UTFT{20B8F}\UTFT{23FEB}%
\UTFT{22DA3}\UTFT{20C77}\UTFT{26B53}\UTFT{20D74}\UTFT{2170D}\UTFT{20EDD}\UTFT{20D4D}\UTFT{289BC}\UTFT{22698}\UTFT{218D7}%
\UTFT{2403A}\UTFT{24435}\UTFT{210B4}\UTFT{2328A}\UTFT{28B66}\UTFT{2124F}\UTFT{241A5}\UTFT{26C7E}\UTFT{21416}\UTFT{21454}%
\UTFT{24363}\UTFT{24BF5}\UTFT{2123C}\UTFT{2A150}\UTFT{24278}\UTFT{2163E}\UTFT{21692}\UTFT{20D4E}\UTFT{26C81}\UTFT{26D2A}%
\UTFT{217DC}\UTFT{217FB}\UTFT{217B2}\UTFT{26DA6}\UTFT{21828}\UTFT{216D5}\UTFT{26E45}\UTFT{249A9}\UTFT{26FA1}\UTFT{22554}%
\UTFT{21911}\UTFT{216B8}\UTFT{27A0E}\UTFT{20204}\UTFT{21A34}\UTFT{259CC}\UTFT{205A5}\UTFT{21B44}\UTFT{21CA5}\UTFT{26B28}%
\UTFT{21DF9}\UTFT{21E37}\UTFT{21EA4}\UTFT{24901}\UTFT{22049}\UTFT{22173}\UTFT{244BC}\UTFT{20CD3}\UTFT{21771}\UTFT{28482}%
\UTFT{201C1}\UTFT{2F894}\UTFT{2133A}\UTFT{26888}\UTFT{223D0}\UTFT{22471}\UTFT{26E6E}\UTFT{28A36}\UTFT{25250}\UTFT{21F6A}%
\UTFT{270F8}\UTFT{22668}\UTFT{2029E}\UTFT{28A29}\UTFT{227B4}\UTFT{24982}\UTFT{2498F}\UTFT{27A53}\UTFT{2F8A6}\UTFT{26ED2}%
\UTFT{20656}\UTFT{23FB7}\UTFT{2285F}\UTFT{28B9D}\UTFT{2995D}\UTFT{22980}\UTFT{228C1}\UTFT{20118}\UTFT{21770}\UTFT{22E0D}%
\UTFT{249DF}\UTFT{2138E}\UTFT{217FC}\UTFT{22E36}\UTFT{2571D}\UTFT{24A28}\UTFT{24A23}\UTFT{24940}\UTFT{21829}\UTFT{23400}%
\UTFT{231F7}\UTFT{231F8}\UTFT{231A4}\UTFT{231A5}\UTFT{20E75}\UTFT{251E6}\UTFT{23231}\UTFT{285F4}\UTFT{231C8}\UTFT{25313}%
\UTFT{228F7}\UTFT{2439C}\UTFT{24A21}\UTFT{237C2}\UTFT{2F8DB}\UTFT{241CD}\UTFT{290ED}\UTFT{233E6}\UTFT{26DA0}\UTFT{2346F}%
\UTFT{28ADF}\UTFT{235CD}\UTFT{2363C}\UTFT{28A4A}\UTFT{203C9}\UTFT{23659}\UTFT{2212A}\UTFT{23703}\UTFT{2919C}\UTFT{20923}%
\UTFT{227CD}\UTFT{23ADB}\UTFT{21958}\UTFT{23B5A}\UTFT{23EFC}\UTFT{2248B}\UTFT{248F1}\UTFT{26B51}\UTFT{23DBC}\UTFT{23DBD}%
\UTFT{241A4}\UTFT{2490C}\UTFT{24900}\UTFT{23CC9}\UTFT{20D32}\UTFT{231F9}\UTFT{22491}\UTFT{26D25}\UTFT{26DA1}\UTFT{26DEB}%
\UTFT{2497F}\UTFT{24085}\UTFT{26E72}\UTFT{26F74}\UTFT{28B21}\UTFT{2F908}\UTFT{23E2F}\UTFT{23F82}\UTFT{2304B}\UTFT{23E30}%
\UTFT{21497}\UTFT{2403D}\UTFT{29170}\UTFT{24144}\UTFT{24091}\UTFT{24155}\UTFT{24039}\UTFT{23FF0}\UTFT{23FB4}\UTFT{2413F}%
\UTFT{24156}\UTFT{24157}\UTFT{24140}\UTFT{261DD}\UTFT{24277}\UTFT{24365}\UTFT{242C1}\UTFT{2445A}\UTFT{24A27}\UTFT{24A22}%
\UTFT{28BE8}\UTFT{25605}\UTFT{24974}\UTFT{23044}\UTFT{24823}\UTFT{2882B}\UTFT{28804}\UTFT{20C3A}\UTFT{26A2E}\UTFT{241E2}%
\UTFT{216E7}\UTFT{24A24}\UTFT{249B7}\UTFT{2498D}\UTFT{249FB}\UTFT{24A26}\UTFT{2F92F}\UTFT{228AD}\UTFT{28EB2}\UTFT{24A8C}%
\UTFT{2415F}\UTFT{24A79}\UTFT{28B8F}\UTFT{28C03}\UTFT{2189E}\UTFT{21988}\UTFT{28ED9}\UTFT{21A4B}\UTFT{28EAC}\UTFT{24F82}%
\UTFT{24D13}\UTFT{263F5}\UTFT{26911}\UTFT{2690E}\UTFT{26F9F}\UTFT{2509D}\UTFT{2517D}\UTFT{21E1C}\UTFT{25220}\UTFT{232AC}%
\UTFT{28964}\UTFT{28968}\UTFT{216C1}\UTFT{255E0}\UTFT{2760C}\UTFT{2261C}\UTFT{25857}\UTFT{27B39}\UTFT{27126}\UTFT{2910D}%
\UTFT{20C42}\UTFT{20D15}\UTFT{2512B}\UTFT{22CC6}\UTFT{20341}\UTFT{24DB8}\UTFT{294E5}\UTFT{280BE}\UTFT{22C38}\UTFT{2815D}%
\UTFT{269F2}\UTFT{24DEA}\UTFT{20D7C}\UTFT{20FB4}\UTFT{20CD5}\UTFT{2BAB3}\UTFT{20E96}\UTFT{20F64}\UTFT{22CA9}\UTFT{28256}%
\UTFT{244D3}\UTFT{20D46}\UTFT{29A4D}\UTFT{280E9}\UTFT{24EA7}\UTFT{22CC2}\UTFT{295F4}\UTFT{252C7}\UTFT{297D4}\UTFT{22D44}%
\UTFT{2BCD7}\UTFT{22BCA}\UTFT{2B977}\UTFT{266DA}\UTFT{26716}\UTFT{279A0}\UTFT{25052}\UTFT{20C43}\UTFT{28B4C}\UTFT{20731}%
\UTFT{201A9}\UTFT{22D8D}\UTFT{245C8}\UTFT{204FC}\UTFT{26097}\UTFT{20F4C}\UTFT{22A66}\UTFT{2109D}\UTFT{20D9C}\UTFT{22775}%
\UTFT{2A601}\UTFT{20E09}\UTFT{22ACF}\UTFT{2C5F8}\UTFT{210C8}\UTFT{239C2}\UTFT{2829B}\UTFT{25E49}\UTFT{220C7}\UTFT{22CB2}%
\UTFT{29720}\UTFT{24E3B}\UTFT{2C9A0}\UTFT{27574}\UTFT{22E8B}\UTFT{22208}\UTFT{2A65B}\UTFT{28CCD}\UTFT{20E7A}\UTFT{20C34}%
\UTFT{27639}\UTFT{22BCE}\UTFT{22C51}\UTFT{210C7}\UTFT{2A632}\UTFT{28CD2}\UTFT{28D99}\UTFT{28CCA}\UTFT{2775E}\UTFT{2F828}%
\UTFT{2107B}\UTFT{210D3}\UTFT{212FE}\UTFT{247EF}\UTFT{24EA5}\UTFT{24F5C}\UTFT{28189}\UTFT{2B42C}

Adobe-CNS1-3\\
\UTFT{2010C}\UTFT{200D1}\UTFT{200CD}\UTFT{200CB}\UTFT{21FE8}\UTFT{200CA}\UTFT{2010E}\UTFT{21BC1}\UTFT{2F878}\UTFT{20086}%
\UTFT{248E9}\UTFT{2626A}\UTFT{2634B}\UTFT{26612}\UTFT{26951}\UTFT{278B2}\UTFT{28E0F}\UTFT{29810}\UTFT{20087}\UTFT{2A3A9}%
\UTFT{21145}\UTFT{27735}\UTFT{209E7}\UTFT{29DF6}\UTFT{2700E}\UTFT{2A133}\UTFT{2846C}\UTFT{21DCA}\UTFT{205D0}\UTFT{22AE6}%
\UTFT{27D84}\UTFT{210F4}\UTFT{20C0B}\UTFT{278C8}\UTFT{260A5}\UTFT{22D4C}\UTFT{21077}\UTFT{2106F}\UTFT{221A1}\UTFT{20D96}%
\UTFT{22CC9}\UTFT{20F31}\UTFT{2681C}\UTFT{210CF}\UTFT{22803}\UTFT{22939}\UTFT{251E3}\UTFT{20E8C}\UTFT{20F8D}\UTFT{20EAA}%
\UTFT{20F30}\UTFT{20D47}\UTFT{2114F}\UTFT{20E4C}\UTFT{20EAB}\UTFT{20BA9}\UTFT{20D48}\UTFT{210C0}\UTFT{2113D}\UTFT{22696}%
\UTFT{20FAD}\UTFT{233F4}\UTFT{20D7E}\UTFT{20D7F}\UTFT{22C55}\UTFT{20E98}\UTFT{20F2E}\UTFT{26B50}\UTFT{29EC3}\UTFT{22DEE}%
\UTFT{26572}\UTFT{280BD}\UTFT{20EFA}\UTFT{20E0F}\UTFT{20E77}\UTFT{20EFB}\UTFT{24DEB}\UTFT{20CD6}\UTFT{227B5}\UTFT{210C9}%
\UTFT{20E10}\UTFT{20E78}\UTFT{21078}\UTFT{21148}\UTFT{28207}\UTFT{21455}\UTFT{20E79}\UTFT{24E50}\UTFT{22DA4}\UTFT{2101D}%
\UTFT{2101E}\UTFT{210F5}\UTFT{210F6}\UTFT{20E11}\UTFT{27694}\UTFT{282CD}\UTFT{20FB5}\UTFT{20E7B}\UTFT{2517E}\UTFT{20FB6}%
\UTFT{21180}\UTFT{252D8}\UTFT{2A2BD}\UTFT{249DA}\UTFT{2183A}\UTFT{24177}\UTFT{2827C}\UTFT{2573D}\UTFT{25B74}\UTFT{2313D}%
\UTFT{289C0}\UTFT{23F41}\UTFT{20325}\UTFT{20ED8}\UTFT{25C65}\UTFT{24FB8}\UTFT{20B0D}\UTFT{26B0A}\UTFT{22EEF}\UTFT{23CB5}%
\UTFT{26E99}\UTFT{23F8F}\UTFT{24CC9}\UTFT{2A014}\UTFT{286BC}\UTFT{28501}\UTFT{2267A}\UTFT{269A8}\UTFT{2424B}\UTFT{2215B}%
\UTFT{2037F}\UTFT{2A45B}\UTFT{249EC}\UTFT{24962}\UTFT{27109}\UTFT{24A4F}\UTFT{24A5D}\UTFT{217DF}\UTFT{23AFA}\UTFT{20214}%
\UTFT{208D5}\UTFT{20619}\UTFT{21F9E}\UTFT{2A2B6}\UTFT{2915B}\UTFT{28A59}\UTFT{29420}\UTFT{248F2}\UTFT{25535}\UTFT{20CCF}%
\UTFT{27967}\UTFT{21BC2}\UTFT{20094}\UTFT{202B7}\UTFT{203A0}\UTFT{204D7}\UTFT{205D5}\UTFT{20615}\UTFT{20676}\UTFT{216BA}%
\UTFT{20AC2}\UTFT{20ACD}\UTFT{20BBF}\UTFT{2F83B}\UTFT{20BCB}\UTFT{20BFB}\UTFT{20C3B}\UTFT{20C53}\UTFT{20C65}\UTFT{20C7C}%
\UTFT{20C8D}\UTFT{20CB5}\UTFT{20CDD}\UTFT{20CED}\UTFT{20D6F}\UTFT{20DB2}\UTFT{20DC8}\UTFT{20E04}\UTFT{20E0E}\UTFT{20ED7}%
\UTFT{20F90}\UTFT{20F2D}\UTFT{20E73}\UTFT{20FBC}\UTFT{2105C}\UTFT{2104F}\UTFT{21076}\UTFT{21088}\UTFT{21096}\UTFT{210BF}%
\UTFT{2112F}\UTFT{2113B}\UTFT{212E3}\UTFT{21375}\UTFT{21336}\UTFT{21577}\UTFT{21619}\UTFT{217C3}\UTFT{217C7}\UTFT{2182D}%
\UTFT{2196A}\UTFT{21A2D}\UTFT{21A45}\UTFT{21C2A}\UTFT{21C70}\UTFT{21CAC}\UTFT{21EC8}\UTFT{21ED5}\UTFT{21F15}\UTFT{22045}%
\UTFT{2227C}\UTFT{223D7}\UTFT{223FA}\UTFT{2272A}\UTFT{22871}\UTFT{2294F}\UTFT{22967}\UTFT{22993}\UTFT{22AD5}\UTFT{22AE8}%
\UTFT{22B0E}\UTFT{22B3F}\UTFT{22C4C}\UTFT{22C88}\UTFT{22CB7}\UTFT{25BE8}\UTFT{22D08}\UTFT{22D12}\UTFT{22DB7}\UTFT{22D95}%
\UTFT{22E42}\UTFT{22F74}\UTFT{22FCC}\UTFT{23033}\UTFT{23066}\UTFT{2331F}\UTFT{233DE}\UTFT{23567}\UTFT{235F3}\UTFT{2361A}%
\UTFT{23716}\UTFT{23AA7}\UTFT{23E11}\UTFT{23EB9}\UTFT{24119}\UTFT{242EE}\UTFT{2430D}\UTFT{24334}\UTFT{24396}\UTFT{24404}%
\UTFT{244D6}\UTFT{24674}\UTFT{2472F}\UTFT{24812}\UTFT{248FB}\UTFT{24A15}\UTFT{24AC0}\UTFT{24F86}\UTFT{2502C}\UTFT{25299}%
\UTFT{25419}\UTFT{25446}\UTFT{2546E}\UTFT{2553F}\UTFT{2555E}\UTFT{25562}\UTFT{25566}\UTFT{257C7}\UTFT{2585D}\UTFT{25903}%
\UTFT{25AAE}\UTFT{25B89}\UTFT{25C06}\UTFT{26102}\UTFT{261B2}\UTFT{26402}\UTFT{2644A}\UTFT{26484}\UTFT{26488}\UTFT{26512}%
\UTFT{265BF}\UTFT{266B5}\UTFT{266FC}\UTFT{26799}\UTFT{2686E}\UTFT{2685E}\UTFT{268C7}\UTFT{26926}\UTFT{26939}\UTFT{269FA}%
\UTFT{26A2D}\UTFT{26A34}\UTFT{26B5B}\UTFT{26B9D}\UTFT{26CA4}\UTFT{26DAE}\UTFT{2704B}\UTFT{271CD}\UTFT{27280}\UTFT{27285}%
\UTFT{2728B}\UTFT{272E6}\UTFT{27450}\UTFT{277CC}\UTFT{27858}\UTFT{279DD}\UTFT{279FD}\UTFT{27A0A}\UTFT{27B0B}\UTFT{27D66}%
\UTFT{28009}\UTFT{28023}\UTFT{28048}\UTFT{28083}\UTFT{28090}\UTFT{280F4}\UTFT{2812E}\UTFT{2814F}\UTFT{281AF}\UTFT{2821A}%
\UTFT{28306}\UTFT{2832F}\UTFT{2838A}\UTFT{28468}\UTFT{286AA}\UTFT{28956}\UTFT{289B8}\UTFT{289E7}\UTFT{289E8}\UTFT{28B46}%
\UTFT{28BD4}\UTFT{28C09}\UTFT{28FC5}\UTFT{290EC}\UTFT{29110}\UTFT{2913C}\UTFT{2915E}\UTFT{24ACA}\UTFT{294E7}\UTFT{295B0}%
\UTFT{295B8}\UTFT{29732}\UTFT{298D1}\UTFT{29949}\UTFT{2996A}\UTFT{299C3}\UTFT{29A28}\UTFT{29B0E}\UTFT{29D5A}\UTFT{29D9B}%
\UTFT{29EF8}\UTFT{29F23}\UTFT{2A293}\UTFT{2A2FF}\UTFT{2A5CB}\UTFT{20C9C}\UTFT{224B0}\UTFT{24A93}\UTFT{28B2C}\UTFT{217F5}%
\UTFT{28B6C}\UTFT{28B99}\UTFT{266AF}\UTFT{27655}\UTFT{25635}\UTFT{25956}\UTFT{25E81}\UTFT{20E6D}\UTFT{23E88}\UTFT{24C9E}%
\UTFT{217F6}\UTFT{2187B}\UTFT{25C4A}\UTFT{25311}\UTFT{25ED8}\UTFT{20FEA}\UTFT{20D49}\UTFT{236BA}\UTFT{25148}\UTFT{210C1}%
\UTFT{24706}\UTFT{26893}\UTFT{226F4}\UTFT{27D2F}\UTFT{241A3}\UTFT{27D73}\UTFT{26ED0}\UTFT{272B6}\UTFT{211D9}\UTFT{23CFC}%
\UTFT{2A6A9}\UTFT{20EAC}\UTFT{21CA2}\UTFT{24FC2}\UTFT{20FEB}\UTFT{22DA0}\UTFT{20FEC}\UTFT{20E0A}\UTFT{20FED}\UTFT{21187}%
\UTFT{24B6E}\UTFT{25A95}\UTFT{20979}\UTFT{22465}\UTFT{23CFE}\UTFT{29F30}\UTFT{24FA9}\UTFT{2959E}\UTFT{23DB6}\UTFT{267B3}%
\UTFT{23720}\UTFT{23EF7}\UTFT{23E2C}\UTFT{230DA}\UTFT{212A9}\UTFT{24963}\UTFT{270AE}\UTFT{2176C}\UTFT{27164}\UTFT{26D22}%
\UTFT{24AE2}\UTFT{2493E}\UTFT{26D23}\UTFT{203FC}\UTFT{23CFD}\UTFT{24919}\UTFT{24A77}\UTFT{28A5A}\UTFT{2F840}\UTFT{2183B}%
\UTFT{26159}\UTFT{233F5}\UTFT{28BC2}\UTFT{21D46}\UTFT{26ED1}\UTFT{28B2D}\UTFT{23CC7}\UTFT{25ED7}\UTFT{27656}\UTFT{25531}%
\UTFT{21944}\UTFT{29903}\UTFT{26DDC}\UTFT{270AD}\UTFT{261AD}\UTFT{28A0F}\UTFT{23677}\UTFT{200EE}\UTFT{26846}\UTFT{24F0E}%
\UTFT{2634C}\UTFT{2626B}\UTFT{21877}\UTFT{2408C}\UTFT{2307E}\UTFT{21E3D}\UTFT{203B5}\UTFT{205C3}\UTFT{21376}\UTFT{24A12}%
\UTFT{28B2B}\UTFT{26083}

Adobe-CNS1-4\\
\UTFT{29C73}\UTFT{2414E}\UTFT{251CD}\UTFT{25D30}\UTFT{28A32}\UTFT{23281}\UTFT{2A107}\UTFT{21980}\UTFT{2870F}\UTFT{2A2BA}%
\UTFT{29947}\UTFT{28AEA}\UTFT{2207E}\UTFT{289E3}\UTFT{21DB6}\UTFT{22712}\UTFT{233F9}\UTFT{23C63}\UTFT{24505}\UTFT{24A13}%
\UTFT{25CA4}\UTFT{25695}\UTFT{28DB9}\UTFT{2143F}\UTFT{2497B}\UTFT{2710D}\UTFT{26D74}\UTFT{26B15}\UTFT{26FBE}

Adobe-CNS1-5\\
\UTFT{27267}\UTFT{27CB1}\UTFT{27CC5}\UTFT{242BF}\UTFT{23617}\UTFT{27352}\UTFT{26E8B}\UTFT{270D2}\UTFT{2A351}\UTFT{27C6C}%
\UTFT{26B23}\UTFT{25A54}\UTFT{21A63}\UTFT{23E06}\UTFT{23F61}\UTFT{28BB9}\UTFT{27BEF}\UTFT{21D5E}\UTFT{29EB0}\UTFT{29945}%
\UTFT{20A6F}\UTFT{23256}\UTFT{22796}\UTFT{23B1A}\UTFT{23551}\UTFT{240EC}\UTFT{21E23}\UTFT{201A4}\UTFT{26C41}\UTFT{20239}%
\UTFT{298FA}\UTFT{20B9F}\UTFT{221C1}\UTFT{2896D}\UTFT{29079}\UTFT{2A1B5}\UTFT{26C46}\UTFT{286B2}\UTFT{273FF}\UTFT{2549A}%
\UTFT{24B0F}

Adobe-CNS1-6\\
\UTFT{21D53}\UTFT{2369E}\UTFT{26021}\UTFT{258DE}\UTFT{24161}\UTFT{2890D}\UTFT{231EA}\UTFT{20A8A}\UTFT{2325E}\UTFT{25DB9}%
\UTFT{2368E}\UTFT{27B65}\UTFT{26E88}\UTFT{25D99}\UTFT{224BC}\UTFT{224C1}\UTFT{224C9}\UTFT{224CC}\UTFT{235BB}\UTFT{2ADFF}%


% end


%
% This file is generated from the data of UniGB-UTF32
% in cid2code.txt (Version 12/05/2017)
% for Adobe-GB1-5
%
% Reference:
%   https://github.com/adobe-type-tools/cmap-resources/
%   Adobe-GB1-5/cid2code.txt
%
% A newer CMap may be required for some code points.
%


Adobe-GB1-2\\
\UTFC{20087}\UTFC{20089}\UTFC{200CC}\UTFC{215D7}\UTFC{2298F}\UTFC{20509}\UTFC{2099D}\UTFC{241FE}

% end


}}

\end{document}
