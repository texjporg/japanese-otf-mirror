% -*- coding: utf-8 -*-
%%%%%%%%
% To control hyperref on command line,
% you can select one of (1),(2a),(2b),(3).
%   (1) do not treat hyperref
%   $ uplatex uotf-sp-utf8.tex
%   (2a) hyperref + dvipdfmx           (with    CMap conversion)
%   $ uplatex "\def\withhyperref{dvipdfmx}\input" uotf-sp-utf8.tex
%   (2b) hyperref + dvipdfmx + out2uni (without CMap conversion)
%   $ uplatex "\def\withhyperref{dvipdfmx}\nocmap{true}\input" uotf-sp-utf8.tex
%   (3) hyperref + dvips + convbkmk.rb + distiller/ps2pdf
%   $ uplatex "\def\withhyperref{dvips}\input" uotf-sp-utf8.tex
%%%%%%

\newif\ifuptexmode\uptexmodefalse
\ifnum\jis"2121="3000
 \uptexmodetrue
 \def\tounicode{pdf:tounicode UTF8-UTF16}
\else
 \ifnum\jis"2121="A1A1
  \def\tounicode{pdf:tounicode EUC-UCS2}
 \fi
 \ifnum\jis"2121="8140
  \def\tounicode{pdf:tounicode 90ms-RKSJ-UCS2}
 \fi
\fi

\makeatletter

\def\@opt@{multi}
\def\@default{default}
\def\@jarticle{jarticle}
\def\@tarticle{tarticle}
\def\@ujarticle{ujarticle}
\def\@noreplace{noreplace}

\ifx\option\@undefined
 \def\option{default}
\fi
\ifx\option\@noreplace
 \ifuptexmode
  \ifx\class\@ujarticle
   \def\@enc@{JY2}\def\@dir@{h}
  \else
   \def\@enc@{JT2}\def\@dir@{v}
  \fi
  \DeclareFontFamily{\@enc@}{mcw}{}
  \DeclareFontFamily{\@enc@}{gtw}{}
  \DeclareFontShape{\@enc@}{mcw}{m}{n}{<->s*[0.962216]upjpnrm-\@dir@}{}
  \DeclareFontShape{\@enc@}{gtw}{m}{n}{<->s*[0.962216]upjpngt-\@dir@}{}
  \DeclareFontShape{\@enc@}{gt}{m}{n}{<->s*[0.962216]upjpngt-\@dir@}{}
  \DeclareFontShape{\@enc@}{mcw}{bx}{n}{<->ssub*gt/m/n}{}
  \DeclareFontShape{\@enc@}{gtw}{bx}{n}{<->ssub*gt/m/n}{}
  \DeclareFontShape{\@enc@}{gt}{bx}{n}{<->ssub*gt/m/n}{}
  \DeclareRobustCommand\mcw{\kanjifamily{mcw}\selectfont}
  \DeclareRobustCommand\gtw{\kanjifamily{gtw}\selectfont}
  \renewcommand\mcdefault{mcw}
  \renewcommand\gtdefault{gtw}
 \fi
\fi
\ifx\option\@default
\else
 \edef\@opt@{\option,\@opt@}
\fi

\ifx\class\@undefined
 \ifuptexmode
  \def\class{ujarticle}
 \else
  \def\class{jarticle}
 \fi
\fi
\ifuptexmode
 \edef\@opt@{uplatex,\@opt@}
\fi
\ifx\class\@jarticle
  \documentclass[a4paper,titlepage,dvipdfmx]{\class}
\else
 \ifx\class\@ujarticle
  \documentclass[a4paper,titlepage,dvipdfmx]{\class}
 \else
  \documentclass[a4paper,titlepage,landscape,dvipdfmx]{\class}
 \fi
\fi

\usepackage[\@opt@]{otf}

\def\@dvipdfmx{dvipdfmx}
\def\@dvips{dvips}

\ifx\withhyperref\@undefined
 \def\withhyperref{undefined}
 \edef\texorpdfstring#1#2{#1}
\else
 \ifx\withhyperref\@dvipdfmx
  \def\@hyperrefkeyval{dvipdfm}
  \usepackage{atbegshi}
  \ifx\nocmap\@undefined
   \AtBeginShipoutFirst{\special{\tounicode}}
  \fi
 \fi
 \ifx\withhyperref\@dvips
  \def\@hyperrefkeyval{dvips}
 \fi

\usepackage[\@hyperrefkeyval,%
bookmarks=true,%
bookmarksnumbered=true,%
bookmarkstype=toc,%
%pdfstartview={FitBH -32768},%
pdftitle={japanese-otfのテスト},%
pdfsubject={Unicode supplementary plane編},%
pdfauthor={upTeXプロジェクト},%
pdfkeywords={TeX; dvips; dvipdfmx; bookmark; hyperref; しおり; pdf}%
]{hyperref}

\fi

\makeatother

\usepackage{redeffont}

\ajUTFVarDef{叱}{20B9F}
\ajUTFVarDef{亭}{20158}
\ajUTFVarDef{吉}{20bb7}
\ajUTFVarDef{座}{2B776}

\AtBeginDvi{\special{papersize=\the\paperwidth,\the\paperheight}}
\pagestyle{empty}
\makeatletter
\ifx\rubyfamily\@undefined\let\rubyfamily=\relax\fi
\ifx\mgfamily\@undefined\let\mgfamily=\relax\fi
\makeatother

\edef\bs{$\backslash$\kern0em}
\setlength\parindent{0em}

\begin{document}
\section{見出し}

コンパイラー:\ifuptexmode upLaTeX\else pLaTeX\fi\\
クラス:\texttt{\class}\\
オプション:\texttt{\option}

\vspace{\baselineskip}
\ifuptexmode
\begin{tabular}{l||ccccccc}
フォント & 仮名 & 漢字 & UTF-8 & \bs kchar & \bs UTF & \bs CID\\
\hline
mc/m & ひらかな & 漢字 & 𠮟𠅘 & \kchar"20B9F\kchar"20158 & \UTF{20B9F}\UTF{20158} & \CID{13803}\CID{20075} \\
mc/bx & {\bfseries ひらかな} & {\bfseries 漢字} & {\bfseries 𠮟𠅘} & {\bfseries\kchar"20B9F\kchar"20158} & {\bfseries\UTF{20B9F}\UTF{20158}} & {\bfseries\CID{13803}\CID{20075}} \\
gt/m & {\gtfamily ひらかな} & {\gtfamily 漢字} & {\gtfamily 𠮟𠅘} & {\gtfamily\kchar"20B9F\kchar"20158} & {\gtfamily\UTF{20B9F}\UTF{20158}} & {\gtfamily\CID{13803}\CID{20075}} \\
gt/bx & {\gtfamily\bfseries ひらかな} & {\gtfamily\bfseries 漢字} & {\gtfamily\bfseries 𠮟𠅘} & {\gtfamily\bfseries\kchar"20B9F\kchar"20158} & {\gtfamily\bfseries\UTF{20B9F}\UTF{20158}} & {\gtfamily\bfseries\CID{13803}\CID{20075}} \\
mg/m & {\mgfamily ひらかな} & {\mgfamily 漢字} & {\mgfamily 𠮟𠅘} & {\mgfamily\kchar"20B9F\kchar"20158} & {\mgfamily\UTF{20B9F}\UTF{20158}} & {\mgfamily\CID{13803}\CID{20075}} \\
\end{tabular}
\else
\begin{tabular}{l||ccccc}
フォント & 仮名 & 漢字 & \bs UTF & \bs CID\\
\hline
mc/m & ひらかな & 漢字 & \UTF{20B9F}\UTF{20158} & \CID{13803}\CID{20075} \\
mc/bx & {\bfseries ひらかな} & {\bfseries 漢字} & {\bfseries\UTF{20B9F}\UTF{20158}} & {\bfseries\CID{13803}\CID{20075}} \\
gt/m & {\gtfamily ひらかな} & {\gtfamily 漢字} & {\gtfamily\UTF{20B9F}\UTF{20158}} & {\gtfamily\CID{13803}\CID{20075}} \\
gt/bx & {\gtfamily\bfseries ひらかな} & {\gtfamily\bfseries 漢字} & {\gtfamily\bfseries\UTF{20B9F}\UTF{20158}} & {\gtfamily\bfseries\CID{13803}\CID{20075}} \\
mg/m & {\mgfamily ひらかな} & {\mgfamily 漢字} & {\mgfamily\UTF{20B9F}\UTF{20158}} & {\mgfamily\CID{13803}\CID{20075}} \\
\end{tabular}
\fi
\vspace{\baselineskip}

日本:\UTF{20509}\UTF{241FE} 簡体字:\UTFC{20509}\UTFC{241FE}  多言語:\UTFM{20509}\UTFM{241FE}

日本:\UTF{20b9f}\UTF{26402} 繁體字:\UTFT{20b9f}\UTFT{26402}  多言語:\UTFM{20b9f}\UTFM{26402}

簡体字:\UTFC{20087}\UTFC{200cc} 繁體字:\UTFT{20087}\UTFT{200cc}  多言語:\UTFM{20087}\UTFM{200cc}

\vspace{\baselineskip}

\ifuptexmode
 \kchar"20B9Fる。
 𠮟る。
\fi
\ajVar{叱}る。
叱る。

\ifuptexmode
 らいおん\kchar"20158。
 らいおん𠅘。
\fi
らいおん\ajVar{亭}。
らいおん亭。

\ifuptexmode
 \kchar"20BB7野家。
 𠮷野家。
\fi
\ajVar{吉}野家。
吉野家。

\ifuptexmode
 銀\kchar"2B776アスター。
 銀𫝶アスター。
\fi
銀\ajVar{座}アスター。
銀座アスター。

\makeatletter
\ifx\withhyperref\@undefined
\else

\section{見出しに\texorpdfstring{\bs}{\134}UTF, \texorpdfstring{\bs}{\134}UTFC, \texorpdfstring{\bs}{\134}UTFMなど}
\subsection{日本:\UTF{9aa8}\UTF{6D77} 簡体字:\UTFC{9aa8}\UTFC{6D77} 繁體字:\UTFT{9AA8}\UTFT{6d77} 朝鮮:\UTFK{9AA8}\UTFK{6d77}}
日本:\UTF{9aa8}\UTF{6D77} 簡体字:\UTFC{9aa8}\UTFC{6D77} 繁體字:\UTFT{9AA8}\UTFT{6d77} 朝鮮:\UTFK{9AA8}\UTFK{6d77}

\subsection{ハングル:\UTFK{c548}\UTFK{b155}\UTFK{d558}\UTFK{C138}\UTFK{C694}}
ハングル:\UTFK{c548}\UTFK{b155}\UTFK{d558}\UTFK{C138}\UTFK{C694}

\subsection{日本:\UTF{20509}\UTF{241FE} 簡体字:\UTFC{20509}\UTFC{241FE}  多言語:\UTFM{20509}\UTFM{241FE}}
日本:\UTF{20509}\UTF{241FE} 簡体字:\UTFC{20509}\UTFC{241FE}  多言語:\UTFM{20509}\UTFM{241FE}

\subsection{日本:\UTF{20509}\UTF{241FE} 簡体字:\UTFC{20509}\UTFC{241FE}  多言語:\UTFM{20509}\UTFM{241FE}}
日本:\UTF{20509}\UTF{241FE} 簡体字:\UTFC{20509}\UTFC{241FE}  多言語:\UTFM{20509}\UTFM{241FE}

\subsection{日本:\UTF{20b9f}\UTF{26402} 繁體字:\UTFT{20b9f}\UTFT{26402}  多言語:\UTFM{20b9f}\UTFM{26402}}
日本:\UTF{20b9f}\UTF{26402} 繁體字:\UTFT{20b9f}\UTFT{26402}  多言語:\UTFM{20b9f}\UTFM{26402}

\subsection{簡体字:\UTFC{20087}\UTFC{200cc} 繁體字:\UTFT{20087}\UTFT{200cc}  多言語:\UTFM{20087}\UTFM{200cc}}
簡体字:\UTFC{20087}\UTFC{200cc} 繁體字:\UTFT{20087}\UTFT{200cc}  多言語:\UTFM{20087}\UTFM{200cc}
\fi
\makeatother

\clearpage
[mc/m]

\ifuptexmode
 %
% This file is generated from the data of UniJIS-UTF32
% in cid2code.txt (Version 05/18/2022)
% for Adobe-Japan1-7
%
% Reference:
%   https://github.com/adobe-type-tools/cmap-resources/
%   Adobe-Japan1-7/cid2code.txt
%
% A newer CMap may be required for some code points.
%


Adobe-Japan1-0\\
𨳝櫛𥡴𨻶杓巽屠兔冕冤
𡨚𤏐爨🄀

Adobe-Japan1-4\\
🄐🄑🄒🄓🄔🄕🄖🄗🄘🄙
🄚🄛🄜🄝🄞🄟🄠🄡🄢🄣
🄤🄥🄦🄧🄨🄩🅐🅑🅒🅓
🅔🅕🅖🅗🅘🅙🅚🅛🅜🅝
🅞🅟🅠🅡🅢🅣🅤🅥🅦🅧
🅨🅩🄰🄱🄲🄳🄴🄵🄶🄷
🄸🄹🄺🄻🄼🄽🄾🄿🅀🅁
🅂🅃🅄🅅🅆🅇🅈🅉🈂🈷
🅰🅱🅲🅳🅴🅵🅶🅷🅸🅹
🅺🅻🅼🅽🅾🅿🆀🆁🆂🆃
🆄🆅🆆🆇🆈🆉🞜𛄲𛅕眞
𠤎𦥑𫟘沿芽槪割𦈢𠮷𩵋
卿𫞎憲𠩤浩𫝆𫝷滋𠮟勺
爵周将𭕄𠀋城𩙿真𠆢𮕩
𫝑成𧾷𣳾炭𥫗彫潮𡈽冬
𤴔姬𫞉諭輸𥙿𦚰𠘨𠂊𠦄
卉寃拔𦦙𣏌杞𪧦𫞽絣𮉸
𠔿𦉪𠂰𮛪𨦇𨸗𫠚𤋮桒𣲾
𠘑嶲你𣘺𣏾𢘉

Adobe-Japan1-5\\
𡌛𡑮𡢽𡚴𡸴𣇄𣗄𣜿𣝣𤟱
𥒎𥔎𥝱𥧄𥶡𦫿𦹀𧃴𧚄𨉷
𨏍𪆐𠂉𠂢𠂤𠈓𠌫𠎁𠍱𠏹
𠑊𠔉𠗖𠝏𠠇𠠺𠢹𠥼𠦝𠫓
𠬝𠵅𠷡𠺕𠹭𠹤𠽟𡈁𡉕𡉻
𡉴𡋤𡋗𡋽𡌶𡍄𡏄𡑭𡗗𦰩
𡙇𡜆𡝂𡧃𡱖𡴭𡵅𡵸𡵢𡶡
𡶜𡶒𡶷𡷠𡸳𡼞𡽶𡿺𢅻𢌞
𢎭𢛳𢡛𢢫𢦏𢪸𢭏𢭐𢭆𢰝
𢮦𢰤𢷡𣇃𣇵𣆶𣍲𣏓𣏒𣏐
𣏤𣏕𣏚𣏟𣑊𣑑𣑋𣑥𣓤𣕚
𣖔𣘹𣙇𣘸𣜜𣜌𣝤𣟿𣟧𣠤
𣠽𣪘𣱿𣴀𣵀𣷺𣷹𣷓𣽾𤂖
𤄃𤇆𤇾𤎼𤘩𤚥𤢖𤩍𤭖𤭯
𤰖𤸎𤸷𤹪𤺋𥁊𥁕𥄢𥆩𥇥
𥇍𥈞𥉌𥐮𥓙𥖧𥞩𥞴𥧔𥫤
𥫣𥫱𥮲𥱋𥱤𥸮𥹖𥹥𥹢𥻘
𥻂𥻨𥼣𥽜𥿠𥿔𦀌𥿻𦀗𦁠
𦃭𦉰𦊆𣴎𦐂𦙾𦜝𦣝𦣪𦥯
𦧝𦨞𦩘𦪌𦪷𦱳𦳝𦹥𦾔𦿸
𦿶𦿷𧄍𧄹𧏛𧏚𧏾𧐐𧑉𧘕
𧘔𧘱𧚓𧜎𧜣𧝒𧦅𧪄𧮳𧮾
𧯇𧲸𧶠𧸐𨂊𨂻𨊂𨋳𨐌𨑕
𨕫𨗈𨗉𨛗𨛺𨥉𨥆𨥫𨦈𨦺
𨦻𨨞𨨩𨩱𨩃𨪙𨫍𨫤𨫝𨯁
𨯯𨴐𨵱𨷻𨸟𨸶𨺉𨻫𨼲𨿸
𩊠𩊱𩒐𩗏𩛰𩜙𩝐𩣆𩩲𩷛
𩸕𩺊𩹉𩻄𩻩𩻛𩿎𩿗𪀯𪀚
𪃹𪂂𢈘𪎌𪐷𪗱𪘂𪚲𱍐𠃵
𤸄𤿲𧵳再善形慈栟軔𪊲
𠅘𠖱𠛬𫝓𠵘𫝚𫝜𥧌𫝶𢹂
𫝼𠟈𢿫𧦴𫞂𫞋𣟱𫞔𤁋𫞬
𫞯𫟉𫟏𫟒𦲞𧰼𫟰𫝥𫠍𫠗
𦍌𩸽𪘚

% end

\fi
%
% This file is generated from the data of UniJIS-UTF32
% in cid2code.txt (Version 05/18/2022)
% for Adobe-Japan1-7
%
% Reference:
%   https://github.com/adobe-type-tools/cmap-resources/
%   Adobe-Japan1-7/cid2code.txt
%
% A newer CMap may be required for some code points.
%


Adobe-Japan1-0\\
\UTF{28CDD}\UTF{2F8ED}\UTF{25874}\UTF{28EF6}\UTF{2F8DC}\UTF{2F884}\UTF{2F877}\UTF{2F80F}\UTF{2F8D3}\UTF{2F818}%
\UTF{21A1A}\UTF{243D0}\UTF{2F920}\UTF{1F100}

Adobe-Japan1-4\\
\UTF{1F110}\UTF{1F111}\UTF{1F112}\UTF{1F113}\UTF{1F114}\UTF{1F115}\UTF{1F116}\UTF{1F117}\UTF{1F118}\UTF{1F119}%
\UTF{1F11A}\UTF{1F11B}\UTF{1F11C}\UTF{1F11D}\UTF{1F11E}\UTF{1F11F}\UTF{1F120}\UTF{1F121}\UTF{1F122}\UTF{1F123}%
\UTF{1F124}\UTF{1F125}\UTF{1F126}\UTF{1F127}\UTF{1F128}\UTF{1F129}\UTF{1F150}\UTF{1F151}\UTF{1F152}\UTF{1F153}%
\UTF{1F154}\UTF{1F155}\UTF{1F156}\UTF{1F157}\UTF{1F158}\UTF{1F159}\UTF{1F15A}\UTF{1F15B}\UTF{1F15C}\UTF{1F15D}%
\UTF{1F15E}\UTF{1F15F}\UTF{1F160}\UTF{1F161}\UTF{1F162}\UTF{1F163}\UTF{1F164}\UTF{1F165}\UTF{1F166}\UTF{1F167}%
\UTF{1F168}\UTF{1F169}\UTF{1F130}\UTF{1F131}\UTF{1F132}\UTF{1F133}\UTF{1F134}\UTF{1F135}\UTF{1F136}\UTF{1F137}%
\UTF{1F138}\UTF{1F139}\UTF{1F13A}\UTF{1F13B}\UTF{1F13C}\UTF{1F13D}\UTF{1F13E}\UTF{1F13F}\UTF{1F140}\UTF{1F141}%
\UTF{1F142}\UTF{1F143}\UTF{1F144}\UTF{1F145}\UTF{1F146}\UTF{1F147}\UTF{1F148}\UTF{1F149}\UTF{1F202}\UTF{1F237}%
\UTF{1F170}\UTF{1F171}\UTF{1F172}\UTF{1F173}\UTF{1F174}\UTF{1F175}\UTF{1F176}\UTF{1F177}\UTF{1F178}\UTF{1F179}%
\UTF{1F17A}\UTF{1F17B}\UTF{1F17C}\UTF{1F17D}\UTF{1F17E}\UTF{1F17F}\UTF{1F180}\UTF{1F181}\UTF{1F182}\UTF{1F183}%
\UTF{1F184}\UTF{1F185}\UTF{1F186}\UTF{1F187}\UTF{1F188}\UTF{1F189}\UTF{1F79C}\UTF{1B132}\UTF{1B155}\UTF{2F945}%
\UTF{2090E}\UTF{26951}\UTF{2B7D8}\UTF{2F8FC}\UTF{2F995}\UTF{2F8EA}\UTF{2F822}\UTF{26222}\UTF{20BB7}\UTF{29D4B}%
\UTF{2F833}\UTF{2B78E}\UTF{2F8AC}\UTF{20A64}\UTF{2F903}\UTF{2B746}\UTF{2B777}\UTF{2F90B}\UTF{20B9F}\UTF{2F828}%
\UTF{2F921}\UTF{2F83F}\UTF{2F873}\UTF{2D544}\UTF{2000B}\UTF{2F852}\UTF{2967F}\UTF{2F947}\UTF{201A2}\UTF{2E569}%
\UTF{2B751}\UTF{2F8B2}\UTF{27FB7}\UTF{23CFE}\UTF{2F91A}\UTF{25AD7}\UTF{2F89A}\UTF{2F90F}\UTF{2123D}\UTF{2F81A}%
\UTF{24D14}\UTF{2F862}\UTF{2B789}\UTF{2F9D0}\UTF{2F9DF}\UTF{2567F}\UTF{266B0}\UTF{20628}\UTF{2008A}\UTF{20984}%
\UTF{2F82C}\UTF{2F86D}\UTF{2F8B6}\UTF{26999}\UTF{233CC}\UTF{2F8DB}\UTF{2A9E6}\UTF{2B7BD}\UTF{2F96C}\UTF{2E278}%
\UTF{2053F}\UTF{2626A}\UTF{200B0}\UTF{2E6EA}\UTF{28987}\UTF{28E17}\UTF{2B81A}\UTF{242EE}\UTF{2F8E1}\UTF{23CBE}%
\UTF{20611}\UTF{2F9F4}\UTF{2F804}\UTF{2363A}\UTF{233FE}\UTF{22609}

Adobe-Japan1-5\\
\UTF{2131B}\UTF{2146E}\UTF{218BD}\UTF{216B4}\UTF{21E34}\UTF{231C4}\UTF{235C4}\UTF{2373F}\UTF{23763}\UTF{247F1}%
\UTF{2548E}\UTF{2550E}\UTF{25771}\UTF{259C4}\UTF{25DA1}\UTF{26AFF}\UTF{26E40}\UTF{270F4}\UTF{27684}\UTF{28277}%
\UTF{283CD}\UTF{2A190}\UTF{20089}\UTF{200A2}\UTF{200A4}\UTF{20213}\UTF{2032B}\UTF{20381}\UTF{20371}\UTF{203F9}%
\UTF{2044A}\UTF{20509}\UTF{205D6}\UTF{2074F}\UTF{20807}\UTF{2083A}\UTF{208B9}\UTF{2097C}\UTF{2099D}\UTF{20AD3}%
\UTF{20B1D}\UTF{20D45}\UTF{20DE1}\UTF{20E95}\UTF{20E6D}\UTF{20E64}\UTF{20F5F}\UTF{21201}\UTF{21255}\UTF{2127B}%
\UTF{21274}\UTF{212E4}\UTF{212D7}\UTF{212FD}\UTF{21336}\UTF{21344}\UTF{213C4}\UTF{2146D}\UTF{215D7}\UTF{26C29}%
\UTF{21647}\UTF{21706}\UTF{21742}\UTF{219C3}\UTF{21C56}\UTF{21D2D}\UTF{21D45}\UTF{21D78}\UTF{21D62}\UTF{21DA1}%
\UTF{21D9C}\UTF{21D92}\UTF{21DB7}\UTF{21DE0}\UTF{21E33}\UTF{21F1E}\UTF{21F76}\UTF{21FFA}\UTF{2217B}\UTF{2231E}%
\UTF{223AD}\UTF{226F3}\UTF{2285B}\UTF{228AB}\UTF{2298F}\UTF{22AB8}\UTF{22B4F}\UTF{22B50}\UTF{22B46}\UTF{22C1D}%
\UTF{22BA6}\UTF{22C24}\UTF{22DE1}\UTF{231C3}\UTF{231F5}\UTF{231B6}\UTF{23372}\UTF{233D3}\UTF{233D2}\UTF{233D0}%
\UTF{233E4}\UTF{233D5}\UTF{233DA}\UTF{233DF}\UTF{2344A}\UTF{23451}\UTF{2344B}\UTF{23465}\UTF{234E4}\UTF{2355A}%
\UTF{23594}\UTF{23639}\UTF{23647}\UTF{23638}\UTF{2371C}\UTF{2370C}\UTF{23764}\UTF{237FF}\UTF{237E7}\UTF{23824}%
\UTF{2383D}\UTF{23A98}\UTF{23C7F}\UTF{23D00}\UTF{23D40}\UTF{23DFA}\UTF{23DF9}\UTF{23DD3}\UTF{23F7E}\UTF{24096}%
\UTF{24103}\UTF{241C6}\UTF{241FE}\UTF{243BC}\UTF{24629}\UTF{246A5}\UTF{24896}\UTF{24A4D}\UTF{24B56}\UTF{24B6F}%
\UTF{24C16}\UTF{24E0E}\UTF{24E37}\UTF{24E6A}\UTF{24E8B}\UTF{2504A}\UTF{25055}\UTF{25122}\UTF{251A9}\UTF{251E5}%
\UTF{251CD}\UTF{2521E}\UTF{2524C}\UTF{2542E}\UTF{254D9}\UTF{255A7}\UTF{257A9}\UTF{257B4}\UTF{259D4}\UTF{25AE4}%
\UTF{25AE3}\UTF{25AF1}\UTF{25BB2}\UTF{25C4B}\UTF{25C64}\UTF{25E2E}\UTF{25E56}\UTF{25E65}\UTF{25E62}\UTF{25ED8}%
\UTF{25EC2}\UTF{25EE8}\UTF{25F23}\UTF{25F5C}\UTF{25FE0}\UTF{25FD4}\UTF{2600C}\UTF{25FFB}\UTF{26017}\UTF{26060}%
\UTF{260ED}\UTF{26270}\UTF{26286}\UTF{23D0E}\UTF{26402}\UTF{2667E}\UTF{2671D}\UTF{268DD}\UTF{268EA}\UTF{2696F}%
\UTF{269DD}\UTF{26A1E}\UTF{26A58}\UTF{26A8C}\UTF{26AB7}\UTF{26C73}\UTF{26CDD}\UTF{26E65}\UTF{26F94}\UTF{26FF8}%
\UTF{26FF6}\UTF{26FF7}\UTF{2710D}\UTF{27139}\UTF{273DB}\UTF{273DA}\UTF{273FE}\UTF{27410}\UTF{27449}\UTF{27615}%
\UTF{27614}\UTF{27631}\UTF{27693}\UTF{2770E}\UTF{27723}\UTF{27752}\UTF{27985}\UTF{27A84}\UTF{27BB3}\UTF{27BBE}%
\UTF{27BC7}\UTF{27CB8}\UTF{27DA0}\UTF{27E10}\UTF{2808A}\UTF{280BB}\UTF{28282}\UTF{282F3}\UTF{2840C}\UTF{28455}%
\UTF{2856B}\UTF{285C8}\UTF{285C9}\UTF{286D7}\UTF{286FA}\UTF{28949}\UTF{28946}\UTF{2896B}\UTF{28988}\UTF{289BA}%
\UTF{289BB}\UTF{28A1E}\UTF{28A29}\UTF{28A71}\UTF{28A43}\UTF{28A99}\UTF{28ACD}\UTF{28AE4}\UTF{28ADD}\UTF{28BC1}%
\UTF{28BEF}\UTF{28D10}\UTF{28D71}\UTF{28DFB}\UTF{28E1F}\UTF{28E36}\UTF{28E89}\UTF{28EEB}\UTF{28F32}\UTF{28FF8}%
\UTF{292A0}\UTF{292B1}\UTF{29490}\UTF{295CF}\UTF{296F0}\UTF{29719}\UTF{29750}\UTF{298C6}\UTF{29A72}\UTF{29DDB}%
\UTF{29E15}\UTF{29E8A}\UTF{29E49}\UTF{29EC4}\UTF{29EE9}\UTF{29EDB}\UTF{29FCE}\UTF{29FD7}\UTF{2A02F}\UTF{2A01A}%
\UTF{2A0F9}\UTF{2A082}\UTF{22218}\UTF{2A38C}\UTF{2A437}\UTF{2A5F1}\UTF{2A602}\UTF{2A6B2}\UTF{31350}\UTF{200F5}%
\UTF{24E04}\UTF{24FF2}\UTF{27D73}\UTF{2F815}\UTF{2F846}\UTF{2F899}\UTF{2F8A6}\UTF{2F8E5}\UTF{2F9DE}\UTF{2A2B2}%
\UTF{20158}\UTF{205B1}\UTF{206EC}\UTF{2B753}\UTF{20D58}\UTF{2B75A}\UTF{2B75C}\UTF{259CC}\UTF{2B776}\UTF{22E42}%
\UTF{2B77C}\UTF{207C8}\UTF{22FEB}\UTF{279B4}\UTF{2B782}\UTF{2B78B}\UTF{237F1}\UTF{2B794}\UTF{2404B}\UTF{2B7AC}%
\UTF{2B7AF}\UTF{2B7C9}\UTF{2B7CF}\UTF{2B7D2}\UTF{26C9E}\UTF{27C3C}\UTF{2B7F0}\UTF{2B765}\UTF{2B80D}\UTF{2B817}%
\UTF{2634C}\UTF{29E3D}\UTF{2A61A}

% end


{\bfseries%
[mc/bx]

\ifuptexmode
 %
% This file is generated from the data of UniJIS-UTF32
% in cid2code.txt (Version 05/18/2022)
% for Adobe-Japan1-7
%
% Reference:
%   https://github.com/adobe-type-tools/cmap-resources/
%   Adobe-Japan1-7/cid2code.txt
%
% A newer CMap may be required for some code points.
%


Adobe-Japan1-0\\
𨳝櫛𥡴𨻶杓巽屠兔冕冤
𡨚𤏐爨🄀

Adobe-Japan1-4\\
🄐🄑🄒🄓🄔🄕🄖🄗🄘🄙
🄚🄛🄜🄝🄞🄟🄠🄡🄢🄣
🄤🄥🄦🄧🄨🄩🅐🅑🅒🅓
🅔🅕🅖🅗🅘🅙🅚🅛🅜🅝
🅞🅟🅠🅡🅢🅣🅤🅥🅦🅧
🅨🅩🄰🄱🄲🄳🄴🄵🄶🄷
🄸🄹🄺🄻🄼🄽🄾🄿🅀🅁
🅂🅃🅄🅅🅆🅇🅈🅉🈂🈷
🅰🅱🅲🅳🅴🅵🅶🅷🅸🅹
🅺🅻🅼🅽🅾🅿🆀🆁🆂🆃
🆄🆅🆆🆇🆈🆉🞜𛄲𛅕眞
𠤎𦥑𫟘沿芽槪割𦈢𠮷𩵋
卿𫞎憲𠩤浩𫝆𫝷滋𠮟勺
爵周将𭕄𠀋城𩙿真𠆢𮕩
𫝑成𧾷𣳾炭𥫗彫潮𡈽冬
𤴔姬𫞉諭輸𥙿𦚰𠘨𠂊𠦄
卉寃拔𦦙𣏌杞𪧦𫞽絣𮉸
𠔿𦉪𠂰𮛪𨦇𨸗𫠚𤋮桒𣲾
𠘑嶲你𣘺𣏾𢘉

Adobe-Japan1-5\\
𡌛𡑮𡢽𡚴𡸴𣇄𣗄𣜿𣝣𤟱
𥒎𥔎𥝱𥧄𥶡𦫿𦹀𧃴𧚄𨉷
𨏍𪆐𠂉𠂢𠂤𠈓𠌫𠎁𠍱𠏹
𠑊𠔉𠗖𠝏𠠇𠠺𠢹𠥼𠦝𠫓
𠬝𠵅𠷡𠺕𠹭𠹤𠽟𡈁𡉕𡉻
𡉴𡋤𡋗𡋽𡌶𡍄𡏄𡑭𡗗𦰩
𡙇𡜆𡝂𡧃𡱖𡴭𡵅𡵸𡵢𡶡
𡶜𡶒𡶷𡷠𡸳𡼞𡽶𡿺𢅻𢌞
𢎭𢛳𢡛𢢫𢦏𢪸𢭏𢭐𢭆𢰝
𢮦𢰤𢷡𣇃𣇵𣆶𣍲𣏓𣏒𣏐
𣏤𣏕𣏚𣏟𣑊𣑑𣑋𣑥𣓤𣕚
𣖔𣘹𣙇𣘸𣜜𣜌𣝤𣟿𣟧𣠤
𣠽𣪘𣱿𣴀𣵀𣷺𣷹𣷓𣽾𤂖
𤄃𤇆𤇾𤎼𤘩𤚥𤢖𤩍𤭖𤭯
𤰖𤸎𤸷𤹪𤺋𥁊𥁕𥄢𥆩𥇥
𥇍𥈞𥉌𥐮𥓙𥖧𥞩𥞴𥧔𥫤
𥫣𥫱𥮲𥱋𥱤𥸮𥹖𥹥𥹢𥻘
𥻂𥻨𥼣𥽜𥿠𥿔𦀌𥿻𦀗𦁠
𦃭𦉰𦊆𣴎𦐂𦙾𦜝𦣝𦣪𦥯
𦧝𦨞𦩘𦪌𦪷𦱳𦳝𦹥𦾔𦿸
𦿶𦿷𧄍𧄹𧏛𧏚𧏾𧐐𧑉𧘕
𧘔𧘱𧚓𧜎𧜣𧝒𧦅𧪄𧮳𧮾
𧯇𧲸𧶠𧸐𨂊𨂻𨊂𨋳𨐌𨑕
𨕫𨗈𨗉𨛗𨛺𨥉𨥆𨥫𨦈𨦺
𨦻𨨞𨨩𨩱𨩃𨪙𨫍𨫤𨫝𨯁
𨯯𨴐𨵱𨷻𨸟𨸶𨺉𨻫𨼲𨿸
𩊠𩊱𩒐𩗏𩛰𩜙𩝐𩣆𩩲𩷛
𩸕𩺊𩹉𩻄𩻩𩻛𩿎𩿗𪀯𪀚
𪃹𪂂𢈘𪎌𪐷𪗱𪘂𪚲𱍐𠃵
𤸄𤿲𧵳再善形慈栟軔𪊲
𠅘𠖱𠛬𫝓𠵘𫝚𫝜𥧌𫝶𢹂
𫝼𠟈𢿫𧦴𫞂𫞋𣟱𫞔𤁋𫞬
𫞯𫟉𫟏𫟒𦲞𧰼𫟰𫝥𫠍𫠗
𦍌𩸽𪘚

% end

\fi
%
% This file is generated from the data of UniJIS-UTF32
% in cid2code.txt (Version 05/18/2022)
% for Adobe-Japan1-7
%
% Reference:
%   https://github.com/adobe-type-tools/cmap-resources/
%   Adobe-Japan1-7/cid2code.txt
%
% A newer CMap may be required for some code points.
%


Adobe-Japan1-0\\
\UTF{28CDD}\UTF{2F8ED}\UTF{25874}\UTF{28EF6}\UTF{2F8DC}\UTF{2F884}\UTF{2F877}\UTF{2F80F}\UTF{2F8D3}\UTF{2F818}%
\UTF{21A1A}\UTF{243D0}\UTF{2F920}\UTF{1F100}

Adobe-Japan1-4\\
\UTF{1F110}\UTF{1F111}\UTF{1F112}\UTF{1F113}\UTF{1F114}\UTF{1F115}\UTF{1F116}\UTF{1F117}\UTF{1F118}\UTF{1F119}%
\UTF{1F11A}\UTF{1F11B}\UTF{1F11C}\UTF{1F11D}\UTF{1F11E}\UTF{1F11F}\UTF{1F120}\UTF{1F121}\UTF{1F122}\UTF{1F123}%
\UTF{1F124}\UTF{1F125}\UTF{1F126}\UTF{1F127}\UTF{1F128}\UTF{1F129}\UTF{1F150}\UTF{1F151}\UTF{1F152}\UTF{1F153}%
\UTF{1F154}\UTF{1F155}\UTF{1F156}\UTF{1F157}\UTF{1F158}\UTF{1F159}\UTF{1F15A}\UTF{1F15B}\UTF{1F15C}\UTF{1F15D}%
\UTF{1F15E}\UTF{1F15F}\UTF{1F160}\UTF{1F161}\UTF{1F162}\UTF{1F163}\UTF{1F164}\UTF{1F165}\UTF{1F166}\UTF{1F167}%
\UTF{1F168}\UTF{1F169}\UTF{1F130}\UTF{1F131}\UTF{1F132}\UTF{1F133}\UTF{1F134}\UTF{1F135}\UTF{1F136}\UTF{1F137}%
\UTF{1F138}\UTF{1F139}\UTF{1F13A}\UTF{1F13B}\UTF{1F13C}\UTF{1F13D}\UTF{1F13E}\UTF{1F13F}\UTF{1F140}\UTF{1F141}%
\UTF{1F142}\UTF{1F143}\UTF{1F144}\UTF{1F145}\UTF{1F146}\UTF{1F147}\UTF{1F148}\UTF{1F149}\UTF{1F202}\UTF{1F237}%
\UTF{1F170}\UTF{1F171}\UTF{1F172}\UTF{1F173}\UTF{1F174}\UTF{1F175}\UTF{1F176}\UTF{1F177}\UTF{1F178}\UTF{1F179}%
\UTF{1F17A}\UTF{1F17B}\UTF{1F17C}\UTF{1F17D}\UTF{1F17E}\UTF{1F17F}\UTF{1F180}\UTF{1F181}\UTF{1F182}\UTF{1F183}%
\UTF{1F184}\UTF{1F185}\UTF{1F186}\UTF{1F187}\UTF{1F188}\UTF{1F189}\UTF{1F79C}\UTF{1B132}\UTF{1B155}\UTF{2F945}%
\UTF{2090E}\UTF{26951}\UTF{2B7D8}\UTF{2F8FC}\UTF{2F995}\UTF{2F8EA}\UTF{2F822}\UTF{26222}\UTF{20BB7}\UTF{29D4B}%
\UTF{2F833}\UTF{2B78E}\UTF{2F8AC}\UTF{20A64}\UTF{2F903}\UTF{2B746}\UTF{2B777}\UTF{2F90B}\UTF{20B9F}\UTF{2F828}%
\UTF{2F921}\UTF{2F83F}\UTF{2F873}\UTF{2D544}\UTF{2000B}\UTF{2F852}\UTF{2967F}\UTF{2F947}\UTF{201A2}\UTF{2E569}%
\UTF{2B751}\UTF{2F8B2}\UTF{27FB7}\UTF{23CFE}\UTF{2F91A}\UTF{25AD7}\UTF{2F89A}\UTF{2F90F}\UTF{2123D}\UTF{2F81A}%
\UTF{24D14}\UTF{2F862}\UTF{2B789}\UTF{2F9D0}\UTF{2F9DF}\UTF{2567F}\UTF{266B0}\UTF{20628}\UTF{2008A}\UTF{20984}%
\UTF{2F82C}\UTF{2F86D}\UTF{2F8B6}\UTF{26999}\UTF{233CC}\UTF{2F8DB}\UTF{2A9E6}\UTF{2B7BD}\UTF{2F96C}\UTF{2E278}%
\UTF{2053F}\UTF{2626A}\UTF{200B0}\UTF{2E6EA}\UTF{28987}\UTF{28E17}\UTF{2B81A}\UTF{242EE}\UTF{2F8E1}\UTF{23CBE}%
\UTF{20611}\UTF{2F9F4}\UTF{2F804}\UTF{2363A}\UTF{233FE}\UTF{22609}

Adobe-Japan1-5\\
\UTF{2131B}\UTF{2146E}\UTF{218BD}\UTF{216B4}\UTF{21E34}\UTF{231C4}\UTF{235C4}\UTF{2373F}\UTF{23763}\UTF{247F1}%
\UTF{2548E}\UTF{2550E}\UTF{25771}\UTF{259C4}\UTF{25DA1}\UTF{26AFF}\UTF{26E40}\UTF{270F4}\UTF{27684}\UTF{28277}%
\UTF{283CD}\UTF{2A190}\UTF{20089}\UTF{200A2}\UTF{200A4}\UTF{20213}\UTF{2032B}\UTF{20381}\UTF{20371}\UTF{203F9}%
\UTF{2044A}\UTF{20509}\UTF{205D6}\UTF{2074F}\UTF{20807}\UTF{2083A}\UTF{208B9}\UTF{2097C}\UTF{2099D}\UTF{20AD3}%
\UTF{20B1D}\UTF{20D45}\UTF{20DE1}\UTF{20E95}\UTF{20E6D}\UTF{20E64}\UTF{20F5F}\UTF{21201}\UTF{21255}\UTF{2127B}%
\UTF{21274}\UTF{212E4}\UTF{212D7}\UTF{212FD}\UTF{21336}\UTF{21344}\UTF{213C4}\UTF{2146D}\UTF{215D7}\UTF{26C29}%
\UTF{21647}\UTF{21706}\UTF{21742}\UTF{219C3}\UTF{21C56}\UTF{21D2D}\UTF{21D45}\UTF{21D78}\UTF{21D62}\UTF{21DA1}%
\UTF{21D9C}\UTF{21D92}\UTF{21DB7}\UTF{21DE0}\UTF{21E33}\UTF{21F1E}\UTF{21F76}\UTF{21FFA}\UTF{2217B}\UTF{2231E}%
\UTF{223AD}\UTF{226F3}\UTF{2285B}\UTF{228AB}\UTF{2298F}\UTF{22AB8}\UTF{22B4F}\UTF{22B50}\UTF{22B46}\UTF{22C1D}%
\UTF{22BA6}\UTF{22C24}\UTF{22DE1}\UTF{231C3}\UTF{231F5}\UTF{231B6}\UTF{23372}\UTF{233D3}\UTF{233D2}\UTF{233D0}%
\UTF{233E4}\UTF{233D5}\UTF{233DA}\UTF{233DF}\UTF{2344A}\UTF{23451}\UTF{2344B}\UTF{23465}\UTF{234E4}\UTF{2355A}%
\UTF{23594}\UTF{23639}\UTF{23647}\UTF{23638}\UTF{2371C}\UTF{2370C}\UTF{23764}\UTF{237FF}\UTF{237E7}\UTF{23824}%
\UTF{2383D}\UTF{23A98}\UTF{23C7F}\UTF{23D00}\UTF{23D40}\UTF{23DFA}\UTF{23DF9}\UTF{23DD3}\UTF{23F7E}\UTF{24096}%
\UTF{24103}\UTF{241C6}\UTF{241FE}\UTF{243BC}\UTF{24629}\UTF{246A5}\UTF{24896}\UTF{24A4D}\UTF{24B56}\UTF{24B6F}%
\UTF{24C16}\UTF{24E0E}\UTF{24E37}\UTF{24E6A}\UTF{24E8B}\UTF{2504A}\UTF{25055}\UTF{25122}\UTF{251A9}\UTF{251E5}%
\UTF{251CD}\UTF{2521E}\UTF{2524C}\UTF{2542E}\UTF{254D9}\UTF{255A7}\UTF{257A9}\UTF{257B4}\UTF{259D4}\UTF{25AE4}%
\UTF{25AE3}\UTF{25AF1}\UTF{25BB2}\UTF{25C4B}\UTF{25C64}\UTF{25E2E}\UTF{25E56}\UTF{25E65}\UTF{25E62}\UTF{25ED8}%
\UTF{25EC2}\UTF{25EE8}\UTF{25F23}\UTF{25F5C}\UTF{25FE0}\UTF{25FD4}\UTF{2600C}\UTF{25FFB}\UTF{26017}\UTF{26060}%
\UTF{260ED}\UTF{26270}\UTF{26286}\UTF{23D0E}\UTF{26402}\UTF{2667E}\UTF{2671D}\UTF{268DD}\UTF{268EA}\UTF{2696F}%
\UTF{269DD}\UTF{26A1E}\UTF{26A58}\UTF{26A8C}\UTF{26AB7}\UTF{26C73}\UTF{26CDD}\UTF{26E65}\UTF{26F94}\UTF{26FF8}%
\UTF{26FF6}\UTF{26FF7}\UTF{2710D}\UTF{27139}\UTF{273DB}\UTF{273DA}\UTF{273FE}\UTF{27410}\UTF{27449}\UTF{27615}%
\UTF{27614}\UTF{27631}\UTF{27693}\UTF{2770E}\UTF{27723}\UTF{27752}\UTF{27985}\UTF{27A84}\UTF{27BB3}\UTF{27BBE}%
\UTF{27BC7}\UTF{27CB8}\UTF{27DA0}\UTF{27E10}\UTF{2808A}\UTF{280BB}\UTF{28282}\UTF{282F3}\UTF{2840C}\UTF{28455}%
\UTF{2856B}\UTF{285C8}\UTF{285C9}\UTF{286D7}\UTF{286FA}\UTF{28949}\UTF{28946}\UTF{2896B}\UTF{28988}\UTF{289BA}%
\UTF{289BB}\UTF{28A1E}\UTF{28A29}\UTF{28A71}\UTF{28A43}\UTF{28A99}\UTF{28ACD}\UTF{28AE4}\UTF{28ADD}\UTF{28BC1}%
\UTF{28BEF}\UTF{28D10}\UTF{28D71}\UTF{28DFB}\UTF{28E1F}\UTF{28E36}\UTF{28E89}\UTF{28EEB}\UTF{28F32}\UTF{28FF8}%
\UTF{292A0}\UTF{292B1}\UTF{29490}\UTF{295CF}\UTF{296F0}\UTF{29719}\UTF{29750}\UTF{298C6}\UTF{29A72}\UTF{29DDB}%
\UTF{29E15}\UTF{29E8A}\UTF{29E49}\UTF{29EC4}\UTF{29EE9}\UTF{29EDB}\UTF{29FCE}\UTF{29FD7}\UTF{2A02F}\UTF{2A01A}%
\UTF{2A0F9}\UTF{2A082}\UTF{22218}\UTF{2A38C}\UTF{2A437}\UTF{2A5F1}\UTF{2A602}\UTF{2A6B2}\UTF{31350}\UTF{200F5}%
\UTF{24E04}\UTF{24FF2}\UTF{27D73}\UTF{2F815}\UTF{2F846}\UTF{2F899}\UTF{2F8A6}\UTF{2F8E5}\UTF{2F9DE}\UTF{2A2B2}%
\UTF{20158}\UTF{205B1}\UTF{206EC}\UTF{2B753}\UTF{20D58}\UTF{2B75A}\UTF{2B75C}\UTF{259CC}\UTF{2B776}\UTF{22E42}%
\UTF{2B77C}\UTF{207C8}\UTF{22FEB}\UTF{279B4}\UTF{2B782}\UTF{2B78B}\UTF{237F1}\UTF{2B794}\UTF{2404B}\UTF{2B7AC}%
\UTF{2B7AF}\UTF{2B7C9}\UTF{2B7CF}\UTF{2B7D2}\UTF{26C9E}\UTF{27C3C}\UTF{2B7F0}\UTF{2B765}\UTF{2B80D}\UTF{2B817}%
\UTF{2634C}\UTF{29E3D}\UTF{2A61A}

% end


}

{\gtfamily
[gt/m]

\ifuptexmode
 %
% This file is generated from the data of UniJIS-UTF32
% in cid2code.txt (Version 05/18/2022)
% for Adobe-Japan1-7
%
% Reference:
%   https://github.com/adobe-type-tools/cmap-resources/
%   Adobe-Japan1-7/cid2code.txt
%
% A newer CMap may be required for some code points.
%


Adobe-Japan1-0\\
𨳝櫛𥡴𨻶杓巽屠兔冕冤
𡨚𤏐爨🄀

Adobe-Japan1-4\\
🄐🄑🄒🄓🄔🄕🄖🄗🄘🄙
🄚🄛🄜🄝🄞🄟🄠🄡🄢🄣
🄤🄥🄦🄧🄨🄩🅐🅑🅒🅓
🅔🅕🅖🅗🅘🅙🅚🅛🅜🅝
🅞🅟🅠🅡🅢🅣🅤🅥🅦🅧
🅨🅩🄰🄱🄲🄳🄴🄵🄶🄷
🄸🄹🄺🄻🄼🄽🄾🄿🅀🅁
🅂🅃🅄🅅🅆🅇🅈🅉🈂🈷
🅰🅱🅲🅳🅴🅵🅶🅷🅸🅹
🅺🅻🅼🅽🅾🅿🆀🆁🆂🆃
🆄🆅🆆🆇🆈🆉🞜𛄲𛅕眞
𠤎𦥑𫟘沿芽槪割𦈢𠮷𩵋
卿𫞎憲𠩤浩𫝆𫝷滋𠮟勺
爵周将𭕄𠀋城𩙿真𠆢𮕩
𫝑成𧾷𣳾炭𥫗彫潮𡈽冬
𤴔姬𫞉諭輸𥙿𦚰𠘨𠂊𠦄
卉寃拔𦦙𣏌杞𪧦𫞽絣𮉸
𠔿𦉪𠂰𮛪𨦇𨸗𫠚𤋮桒𣲾
𠘑嶲你𣘺𣏾𢘉

Adobe-Japan1-5\\
𡌛𡑮𡢽𡚴𡸴𣇄𣗄𣜿𣝣𤟱
𥒎𥔎𥝱𥧄𥶡𦫿𦹀𧃴𧚄𨉷
𨏍𪆐𠂉𠂢𠂤𠈓𠌫𠎁𠍱𠏹
𠑊𠔉𠗖𠝏𠠇𠠺𠢹𠥼𠦝𠫓
𠬝𠵅𠷡𠺕𠹭𠹤𠽟𡈁𡉕𡉻
𡉴𡋤𡋗𡋽𡌶𡍄𡏄𡑭𡗗𦰩
𡙇𡜆𡝂𡧃𡱖𡴭𡵅𡵸𡵢𡶡
𡶜𡶒𡶷𡷠𡸳𡼞𡽶𡿺𢅻𢌞
𢎭𢛳𢡛𢢫𢦏𢪸𢭏𢭐𢭆𢰝
𢮦𢰤𢷡𣇃𣇵𣆶𣍲𣏓𣏒𣏐
𣏤𣏕𣏚𣏟𣑊𣑑𣑋𣑥𣓤𣕚
𣖔𣘹𣙇𣘸𣜜𣜌𣝤𣟿𣟧𣠤
𣠽𣪘𣱿𣴀𣵀𣷺𣷹𣷓𣽾𤂖
𤄃𤇆𤇾𤎼𤘩𤚥𤢖𤩍𤭖𤭯
𤰖𤸎𤸷𤹪𤺋𥁊𥁕𥄢𥆩𥇥
𥇍𥈞𥉌𥐮𥓙𥖧𥞩𥞴𥧔𥫤
𥫣𥫱𥮲𥱋𥱤𥸮𥹖𥹥𥹢𥻘
𥻂𥻨𥼣𥽜𥿠𥿔𦀌𥿻𦀗𦁠
𦃭𦉰𦊆𣴎𦐂𦙾𦜝𦣝𦣪𦥯
𦧝𦨞𦩘𦪌𦪷𦱳𦳝𦹥𦾔𦿸
𦿶𦿷𧄍𧄹𧏛𧏚𧏾𧐐𧑉𧘕
𧘔𧘱𧚓𧜎𧜣𧝒𧦅𧪄𧮳𧮾
𧯇𧲸𧶠𧸐𨂊𨂻𨊂𨋳𨐌𨑕
𨕫𨗈𨗉𨛗𨛺𨥉𨥆𨥫𨦈𨦺
𨦻𨨞𨨩𨩱𨩃𨪙𨫍𨫤𨫝𨯁
𨯯𨴐𨵱𨷻𨸟𨸶𨺉𨻫𨼲𨿸
𩊠𩊱𩒐𩗏𩛰𩜙𩝐𩣆𩩲𩷛
𩸕𩺊𩹉𩻄𩻩𩻛𩿎𩿗𪀯𪀚
𪃹𪂂𢈘𪎌𪐷𪗱𪘂𪚲𱍐𠃵
𤸄𤿲𧵳再善形慈栟軔𪊲
𠅘𠖱𠛬𫝓𠵘𫝚𫝜𥧌𫝶𢹂
𫝼𠟈𢿫𧦴𫞂𫞋𣟱𫞔𤁋𫞬
𫞯𫟉𫟏𫟒𦲞𧰼𫟰𫝥𫠍𫠗
𦍌𩸽𪘚

% end

\fi
%
% This file is generated from the data of UniJIS-UTF32
% in cid2code.txt (Version 05/18/2022)
% for Adobe-Japan1-7
%
% Reference:
%   https://github.com/adobe-type-tools/cmap-resources/
%   Adobe-Japan1-7/cid2code.txt
%
% A newer CMap may be required for some code points.
%


Adobe-Japan1-0\\
\UTF{28CDD}\UTF{2F8ED}\UTF{25874}\UTF{28EF6}\UTF{2F8DC}\UTF{2F884}\UTF{2F877}\UTF{2F80F}\UTF{2F8D3}\UTF{2F818}%
\UTF{21A1A}\UTF{243D0}\UTF{2F920}\UTF{1F100}

Adobe-Japan1-4\\
\UTF{1F110}\UTF{1F111}\UTF{1F112}\UTF{1F113}\UTF{1F114}\UTF{1F115}\UTF{1F116}\UTF{1F117}\UTF{1F118}\UTF{1F119}%
\UTF{1F11A}\UTF{1F11B}\UTF{1F11C}\UTF{1F11D}\UTF{1F11E}\UTF{1F11F}\UTF{1F120}\UTF{1F121}\UTF{1F122}\UTF{1F123}%
\UTF{1F124}\UTF{1F125}\UTF{1F126}\UTF{1F127}\UTF{1F128}\UTF{1F129}\UTF{1F150}\UTF{1F151}\UTF{1F152}\UTF{1F153}%
\UTF{1F154}\UTF{1F155}\UTF{1F156}\UTF{1F157}\UTF{1F158}\UTF{1F159}\UTF{1F15A}\UTF{1F15B}\UTF{1F15C}\UTF{1F15D}%
\UTF{1F15E}\UTF{1F15F}\UTF{1F160}\UTF{1F161}\UTF{1F162}\UTF{1F163}\UTF{1F164}\UTF{1F165}\UTF{1F166}\UTF{1F167}%
\UTF{1F168}\UTF{1F169}\UTF{1F130}\UTF{1F131}\UTF{1F132}\UTF{1F133}\UTF{1F134}\UTF{1F135}\UTF{1F136}\UTF{1F137}%
\UTF{1F138}\UTF{1F139}\UTF{1F13A}\UTF{1F13B}\UTF{1F13C}\UTF{1F13D}\UTF{1F13E}\UTF{1F13F}\UTF{1F140}\UTF{1F141}%
\UTF{1F142}\UTF{1F143}\UTF{1F144}\UTF{1F145}\UTF{1F146}\UTF{1F147}\UTF{1F148}\UTF{1F149}\UTF{1F202}\UTF{1F237}%
\UTF{1F170}\UTF{1F171}\UTF{1F172}\UTF{1F173}\UTF{1F174}\UTF{1F175}\UTF{1F176}\UTF{1F177}\UTF{1F178}\UTF{1F179}%
\UTF{1F17A}\UTF{1F17B}\UTF{1F17C}\UTF{1F17D}\UTF{1F17E}\UTF{1F17F}\UTF{1F180}\UTF{1F181}\UTF{1F182}\UTF{1F183}%
\UTF{1F184}\UTF{1F185}\UTF{1F186}\UTF{1F187}\UTF{1F188}\UTF{1F189}\UTF{1F79C}\UTF{1B132}\UTF{1B155}\UTF{2F945}%
\UTF{2090E}\UTF{26951}\UTF{2B7D8}\UTF{2F8FC}\UTF{2F995}\UTF{2F8EA}\UTF{2F822}\UTF{26222}\UTF{20BB7}\UTF{29D4B}%
\UTF{2F833}\UTF{2B78E}\UTF{2F8AC}\UTF{20A64}\UTF{2F903}\UTF{2B746}\UTF{2B777}\UTF{2F90B}\UTF{20B9F}\UTF{2F828}%
\UTF{2F921}\UTF{2F83F}\UTF{2F873}\UTF{2D544}\UTF{2000B}\UTF{2F852}\UTF{2967F}\UTF{2F947}\UTF{201A2}\UTF{2E569}%
\UTF{2B751}\UTF{2F8B2}\UTF{27FB7}\UTF{23CFE}\UTF{2F91A}\UTF{25AD7}\UTF{2F89A}\UTF{2F90F}\UTF{2123D}\UTF{2F81A}%
\UTF{24D14}\UTF{2F862}\UTF{2B789}\UTF{2F9D0}\UTF{2F9DF}\UTF{2567F}\UTF{266B0}\UTF{20628}\UTF{2008A}\UTF{20984}%
\UTF{2F82C}\UTF{2F86D}\UTF{2F8B6}\UTF{26999}\UTF{233CC}\UTF{2F8DB}\UTF{2A9E6}\UTF{2B7BD}\UTF{2F96C}\UTF{2E278}%
\UTF{2053F}\UTF{2626A}\UTF{200B0}\UTF{2E6EA}\UTF{28987}\UTF{28E17}\UTF{2B81A}\UTF{242EE}\UTF{2F8E1}\UTF{23CBE}%
\UTF{20611}\UTF{2F9F4}\UTF{2F804}\UTF{2363A}\UTF{233FE}\UTF{22609}

Adobe-Japan1-5\\
\UTF{2131B}\UTF{2146E}\UTF{218BD}\UTF{216B4}\UTF{21E34}\UTF{231C4}\UTF{235C4}\UTF{2373F}\UTF{23763}\UTF{247F1}%
\UTF{2548E}\UTF{2550E}\UTF{25771}\UTF{259C4}\UTF{25DA1}\UTF{26AFF}\UTF{26E40}\UTF{270F4}\UTF{27684}\UTF{28277}%
\UTF{283CD}\UTF{2A190}\UTF{20089}\UTF{200A2}\UTF{200A4}\UTF{20213}\UTF{2032B}\UTF{20381}\UTF{20371}\UTF{203F9}%
\UTF{2044A}\UTF{20509}\UTF{205D6}\UTF{2074F}\UTF{20807}\UTF{2083A}\UTF{208B9}\UTF{2097C}\UTF{2099D}\UTF{20AD3}%
\UTF{20B1D}\UTF{20D45}\UTF{20DE1}\UTF{20E95}\UTF{20E6D}\UTF{20E64}\UTF{20F5F}\UTF{21201}\UTF{21255}\UTF{2127B}%
\UTF{21274}\UTF{212E4}\UTF{212D7}\UTF{212FD}\UTF{21336}\UTF{21344}\UTF{213C4}\UTF{2146D}\UTF{215D7}\UTF{26C29}%
\UTF{21647}\UTF{21706}\UTF{21742}\UTF{219C3}\UTF{21C56}\UTF{21D2D}\UTF{21D45}\UTF{21D78}\UTF{21D62}\UTF{21DA1}%
\UTF{21D9C}\UTF{21D92}\UTF{21DB7}\UTF{21DE0}\UTF{21E33}\UTF{21F1E}\UTF{21F76}\UTF{21FFA}\UTF{2217B}\UTF{2231E}%
\UTF{223AD}\UTF{226F3}\UTF{2285B}\UTF{228AB}\UTF{2298F}\UTF{22AB8}\UTF{22B4F}\UTF{22B50}\UTF{22B46}\UTF{22C1D}%
\UTF{22BA6}\UTF{22C24}\UTF{22DE1}\UTF{231C3}\UTF{231F5}\UTF{231B6}\UTF{23372}\UTF{233D3}\UTF{233D2}\UTF{233D0}%
\UTF{233E4}\UTF{233D5}\UTF{233DA}\UTF{233DF}\UTF{2344A}\UTF{23451}\UTF{2344B}\UTF{23465}\UTF{234E4}\UTF{2355A}%
\UTF{23594}\UTF{23639}\UTF{23647}\UTF{23638}\UTF{2371C}\UTF{2370C}\UTF{23764}\UTF{237FF}\UTF{237E7}\UTF{23824}%
\UTF{2383D}\UTF{23A98}\UTF{23C7F}\UTF{23D00}\UTF{23D40}\UTF{23DFA}\UTF{23DF9}\UTF{23DD3}\UTF{23F7E}\UTF{24096}%
\UTF{24103}\UTF{241C6}\UTF{241FE}\UTF{243BC}\UTF{24629}\UTF{246A5}\UTF{24896}\UTF{24A4D}\UTF{24B56}\UTF{24B6F}%
\UTF{24C16}\UTF{24E0E}\UTF{24E37}\UTF{24E6A}\UTF{24E8B}\UTF{2504A}\UTF{25055}\UTF{25122}\UTF{251A9}\UTF{251E5}%
\UTF{251CD}\UTF{2521E}\UTF{2524C}\UTF{2542E}\UTF{254D9}\UTF{255A7}\UTF{257A9}\UTF{257B4}\UTF{259D4}\UTF{25AE4}%
\UTF{25AE3}\UTF{25AF1}\UTF{25BB2}\UTF{25C4B}\UTF{25C64}\UTF{25E2E}\UTF{25E56}\UTF{25E65}\UTF{25E62}\UTF{25ED8}%
\UTF{25EC2}\UTF{25EE8}\UTF{25F23}\UTF{25F5C}\UTF{25FE0}\UTF{25FD4}\UTF{2600C}\UTF{25FFB}\UTF{26017}\UTF{26060}%
\UTF{260ED}\UTF{26270}\UTF{26286}\UTF{23D0E}\UTF{26402}\UTF{2667E}\UTF{2671D}\UTF{268DD}\UTF{268EA}\UTF{2696F}%
\UTF{269DD}\UTF{26A1E}\UTF{26A58}\UTF{26A8C}\UTF{26AB7}\UTF{26C73}\UTF{26CDD}\UTF{26E65}\UTF{26F94}\UTF{26FF8}%
\UTF{26FF6}\UTF{26FF7}\UTF{2710D}\UTF{27139}\UTF{273DB}\UTF{273DA}\UTF{273FE}\UTF{27410}\UTF{27449}\UTF{27615}%
\UTF{27614}\UTF{27631}\UTF{27693}\UTF{2770E}\UTF{27723}\UTF{27752}\UTF{27985}\UTF{27A84}\UTF{27BB3}\UTF{27BBE}%
\UTF{27BC7}\UTF{27CB8}\UTF{27DA0}\UTF{27E10}\UTF{2808A}\UTF{280BB}\UTF{28282}\UTF{282F3}\UTF{2840C}\UTF{28455}%
\UTF{2856B}\UTF{285C8}\UTF{285C9}\UTF{286D7}\UTF{286FA}\UTF{28949}\UTF{28946}\UTF{2896B}\UTF{28988}\UTF{289BA}%
\UTF{289BB}\UTF{28A1E}\UTF{28A29}\UTF{28A71}\UTF{28A43}\UTF{28A99}\UTF{28ACD}\UTF{28AE4}\UTF{28ADD}\UTF{28BC1}%
\UTF{28BEF}\UTF{28D10}\UTF{28D71}\UTF{28DFB}\UTF{28E1F}\UTF{28E36}\UTF{28E89}\UTF{28EEB}\UTF{28F32}\UTF{28FF8}%
\UTF{292A0}\UTF{292B1}\UTF{29490}\UTF{295CF}\UTF{296F0}\UTF{29719}\UTF{29750}\UTF{298C6}\UTF{29A72}\UTF{29DDB}%
\UTF{29E15}\UTF{29E8A}\UTF{29E49}\UTF{29EC4}\UTF{29EE9}\UTF{29EDB}\UTF{29FCE}\UTF{29FD7}\UTF{2A02F}\UTF{2A01A}%
\UTF{2A0F9}\UTF{2A082}\UTF{22218}\UTF{2A38C}\UTF{2A437}\UTF{2A5F1}\UTF{2A602}\UTF{2A6B2}\UTF{31350}\UTF{200F5}%
\UTF{24E04}\UTF{24FF2}\UTF{27D73}\UTF{2F815}\UTF{2F846}\UTF{2F899}\UTF{2F8A6}\UTF{2F8E5}\UTF{2F9DE}\UTF{2A2B2}%
\UTF{20158}\UTF{205B1}\UTF{206EC}\UTF{2B753}\UTF{20D58}\UTF{2B75A}\UTF{2B75C}\UTF{259CC}\UTF{2B776}\UTF{22E42}%
\UTF{2B77C}\UTF{207C8}\UTF{22FEB}\UTF{279B4}\UTF{2B782}\UTF{2B78B}\UTF{237F1}\UTF{2B794}\UTF{2404B}\UTF{2B7AC}%
\UTF{2B7AF}\UTF{2B7C9}\UTF{2B7CF}\UTF{2B7D2}\UTF{26C9E}\UTF{27C3C}\UTF{2B7F0}\UTF{2B765}\UTF{2B80D}\UTF{2B817}%
\UTF{2634C}\UTF{29E3D}\UTF{2A61A}

% end


{\bfseries%
[gt/bx]

\ifuptexmode
 %
% This file is generated from the data of UniJIS-UTF32
% in cid2code.txt (Version 05/18/2022)
% for Adobe-Japan1-7
%
% Reference:
%   https://github.com/adobe-type-tools/cmap-resources/
%   Adobe-Japan1-7/cid2code.txt
%
% A newer CMap may be required for some code points.
%


Adobe-Japan1-0\\
𨳝櫛𥡴𨻶杓巽屠兔冕冤
𡨚𤏐爨🄀

Adobe-Japan1-4\\
🄐🄑🄒🄓🄔🄕🄖🄗🄘🄙
🄚🄛🄜🄝🄞🄟🄠🄡🄢🄣
🄤🄥🄦🄧🄨🄩🅐🅑🅒🅓
🅔🅕🅖🅗🅘🅙🅚🅛🅜🅝
🅞🅟🅠🅡🅢🅣🅤🅥🅦🅧
🅨🅩🄰🄱🄲🄳🄴🄵🄶🄷
🄸🄹🄺🄻🄼🄽🄾🄿🅀🅁
🅂🅃🅄🅅🅆🅇🅈🅉🈂🈷
🅰🅱🅲🅳🅴🅵🅶🅷🅸🅹
🅺🅻🅼🅽🅾🅿🆀🆁🆂🆃
🆄🆅🆆🆇🆈🆉🞜𛄲𛅕眞
𠤎𦥑𫟘沿芽槪割𦈢𠮷𩵋
卿𫞎憲𠩤浩𫝆𫝷滋𠮟勺
爵周将𭕄𠀋城𩙿真𠆢𮕩
𫝑成𧾷𣳾炭𥫗彫潮𡈽冬
𤴔姬𫞉諭輸𥙿𦚰𠘨𠂊𠦄
卉寃拔𦦙𣏌杞𪧦𫞽絣𮉸
𠔿𦉪𠂰𮛪𨦇𨸗𫠚𤋮桒𣲾
𠘑嶲你𣘺𣏾𢘉

Adobe-Japan1-5\\
𡌛𡑮𡢽𡚴𡸴𣇄𣗄𣜿𣝣𤟱
𥒎𥔎𥝱𥧄𥶡𦫿𦹀𧃴𧚄𨉷
𨏍𪆐𠂉𠂢𠂤𠈓𠌫𠎁𠍱𠏹
𠑊𠔉𠗖𠝏𠠇𠠺𠢹𠥼𠦝𠫓
𠬝𠵅𠷡𠺕𠹭𠹤𠽟𡈁𡉕𡉻
𡉴𡋤𡋗𡋽𡌶𡍄𡏄𡑭𡗗𦰩
𡙇𡜆𡝂𡧃𡱖𡴭𡵅𡵸𡵢𡶡
𡶜𡶒𡶷𡷠𡸳𡼞𡽶𡿺𢅻𢌞
𢎭𢛳𢡛𢢫𢦏𢪸𢭏𢭐𢭆𢰝
𢮦𢰤𢷡𣇃𣇵𣆶𣍲𣏓𣏒𣏐
𣏤𣏕𣏚𣏟𣑊𣑑𣑋𣑥𣓤𣕚
𣖔𣘹𣙇𣘸𣜜𣜌𣝤𣟿𣟧𣠤
𣠽𣪘𣱿𣴀𣵀𣷺𣷹𣷓𣽾𤂖
𤄃𤇆𤇾𤎼𤘩𤚥𤢖𤩍𤭖𤭯
𤰖𤸎𤸷𤹪𤺋𥁊𥁕𥄢𥆩𥇥
𥇍𥈞𥉌𥐮𥓙𥖧𥞩𥞴𥧔𥫤
𥫣𥫱𥮲𥱋𥱤𥸮𥹖𥹥𥹢𥻘
𥻂𥻨𥼣𥽜𥿠𥿔𦀌𥿻𦀗𦁠
𦃭𦉰𦊆𣴎𦐂𦙾𦜝𦣝𦣪𦥯
𦧝𦨞𦩘𦪌𦪷𦱳𦳝𦹥𦾔𦿸
𦿶𦿷𧄍𧄹𧏛𧏚𧏾𧐐𧑉𧘕
𧘔𧘱𧚓𧜎𧜣𧝒𧦅𧪄𧮳𧮾
𧯇𧲸𧶠𧸐𨂊𨂻𨊂𨋳𨐌𨑕
𨕫𨗈𨗉𨛗𨛺𨥉𨥆𨥫𨦈𨦺
𨦻𨨞𨨩𨩱𨩃𨪙𨫍𨫤𨫝𨯁
𨯯𨴐𨵱𨷻𨸟𨸶𨺉𨻫𨼲𨿸
𩊠𩊱𩒐𩗏𩛰𩜙𩝐𩣆𩩲𩷛
𩸕𩺊𩹉𩻄𩻩𩻛𩿎𩿗𪀯𪀚
𪃹𪂂𢈘𪎌𪐷𪗱𪘂𪚲𱍐𠃵
𤸄𤿲𧵳再善形慈栟軔𪊲
𠅘𠖱𠛬𫝓𠵘𫝚𫝜𥧌𫝶𢹂
𫝼𠟈𢿫𧦴𫞂𫞋𣟱𫞔𤁋𫞬
𫞯𫟉𫟏𫟒𦲞𧰼𫟰𫝥𫠍𫠗
𦍌𩸽𪘚

% end

\fi
%
% This file is generated from the data of UniJIS-UTF32
% in cid2code.txt (Version 05/18/2022)
% for Adobe-Japan1-7
%
% Reference:
%   https://github.com/adobe-type-tools/cmap-resources/
%   Adobe-Japan1-7/cid2code.txt
%
% A newer CMap may be required for some code points.
%


Adobe-Japan1-0\\
\UTF{28CDD}\UTF{2F8ED}\UTF{25874}\UTF{28EF6}\UTF{2F8DC}\UTF{2F884}\UTF{2F877}\UTF{2F80F}\UTF{2F8D3}\UTF{2F818}%
\UTF{21A1A}\UTF{243D0}\UTF{2F920}\UTF{1F100}

Adobe-Japan1-4\\
\UTF{1F110}\UTF{1F111}\UTF{1F112}\UTF{1F113}\UTF{1F114}\UTF{1F115}\UTF{1F116}\UTF{1F117}\UTF{1F118}\UTF{1F119}%
\UTF{1F11A}\UTF{1F11B}\UTF{1F11C}\UTF{1F11D}\UTF{1F11E}\UTF{1F11F}\UTF{1F120}\UTF{1F121}\UTF{1F122}\UTF{1F123}%
\UTF{1F124}\UTF{1F125}\UTF{1F126}\UTF{1F127}\UTF{1F128}\UTF{1F129}\UTF{1F150}\UTF{1F151}\UTF{1F152}\UTF{1F153}%
\UTF{1F154}\UTF{1F155}\UTF{1F156}\UTF{1F157}\UTF{1F158}\UTF{1F159}\UTF{1F15A}\UTF{1F15B}\UTF{1F15C}\UTF{1F15D}%
\UTF{1F15E}\UTF{1F15F}\UTF{1F160}\UTF{1F161}\UTF{1F162}\UTF{1F163}\UTF{1F164}\UTF{1F165}\UTF{1F166}\UTF{1F167}%
\UTF{1F168}\UTF{1F169}\UTF{1F130}\UTF{1F131}\UTF{1F132}\UTF{1F133}\UTF{1F134}\UTF{1F135}\UTF{1F136}\UTF{1F137}%
\UTF{1F138}\UTF{1F139}\UTF{1F13A}\UTF{1F13B}\UTF{1F13C}\UTF{1F13D}\UTF{1F13E}\UTF{1F13F}\UTF{1F140}\UTF{1F141}%
\UTF{1F142}\UTF{1F143}\UTF{1F144}\UTF{1F145}\UTF{1F146}\UTF{1F147}\UTF{1F148}\UTF{1F149}\UTF{1F202}\UTF{1F237}%
\UTF{1F170}\UTF{1F171}\UTF{1F172}\UTF{1F173}\UTF{1F174}\UTF{1F175}\UTF{1F176}\UTF{1F177}\UTF{1F178}\UTF{1F179}%
\UTF{1F17A}\UTF{1F17B}\UTF{1F17C}\UTF{1F17D}\UTF{1F17E}\UTF{1F17F}\UTF{1F180}\UTF{1F181}\UTF{1F182}\UTF{1F183}%
\UTF{1F184}\UTF{1F185}\UTF{1F186}\UTF{1F187}\UTF{1F188}\UTF{1F189}\UTF{1F79C}\UTF{1B132}\UTF{1B155}\UTF{2F945}%
\UTF{2090E}\UTF{26951}\UTF{2B7D8}\UTF{2F8FC}\UTF{2F995}\UTF{2F8EA}\UTF{2F822}\UTF{26222}\UTF{20BB7}\UTF{29D4B}%
\UTF{2F833}\UTF{2B78E}\UTF{2F8AC}\UTF{20A64}\UTF{2F903}\UTF{2B746}\UTF{2B777}\UTF{2F90B}\UTF{20B9F}\UTF{2F828}%
\UTF{2F921}\UTF{2F83F}\UTF{2F873}\UTF{2D544}\UTF{2000B}\UTF{2F852}\UTF{2967F}\UTF{2F947}\UTF{201A2}\UTF{2E569}%
\UTF{2B751}\UTF{2F8B2}\UTF{27FB7}\UTF{23CFE}\UTF{2F91A}\UTF{25AD7}\UTF{2F89A}\UTF{2F90F}\UTF{2123D}\UTF{2F81A}%
\UTF{24D14}\UTF{2F862}\UTF{2B789}\UTF{2F9D0}\UTF{2F9DF}\UTF{2567F}\UTF{266B0}\UTF{20628}\UTF{2008A}\UTF{20984}%
\UTF{2F82C}\UTF{2F86D}\UTF{2F8B6}\UTF{26999}\UTF{233CC}\UTF{2F8DB}\UTF{2A9E6}\UTF{2B7BD}\UTF{2F96C}\UTF{2E278}%
\UTF{2053F}\UTF{2626A}\UTF{200B0}\UTF{2E6EA}\UTF{28987}\UTF{28E17}\UTF{2B81A}\UTF{242EE}\UTF{2F8E1}\UTF{23CBE}%
\UTF{20611}\UTF{2F9F4}\UTF{2F804}\UTF{2363A}\UTF{233FE}\UTF{22609}

Adobe-Japan1-5\\
\UTF{2131B}\UTF{2146E}\UTF{218BD}\UTF{216B4}\UTF{21E34}\UTF{231C4}\UTF{235C4}\UTF{2373F}\UTF{23763}\UTF{247F1}%
\UTF{2548E}\UTF{2550E}\UTF{25771}\UTF{259C4}\UTF{25DA1}\UTF{26AFF}\UTF{26E40}\UTF{270F4}\UTF{27684}\UTF{28277}%
\UTF{283CD}\UTF{2A190}\UTF{20089}\UTF{200A2}\UTF{200A4}\UTF{20213}\UTF{2032B}\UTF{20381}\UTF{20371}\UTF{203F9}%
\UTF{2044A}\UTF{20509}\UTF{205D6}\UTF{2074F}\UTF{20807}\UTF{2083A}\UTF{208B9}\UTF{2097C}\UTF{2099D}\UTF{20AD3}%
\UTF{20B1D}\UTF{20D45}\UTF{20DE1}\UTF{20E95}\UTF{20E6D}\UTF{20E64}\UTF{20F5F}\UTF{21201}\UTF{21255}\UTF{2127B}%
\UTF{21274}\UTF{212E4}\UTF{212D7}\UTF{212FD}\UTF{21336}\UTF{21344}\UTF{213C4}\UTF{2146D}\UTF{215D7}\UTF{26C29}%
\UTF{21647}\UTF{21706}\UTF{21742}\UTF{219C3}\UTF{21C56}\UTF{21D2D}\UTF{21D45}\UTF{21D78}\UTF{21D62}\UTF{21DA1}%
\UTF{21D9C}\UTF{21D92}\UTF{21DB7}\UTF{21DE0}\UTF{21E33}\UTF{21F1E}\UTF{21F76}\UTF{21FFA}\UTF{2217B}\UTF{2231E}%
\UTF{223AD}\UTF{226F3}\UTF{2285B}\UTF{228AB}\UTF{2298F}\UTF{22AB8}\UTF{22B4F}\UTF{22B50}\UTF{22B46}\UTF{22C1D}%
\UTF{22BA6}\UTF{22C24}\UTF{22DE1}\UTF{231C3}\UTF{231F5}\UTF{231B6}\UTF{23372}\UTF{233D3}\UTF{233D2}\UTF{233D0}%
\UTF{233E4}\UTF{233D5}\UTF{233DA}\UTF{233DF}\UTF{2344A}\UTF{23451}\UTF{2344B}\UTF{23465}\UTF{234E4}\UTF{2355A}%
\UTF{23594}\UTF{23639}\UTF{23647}\UTF{23638}\UTF{2371C}\UTF{2370C}\UTF{23764}\UTF{237FF}\UTF{237E7}\UTF{23824}%
\UTF{2383D}\UTF{23A98}\UTF{23C7F}\UTF{23D00}\UTF{23D40}\UTF{23DFA}\UTF{23DF9}\UTF{23DD3}\UTF{23F7E}\UTF{24096}%
\UTF{24103}\UTF{241C6}\UTF{241FE}\UTF{243BC}\UTF{24629}\UTF{246A5}\UTF{24896}\UTF{24A4D}\UTF{24B56}\UTF{24B6F}%
\UTF{24C16}\UTF{24E0E}\UTF{24E37}\UTF{24E6A}\UTF{24E8B}\UTF{2504A}\UTF{25055}\UTF{25122}\UTF{251A9}\UTF{251E5}%
\UTF{251CD}\UTF{2521E}\UTF{2524C}\UTF{2542E}\UTF{254D9}\UTF{255A7}\UTF{257A9}\UTF{257B4}\UTF{259D4}\UTF{25AE4}%
\UTF{25AE3}\UTF{25AF1}\UTF{25BB2}\UTF{25C4B}\UTF{25C64}\UTF{25E2E}\UTF{25E56}\UTF{25E65}\UTF{25E62}\UTF{25ED8}%
\UTF{25EC2}\UTF{25EE8}\UTF{25F23}\UTF{25F5C}\UTF{25FE0}\UTF{25FD4}\UTF{2600C}\UTF{25FFB}\UTF{26017}\UTF{26060}%
\UTF{260ED}\UTF{26270}\UTF{26286}\UTF{23D0E}\UTF{26402}\UTF{2667E}\UTF{2671D}\UTF{268DD}\UTF{268EA}\UTF{2696F}%
\UTF{269DD}\UTF{26A1E}\UTF{26A58}\UTF{26A8C}\UTF{26AB7}\UTF{26C73}\UTF{26CDD}\UTF{26E65}\UTF{26F94}\UTF{26FF8}%
\UTF{26FF6}\UTF{26FF7}\UTF{2710D}\UTF{27139}\UTF{273DB}\UTF{273DA}\UTF{273FE}\UTF{27410}\UTF{27449}\UTF{27615}%
\UTF{27614}\UTF{27631}\UTF{27693}\UTF{2770E}\UTF{27723}\UTF{27752}\UTF{27985}\UTF{27A84}\UTF{27BB3}\UTF{27BBE}%
\UTF{27BC7}\UTF{27CB8}\UTF{27DA0}\UTF{27E10}\UTF{2808A}\UTF{280BB}\UTF{28282}\UTF{282F3}\UTF{2840C}\UTF{28455}%
\UTF{2856B}\UTF{285C8}\UTF{285C9}\UTF{286D7}\UTF{286FA}\UTF{28949}\UTF{28946}\UTF{2896B}\UTF{28988}\UTF{289BA}%
\UTF{289BB}\UTF{28A1E}\UTF{28A29}\UTF{28A71}\UTF{28A43}\UTF{28A99}\UTF{28ACD}\UTF{28AE4}\UTF{28ADD}\UTF{28BC1}%
\UTF{28BEF}\UTF{28D10}\UTF{28D71}\UTF{28DFB}\UTF{28E1F}\UTF{28E36}\UTF{28E89}\UTF{28EEB}\UTF{28F32}\UTF{28FF8}%
\UTF{292A0}\UTF{292B1}\UTF{29490}\UTF{295CF}\UTF{296F0}\UTF{29719}\UTF{29750}\UTF{298C6}\UTF{29A72}\UTF{29DDB}%
\UTF{29E15}\UTF{29E8A}\UTF{29E49}\UTF{29EC4}\UTF{29EE9}\UTF{29EDB}\UTF{29FCE}\UTF{29FD7}\UTF{2A02F}\UTF{2A01A}%
\UTF{2A0F9}\UTF{2A082}\UTF{22218}\UTF{2A38C}\UTF{2A437}\UTF{2A5F1}\UTF{2A602}\UTF{2A6B2}\UTF{31350}\UTF{200F5}%
\UTF{24E04}\UTF{24FF2}\UTF{27D73}\UTF{2F815}\UTF{2F846}\UTF{2F899}\UTF{2F8A6}\UTF{2F8E5}\UTF{2F9DE}\UTF{2A2B2}%
\UTF{20158}\UTF{205B1}\UTF{206EC}\UTF{2B753}\UTF{20D58}\UTF{2B75A}\UTF{2B75C}\UTF{259CC}\UTF{2B776}\UTF{22E42}%
\UTF{2B77C}\UTF{207C8}\UTF{22FEB}\UTF{279B4}\UTF{2B782}\UTF{2B78B}\UTF{237F1}\UTF{2B794}\UTF{2404B}\UTF{2B7AC}%
\UTF{2B7AF}\UTF{2B7C9}\UTF{2B7CF}\UTF{2B7D2}\UTF{26C9E}\UTF{27C3C}\UTF{2B7F0}\UTF{2B765}\UTF{2B80D}\UTF{2B817}%
\UTF{2634C}\UTF{29E3D}\UTF{2A61A}

% end


}}

{\mgfamily
[mg/m]

\ifuptexmode
 %
% This file is generated from the data of UniJIS-UTF32
% in cid2code.txt (Version 05/18/2022)
% for Adobe-Japan1-7
%
% Reference:
%   https://github.com/adobe-type-tools/cmap-resources/
%   Adobe-Japan1-7/cid2code.txt
%
% A newer CMap may be required for some code points.
%


Adobe-Japan1-0\\
𨳝櫛𥡴𨻶杓巽屠兔冕冤
𡨚𤏐爨🄀

Adobe-Japan1-4\\
🄐🄑🄒🄓🄔🄕🄖🄗🄘🄙
🄚🄛🄜🄝🄞🄟🄠🄡🄢🄣
🄤🄥🄦🄧🄨🄩🅐🅑🅒🅓
🅔🅕🅖🅗🅘🅙🅚🅛🅜🅝
🅞🅟🅠🅡🅢🅣🅤🅥🅦🅧
🅨🅩🄰🄱🄲🄳🄴🄵🄶🄷
🄸🄹🄺🄻🄼🄽🄾🄿🅀🅁
🅂🅃🅄🅅🅆🅇🅈🅉🈂🈷
🅰🅱🅲🅳🅴🅵🅶🅷🅸🅹
🅺🅻🅼🅽🅾🅿🆀🆁🆂🆃
🆄🆅🆆🆇🆈🆉🞜𛄲𛅕眞
𠤎𦥑𫟘沿芽槪割𦈢𠮷𩵋
卿𫞎憲𠩤浩𫝆𫝷滋𠮟勺
爵周将𭕄𠀋城𩙿真𠆢𮕩
𫝑成𧾷𣳾炭𥫗彫潮𡈽冬
𤴔姬𫞉諭輸𥙿𦚰𠘨𠂊𠦄
卉寃拔𦦙𣏌杞𪧦𫞽絣𮉸
𠔿𦉪𠂰𮛪𨦇𨸗𫠚𤋮桒𣲾
𠘑嶲你𣘺𣏾𢘉

Adobe-Japan1-5\\
𡌛𡑮𡢽𡚴𡸴𣇄𣗄𣜿𣝣𤟱
𥒎𥔎𥝱𥧄𥶡𦫿𦹀𧃴𧚄𨉷
𨏍𪆐𠂉𠂢𠂤𠈓𠌫𠎁𠍱𠏹
𠑊𠔉𠗖𠝏𠠇𠠺𠢹𠥼𠦝𠫓
𠬝𠵅𠷡𠺕𠹭𠹤𠽟𡈁𡉕𡉻
𡉴𡋤𡋗𡋽𡌶𡍄𡏄𡑭𡗗𦰩
𡙇𡜆𡝂𡧃𡱖𡴭𡵅𡵸𡵢𡶡
𡶜𡶒𡶷𡷠𡸳𡼞𡽶𡿺𢅻𢌞
𢎭𢛳𢡛𢢫𢦏𢪸𢭏𢭐𢭆𢰝
𢮦𢰤𢷡𣇃𣇵𣆶𣍲𣏓𣏒𣏐
𣏤𣏕𣏚𣏟𣑊𣑑𣑋𣑥𣓤𣕚
𣖔𣘹𣙇𣘸𣜜𣜌𣝤𣟿𣟧𣠤
𣠽𣪘𣱿𣴀𣵀𣷺𣷹𣷓𣽾𤂖
𤄃𤇆𤇾𤎼𤘩𤚥𤢖𤩍𤭖𤭯
𤰖𤸎𤸷𤹪𤺋𥁊𥁕𥄢𥆩𥇥
𥇍𥈞𥉌𥐮𥓙𥖧𥞩𥞴𥧔𥫤
𥫣𥫱𥮲𥱋𥱤𥸮𥹖𥹥𥹢𥻘
𥻂𥻨𥼣𥽜𥿠𥿔𦀌𥿻𦀗𦁠
𦃭𦉰𦊆𣴎𦐂𦙾𦜝𦣝𦣪𦥯
𦧝𦨞𦩘𦪌𦪷𦱳𦳝𦹥𦾔𦿸
𦿶𦿷𧄍𧄹𧏛𧏚𧏾𧐐𧑉𧘕
𧘔𧘱𧚓𧜎𧜣𧝒𧦅𧪄𧮳𧮾
𧯇𧲸𧶠𧸐𨂊𨂻𨊂𨋳𨐌𨑕
𨕫𨗈𨗉𨛗𨛺𨥉𨥆𨥫𨦈𨦺
𨦻𨨞𨨩𨩱𨩃𨪙𨫍𨫤𨫝𨯁
𨯯𨴐𨵱𨷻𨸟𨸶𨺉𨻫𨼲𨿸
𩊠𩊱𩒐𩗏𩛰𩜙𩝐𩣆𩩲𩷛
𩸕𩺊𩹉𩻄𩻩𩻛𩿎𩿗𪀯𪀚
𪃹𪂂𢈘𪎌𪐷𪗱𪘂𪚲𱍐𠃵
𤸄𤿲𧵳再善形慈栟軔𪊲
𠅘𠖱𠛬𫝓𠵘𫝚𫝜𥧌𫝶𢹂
𫝼𠟈𢿫𧦴𫞂𫞋𣟱𫞔𤁋𫞬
𫞯𫟉𫟏𫟒𦲞𧰼𫟰𫝥𫠍𫠗
𦍌𩸽𪘚

% end

\fi
%
% This file is generated from the data of UniJIS-UTF32
% in cid2code.txt (Version 05/18/2022)
% for Adobe-Japan1-7
%
% Reference:
%   https://github.com/adobe-type-tools/cmap-resources/
%   Adobe-Japan1-7/cid2code.txt
%
% A newer CMap may be required for some code points.
%


Adobe-Japan1-0\\
\UTF{28CDD}\UTF{2F8ED}\UTF{25874}\UTF{28EF6}\UTF{2F8DC}\UTF{2F884}\UTF{2F877}\UTF{2F80F}\UTF{2F8D3}\UTF{2F818}%
\UTF{21A1A}\UTF{243D0}\UTF{2F920}\UTF{1F100}

Adobe-Japan1-4\\
\UTF{1F110}\UTF{1F111}\UTF{1F112}\UTF{1F113}\UTF{1F114}\UTF{1F115}\UTF{1F116}\UTF{1F117}\UTF{1F118}\UTF{1F119}%
\UTF{1F11A}\UTF{1F11B}\UTF{1F11C}\UTF{1F11D}\UTF{1F11E}\UTF{1F11F}\UTF{1F120}\UTF{1F121}\UTF{1F122}\UTF{1F123}%
\UTF{1F124}\UTF{1F125}\UTF{1F126}\UTF{1F127}\UTF{1F128}\UTF{1F129}\UTF{1F150}\UTF{1F151}\UTF{1F152}\UTF{1F153}%
\UTF{1F154}\UTF{1F155}\UTF{1F156}\UTF{1F157}\UTF{1F158}\UTF{1F159}\UTF{1F15A}\UTF{1F15B}\UTF{1F15C}\UTF{1F15D}%
\UTF{1F15E}\UTF{1F15F}\UTF{1F160}\UTF{1F161}\UTF{1F162}\UTF{1F163}\UTF{1F164}\UTF{1F165}\UTF{1F166}\UTF{1F167}%
\UTF{1F168}\UTF{1F169}\UTF{1F130}\UTF{1F131}\UTF{1F132}\UTF{1F133}\UTF{1F134}\UTF{1F135}\UTF{1F136}\UTF{1F137}%
\UTF{1F138}\UTF{1F139}\UTF{1F13A}\UTF{1F13B}\UTF{1F13C}\UTF{1F13D}\UTF{1F13E}\UTF{1F13F}\UTF{1F140}\UTF{1F141}%
\UTF{1F142}\UTF{1F143}\UTF{1F144}\UTF{1F145}\UTF{1F146}\UTF{1F147}\UTF{1F148}\UTF{1F149}\UTF{1F202}\UTF{1F237}%
\UTF{1F170}\UTF{1F171}\UTF{1F172}\UTF{1F173}\UTF{1F174}\UTF{1F175}\UTF{1F176}\UTF{1F177}\UTF{1F178}\UTF{1F179}%
\UTF{1F17A}\UTF{1F17B}\UTF{1F17C}\UTF{1F17D}\UTF{1F17E}\UTF{1F17F}\UTF{1F180}\UTF{1F181}\UTF{1F182}\UTF{1F183}%
\UTF{1F184}\UTF{1F185}\UTF{1F186}\UTF{1F187}\UTF{1F188}\UTF{1F189}\UTF{1F79C}\UTF{1B132}\UTF{1B155}\UTF{2F945}%
\UTF{2090E}\UTF{26951}\UTF{2B7D8}\UTF{2F8FC}\UTF{2F995}\UTF{2F8EA}\UTF{2F822}\UTF{26222}\UTF{20BB7}\UTF{29D4B}%
\UTF{2F833}\UTF{2B78E}\UTF{2F8AC}\UTF{20A64}\UTF{2F903}\UTF{2B746}\UTF{2B777}\UTF{2F90B}\UTF{20B9F}\UTF{2F828}%
\UTF{2F921}\UTF{2F83F}\UTF{2F873}\UTF{2D544}\UTF{2000B}\UTF{2F852}\UTF{2967F}\UTF{2F947}\UTF{201A2}\UTF{2E569}%
\UTF{2B751}\UTF{2F8B2}\UTF{27FB7}\UTF{23CFE}\UTF{2F91A}\UTF{25AD7}\UTF{2F89A}\UTF{2F90F}\UTF{2123D}\UTF{2F81A}%
\UTF{24D14}\UTF{2F862}\UTF{2B789}\UTF{2F9D0}\UTF{2F9DF}\UTF{2567F}\UTF{266B0}\UTF{20628}\UTF{2008A}\UTF{20984}%
\UTF{2F82C}\UTF{2F86D}\UTF{2F8B6}\UTF{26999}\UTF{233CC}\UTF{2F8DB}\UTF{2A9E6}\UTF{2B7BD}\UTF{2F96C}\UTF{2E278}%
\UTF{2053F}\UTF{2626A}\UTF{200B0}\UTF{2E6EA}\UTF{28987}\UTF{28E17}\UTF{2B81A}\UTF{242EE}\UTF{2F8E1}\UTF{23CBE}%
\UTF{20611}\UTF{2F9F4}\UTF{2F804}\UTF{2363A}\UTF{233FE}\UTF{22609}

Adobe-Japan1-5\\
\UTF{2131B}\UTF{2146E}\UTF{218BD}\UTF{216B4}\UTF{21E34}\UTF{231C4}\UTF{235C4}\UTF{2373F}\UTF{23763}\UTF{247F1}%
\UTF{2548E}\UTF{2550E}\UTF{25771}\UTF{259C4}\UTF{25DA1}\UTF{26AFF}\UTF{26E40}\UTF{270F4}\UTF{27684}\UTF{28277}%
\UTF{283CD}\UTF{2A190}\UTF{20089}\UTF{200A2}\UTF{200A4}\UTF{20213}\UTF{2032B}\UTF{20381}\UTF{20371}\UTF{203F9}%
\UTF{2044A}\UTF{20509}\UTF{205D6}\UTF{2074F}\UTF{20807}\UTF{2083A}\UTF{208B9}\UTF{2097C}\UTF{2099D}\UTF{20AD3}%
\UTF{20B1D}\UTF{20D45}\UTF{20DE1}\UTF{20E95}\UTF{20E6D}\UTF{20E64}\UTF{20F5F}\UTF{21201}\UTF{21255}\UTF{2127B}%
\UTF{21274}\UTF{212E4}\UTF{212D7}\UTF{212FD}\UTF{21336}\UTF{21344}\UTF{213C4}\UTF{2146D}\UTF{215D7}\UTF{26C29}%
\UTF{21647}\UTF{21706}\UTF{21742}\UTF{219C3}\UTF{21C56}\UTF{21D2D}\UTF{21D45}\UTF{21D78}\UTF{21D62}\UTF{21DA1}%
\UTF{21D9C}\UTF{21D92}\UTF{21DB7}\UTF{21DE0}\UTF{21E33}\UTF{21F1E}\UTF{21F76}\UTF{21FFA}\UTF{2217B}\UTF{2231E}%
\UTF{223AD}\UTF{226F3}\UTF{2285B}\UTF{228AB}\UTF{2298F}\UTF{22AB8}\UTF{22B4F}\UTF{22B50}\UTF{22B46}\UTF{22C1D}%
\UTF{22BA6}\UTF{22C24}\UTF{22DE1}\UTF{231C3}\UTF{231F5}\UTF{231B6}\UTF{23372}\UTF{233D3}\UTF{233D2}\UTF{233D0}%
\UTF{233E4}\UTF{233D5}\UTF{233DA}\UTF{233DF}\UTF{2344A}\UTF{23451}\UTF{2344B}\UTF{23465}\UTF{234E4}\UTF{2355A}%
\UTF{23594}\UTF{23639}\UTF{23647}\UTF{23638}\UTF{2371C}\UTF{2370C}\UTF{23764}\UTF{237FF}\UTF{237E7}\UTF{23824}%
\UTF{2383D}\UTF{23A98}\UTF{23C7F}\UTF{23D00}\UTF{23D40}\UTF{23DFA}\UTF{23DF9}\UTF{23DD3}\UTF{23F7E}\UTF{24096}%
\UTF{24103}\UTF{241C6}\UTF{241FE}\UTF{243BC}\UTF{24629}\UTF{246A5}\UTF{24896}\UTF{24A4D}\UTF{24B56}\UTF{24B6F}%
\UTF{24C16}\UTF{24E0E}\UTF{24E37}\UTF{24E6A}\UTF{24E8B}\UTF{2504A}\UTF{25055}\UTF{25122}\UTF{251A9}\UTF{251E5}%
\UTF{251CD}\UTF{2521E}\UTF{2524C}\UTF{2542E}\UTF{254D9}\UTF{255A7}\UTF{257A9}\UTF{257B4}\UTF{259D4}\UTF{25AE4}%
\UTF{25AE3}\UTF{25AF1}\UTF{25BB2}\UTF{25C4B}\UTF{25C64}\UTF{25E2E}\UTF{25E56}\UTF{25E65}\UTF{25E62}\UTF{25ED8}%
\UTF{25EC2}\UTF{25EE8}\UTF{25F23}\UTF{25F5C}\UTF{25FE0}\UTF{25FD4}\UTF{2600C}\UTF{25FFB}\UTF{26017}\UTF{26060}%
\UTF{260ED}\UTF{26270}\UTF{26286}\UTF{23D0E}\UTF{26402}\UTF{2667E}\UTF{2671D}\UTF{268DD}\UTF{268EA}\UTF{2696F}%
\UTF{269DD}\UTF{26A1E}\UTF{26A58}\UTF{26A8C}\UTF{26AB7}\UTF{26C73}\UTF{26CDD}\UTF{26E65}\UTF{26F94}\UTF{26FF8}%
\UTF{26FF6}\UTF{26FF7}\UTF{2710D}\UTF{27139}\UTF{273DB}\UTF{273DA}\UTF{273FE}\UTF{27410}\UTF{27449}\UTF{27615}%
\UTF{27614}\UTF{27631}\UTF{27693}\UTF{2770E}\UTF{27723}\UTF{27752}\UTF{27985}\UTF{27A84}\UTF{27BB3}\UTF{27BBE}%
\UTF{27BC7}\UTF{27CB8}\UTF{27DA0}\UTF{27E10}\UTF{2808A}\UTF{280BB}\UTF{28282}\UTF{282F3}\UTF{2840C}\UTF{28455}%
\UTF{2856B}\UTF{285C8}\UTF{285C9}\UTF{286D7}\UTF{286FA}\UTF{28949}\UTF{28946}\UTF{2896B}\UTF{28988}\UTF{289BA}%
\UTF{289BB}\UTF{28A1E}\UTF{28A29}\UTF{28A71}\UTF{28A43}\UTF{28A99}\UTF{28ACD}\UTF{28AE4}\UTF{28ADD}\UTF{28BC1}%
\UTF{28BEF}\UTF{28D10}\UTF{28D71}\UTF{28DFB}\UTF{28E1F}\UTF{28E36}\UTF{28E89}\UTF{28EEB}\UTF{28F32}\UTF{28FF8}%
\UTF{292A0}\UTF{292B1}\UTF{29490}\UTF{295CF}\UTF{296F0}\UTF{29719}\UTF{29750}\UTF{298C6}\UTF{29A72}\UTF{29DDB}%
\UTF{29E15}\UTF{29E8A}\UTF{29E49}\UTF{29EC4}\UTF{29EE9}\UTF{29EDB}\UTF{29FCE}\UTF{29FD7}\UTF{2A02F}\UTF{2A01A}%
\UTF{2A0F9}\UTF{2A082}\UTF{22218}\UTF{2A38C}\UTF{2A437}\UTF{2A5F1}\UTF{2A602}\UTF{2A6B2}\UTF{31350}\UTF{200F5}%
\UTF{24E04}\UTF{24FF2}\UTF{27D73}\UTF{2F815}\UTF{2F846}\UTF{2F899}\UTF{2F8A6}\UTF{2F8E5}\UTF{2F9DE}\UTF{2A2B2}%
\UTF{20158}\UTF{205B1}\UTF{206EC}\UTF{2B753}\UTF{20D58}\UTF{2B75A}\UTF{2B75C}\UTF{259CC}\UTF{2B776}\UTF{22E42}%
\UTF{2B77C}\UTF{207C8}\UTF{22FEB}\UTF{279B4}\UTF{2B782}\UTF{2B78B}\UTF{237F1}\UTF{2B794}\UTF{2404B}\UTF{2B7AC}%
\UTF{2B7AF}\UTF{2B7C9}\UTF{2B7CF}\UTF{2B7D2}\UTF{26C9E}\UTF{27C3C}\UTF{2B7F0}\UTF{2B765}\UTF{2B80D}\UTF{2B817}%
\UTF{2634C}\UTF{29E3D}\UTF{2A61A}

% end


}

\clearpage
[mc/m]

\input{sp_cns_utf}

%
% This file is generated from the data of UniGB-UTF32
% in cid2code.txt (Version 09/21/2023)
% for Adobe-GB1-6
%
% Reference:
%   https://github.com/adobe-type-tools/cmap-resources/
%   Adobe-GB1-6/cid2code.txt
%
% A newer CMap may be required for some code points.
%


Adobe-GB1-2\\
\UTFC{20087}\UTFC{20089}\UTFC{200CC}\UTFC{215D7}\UTFC{2298F}\UTFC{20509}\UTFC{2099D}\UTFC{241FE}

Adobe-GB1-6\\
\UTFC{20164}\UTFC{20676}\UTFC{20CD0}\UTFC{2139A}\UTFC{21413}\UTFC{235CB}\UTFC{23C97}\UTFC{23C98}\UTFC{23E23}\UTFC{249DB}%
\UTFC{24A7D}\UTFC{24AC9}\UTFC{25532}\UTFC{25562}\UTFC{255A8}\UTFC{25ED7}\UTFC{26221}\UTFC{2648D}\UTFC{26676}\UTFC{2677C}%
\UTFC{26B5C}\UTFC{26C21}\UTFC{27FF9}\UTFC{28408}\UTFC{28678}\UTFC{28695}\UTFC{287E0}\UTFC{28B49}\UTFC{28C47}\UTFC{28C4F}%
\UTFC{28C51}\UTFC{28C54}\UTFC{28E99}\UTFC{29F7E}\UTFC{29F83}\UTFC{29F8C}\UTFC{2A7DD}\UTFC{2A8FB}\UTFC{2A917}\UTFC{2AA30}%
\UTFC{2AA36}\UTFC{2AA58}\UTFC{2AFA2}\UTFC{2B127}\UTFC{2B128}\UTFC{2B137}\UTFC{2B138}\UTFC{2B1ED}\UTFC{2B300}\UTFC{2B363}%
\UTFC{2B36F}\UTFC{2B372}\UTFC{2B37D}\UTFC{2B404}\UTFC{2B410}\UTFC{2B413}\UTFC{2B461}\UTFC{2B4E7}\UTFC{2B4EF}\UTFC{2B4F6}%
\UTFC{2B4F9}\UTFC{2B50D}\UTFC{2B50E}\UTFC{2B536}\UTFC{2B5AE}\UTFC{2B5AF}\UTFC{2B5B3}\UTFC{2B5E7}\UTFC{2B5F4}\UTFC{2B61C}%
\UTFC{2B61D}\UTFC{2B626}\UTFC{2B627}\UTFC{2B628}\UTFC{2B62A}\UTFC{2B62C}\UTFC{2B695}\UTFC{2B696}\UTFC{2B6AD}\UTFC{2B6ED}%
\UTFC{2B7A9}\UTFC{2B7C5}\UTFC{2B7E6}\UTFC{2B7F9}\UTFC{2B7FC}\UTFC{2B806}\UTFC{2B80A}\UTFC{2B81C}\UTFC{2B8B8}\UTFC{2BAC7}%
\UTFC{2BB5F}\UTFC{2BB62}\UTFC{2BB7C}\UTFC{2BB83}\UTFC{2BC1B}\UTFC{2BD77}\UTFC{2BD87}\UTFC{2BDF7}\UTFC{2BE29}\UTFC{2C029}%
\UTFC{2C02A}\UTFC{2C0A9}\UTFC{2C0CA}\UTFC{2C1D5}\UTFC{2C1D9}\UTFC{2C1F9}\UTFC{2C27C}\UTFC{2C288}\UTFC{2C2A4}\UTFC{2C317}%
\UTFC{2C35B}\UTFC{2C361}\UTFC{2C364}\UTFC{2C488}\UTFC{2C494}\UTFC{2C497}\UTFC{2C542}\UTFC{2C613}\UTFC{2C618}\UTFC{2C621}%
\UTFC{2C629}\UTFC{2C62B}\UTFC{2C62C}\UTFC{2C62D}\UTFC{2C62F}\UTFC{2C642}\UTFC{2C64A}\UTFC{2C64B}\UTFC{2C72C}\UTFC{2C72F}%
\UTFC{2C79F}\UTFC{2C7C1}\UTFC{2C7FD}\UTFC{2C8D9}\UTFC{2C8DE}\UTFC{2C8E1}\UTFC{2C8F3}\UTFC{2C907}\UTFC{2C90A}\UTFC{2C91D}%
\UTFC{2CA02}\UTFC{2CA0E}\UTFC{2CA7D}\UTFC{2CAA9}\UTFC{2CB29}\UTFC{2CB2D}\UTFC{2CB2E}\UTFC{2CB31}\UTFC{2CB38}\UTFC{2CB39}%
\UTFC{2CB3B}\UTFC{2CB3F}\UTFC{2CB41}\UTFC{2CB4A}\UTFC{2CB4E}\UTFC{2CB5A}\UTFC{2CB5B}\UTFC{2CB64}\UTFC{2CB69}\UTFC{2CB6C}%
\UTFC{2CB6F}\UTFC{2CB73}\UTFC{2CB76}\UTFC{2CB78}\UTFC{2CB7C}\UTFC{2CBB1}\UTFC{2CBBF}\UTFC{2CBC0}\UTFC{2CBCE}\UTFC{2CC56}%
\UTFC{2CC5F}\UTFC{2CCF5}\UTFC{2CCF6}\UTFC{2CCFD}\UTFC{2CCFF}\UTFC{2CD02}\UTFC{2CD03}\UTFC{2CD0A}\UTFC{2CD8B}\UTFC{2CD8D}%
\UTFC{2CD8F}\UTFC{2CD90}\UTFC{2CD9F}\UTFC{2CDA0}\UTFC{2CDA8}\UTFC{2CDAD}\UTFC{2CDAE}\UTFC{2CDD5}\UTFC{2CE18}\UTFC{2CE1A}%
\UTFC{2CE23}\UTFC{2CE26}\UTFC{2CE2A}\UTFC{2CE7C}\UTFC{2CE88}\UTFC{2CE93}

% end


%\end{document}

{\bfseries%
[mc/bx]

\input{sp_cns_utf}

%
% This file is generated from the data of UniGB-UTF32
% in cid2code.txt (Version 09/21/2023)
% for Adobe-GB1-6
%
% Reference:
%   https://github.com/adobe-type-tools/cmap-resources/
%   Adobe-GB1-6/cid2code.txt
%
% A newer CMap may be required for some code points.
%


Adobe-GB1-2\\
\UTFC{20087}\UTFC{20089}\UTFC{200CC}\UTFC{215D7}\UTFC{2298F}\UTFC{20509}\UTFC{2099D}\UTFC{241FE}

Adobe-GB1-6\\
\UTFC{20164}\UTFC{20676}\UTFC{20CD0}\UTFC{2139A}\UTFC{21413}\UTFC{235CB}\UTFC{23C97}\UTFC{23C98}\UTFC{23E23}\UTFC{249DB}%
\UTFC{24A7D}\UTFC{24AC9}\UTFC{25532}\UTFC{25562}\UTFC{255A8}\UTFC{25ED7}\UTFC{26221}\UTFC{2648D}\UTFC{26676}\UTFC{2677C}%
\UTFC{26B5C}\UTFC{26C21}\UTFC{27FF9}\UTFC{28408}\UTFC{28678}\UTFC{28695}\UTFC{287E0}\UTFC{28B49}\UTFC{28C47}\UTFC{28C4F}%
\UTFC{28C51}\UTFC{28C54}\UTFC{28E99}\UTFC{29F7E}\UTFC{29F83}\UTFC{29F8C}\UTFC{2A7DD}\UTFC{2A8FB}\UTFC{2A917}\UTFC{2AA30}%
\UTFC{2AA36}\UTFC{2AA58}\UTFC{2AFA2}\UTFC{2B127}\UTFC{2B128}\UTFC{2B137}\UTFC{2B138}\UTFC{2B1ED}\UTFC{2B300}\UTFC{2B363}%
\UTFC{2B36F}\UTFC{2B372}\UTFC{2B37D}\UTFC{2B404}\UTFC{2B410}\UTFC{2B413}\UTFC{2B461}\UTFC{2B4E7}\UTFC{2B4EF}\UTFC{2B4F6}%
\UTFC{2B4F9}\UTFC{2B50D}\UTFC{2B50E}\UTFC{2B536}\UTFC{2B5AE}\UTFC{2B5AF}\UTFC{2B5B3}\UTFC{2B5E7}\UTFC{2B5F4}\UTFC{2B61C}%
\UTFC{2B61D}\UTFC{2B626}\UTFC{2B627}\UTFC{2B628}\UTFC{2B62A}\UTFC{2B62C}\UTFC{2B695}\UTFC{2B696}\UTFC{2B6AD}\UTFC{2B6ED}%
\UTFC{2B7A9}\UTFC{2B7C5}\UTFC{2B7E6}\UTFC{2B7F9}\UTFC{2B7FC}\UTFC{2B806}\UTFC{2B80A}\UTFC{2B81C}\UTFC{2B8B8}\UTFC{2BAC7}%
\UTFC{2BB5F}\UTFC{2BB62}\UTFC{2BB7C}\UTFC{2BB83}\UTFC{2BC1B}\UTFC{2BD77}\UTFC{2BD87}\UTFC{2BDF7}\UTFC{2BE29}\UTFC{2C029}%
\UTFC{2C02A}\UTFC{2C0A9}\UTFC{2C0CA}\UTFC{2C1D5}\UTFC{2C1D9}\UTFC{2C1F9}\UTFC{2C27C}\UTFC{2C288}\UTFC{2C2A4}\UTFC{2C317}%
\UTFC{2C35B}\UTFC{2C361}\UTFC{2C364}\UTFC{2C488}\UTFC{2C494}\UTFC{2C497}\UTFC{2C542}\UTFC{2C613}\UTFC{2C618}\UTFC{2C621}%
\UTFC{2C629}\UTFC{2C62B}\UTFC{2C62C}\UTFC{2C62D}\UTFC{2C62F}\UTFC{2C642}\UTFC{2C64A}\UTFC{2C64B}\UTFC{2C72C}\UTFC{2C72F}%
\UTFC{2C79F}\UTFC{2C7C1}\UTFC{2C7FD}\UTFC{2C8D9}\UTFC{2C8DE}\UTFC{2C8E1}\UTFC{2C8F3}\UTFC{2C907}\UTFC{2C90A}\UTFC{2C91D}%
\UTFC{2CA02}\UTFC{2CA0E}\UTFC{2CA7D}\UTFC{2CAA9}\UTFC{2CB29}\UTFC{2CB2D}\UTFC{2CB2E}\UTFC{2CB31}\UTFC{2CB38}\UTFC{2CB39}%
\UTFC{2CB3B}\UTFC{2CB3F}\UTFC{2CB41}\UTFC{2CB4A}\UTFC{2CB4E}\UTFC{2CB5A}\UTFC{2CB5B}\UTFC{2CB64}\UTFC{2CB69}\UTFC{2CB6C}%
\UTFC{2CB6F}\UTFC{2CB73}\UTFC{2CB76}\UTFC{2CB78}\UTFC{2CB7C}\UTFC{2CBB1}\UTFC{2CBBF}\UTFC{2CBC0}\UTFC{2CBCE}\UTFC{2CC56}%
\UTFC{2CC5F}\UTFC{2CCF5}\UTFC{2CCF6}\UTFC{2CCFD}\UTFC{2CCFF}\UTFC{2CD02}\UTFC{2CD03}\UTFC{2CD0A}\UTFC{2CD8B}\UTFC{2CD8D}%
\UTFC{2CD8F}\UTFC{2CD90}\UTFC{2CD9F}\UTFC{2CDA0}\UTFC{2CDA8}\UTFC{2CDAD}\UTFC{2CDAE}\UTFC{2CDD5}\UTFC{2CE18}\UTFC{2CE1A}%
\UTFC{2CE23}\UTFC{2CE26}\UTFC{2CE2A}\UTFC{2CE7C}\UTFC{2CE88}\UTFC{2CE93}

% end


}

\end{document}


{\gtfamily
[gt/m]

\input{sp_cns_utf}

%
% This file is generated from the data of UniGB-UTF32
% in cid2code.txt (Version 09/21/2023)
% for Adobe-GB1-6
%
% Reference:
%   https://github.com/adobe-type-tools/cmap-resources/
%   Adobe-GB1-6/cid2code.txt
%
% A newer CMap may be required for some code points.
%


Adobe-GB1-2\\
\UTFC{20087}\UTFC{20089}\UTFC{200CC}\UTFC{215D7}\UTFC{2298F}\UTFC{20509}\UTFC{2099D}\UTFC{241FE}

Adobe-GB1-6\\
\UTFC{20164}\UTFC{20676}\UTFC{20CD0}\UTFC{2139A}\UTFC{21413}\UTFC{235CB}\UTFC{23C97}\UTFC{23C98}\UTFC{23E23}\UTFC{249DB}%
\UTFC{24A7D}\UTFC{24AC9}\UTFC{25532}\UTFC{25562}\UTFC{255A8}\UTFC{25ED7}\UTFC{26221}\UTFC{2648D}\UTFC{26676}\UTFC{2677C}%
\UTFC{26B5C}\UTFC{26C21}\UTFC{27FF9}\UTFC{28408}\UTFC{28678}\UTFC{28695}\UTFC{287E0}\UTFC{28B49}\UTFC{28C47}\UTFC{28C4F}%
\UTFC{28C51}\UTFC{28C54}\UTFC{28E99}\UTFC{29F7E}\UTFC{29F83}\UTFC{29F8C}\UTFC{2A7DD}\UTFC{2A8FB}\UTFC{2A917}\UTFC{2AA30}%
\UTFC{2AA36}\UTFC{2AA58}\UTFC{2AFA2}\UTFC{2B127}\UTFC{2B128}\UTFC{2B137}\UTFC{2B138}\UTFC{2B1ED}\UTFC{2B300}\UTFC{2B363}%
\UTFC{2B36F}\UTFC{2B372}\UTFC{2B37D}\UTFC{2B404}\UTFC{2B410}\UTFC{2B413}\UTFC{2B461}\UTFC{2B4E7}\UTFC{2B4EF}\UTFC{2B4F6}%
\UTFC{2B4F9}\UTFC{2B50D}\UTFC{2B50E}\UTFC{2B536}\UTFC{2B5AE}\UTFC{2B5AF}\UTFC{2B5B3}\UTFC{2B5E7}\UTFC{2B5F4}\UTFC{2B61C}%
\UTFC{2B61D}\UTFC{2B626}\UTFC{2B627}\UTFC{2B628}\UTFC{2B62A}\UTFC{2B62C}\UTFC{2B695}\UTFC{2B696}\UTFC{2B6AD}\UTFC{2B6ED}%
\UTFC{2B7A9}\UTFC{2B7C5}\UTFC{2B7E6}\UTFC{2B7F9}\UTFC{2B7FC}\UTFC{2B806}\UTFC{2B80A}\UTFC{2B81C}\UTFC{2B8B8}\UTFC{2BAC7}%
\UTFC{2BB5F}\UTFC{2BB62}\UTFC{2BB7C}\UTFC{2BB83}\UTFC{2BC1B}\UTFC{2BD77}\UTFC{2BD87}\UTFC{2BDF7}\UTFC{2BE29}\UTFC{2C029}%
\UTFC{2C02A}\UTFC{2C0A9}\UTFC{2C0CA}\UTFC{2C1D5}\UTFC{2C1D9}\UTFC{2C1F9}\UTFC{2C27C}\UTFC{2C288}\UTFC{2C2A4}\UTFC{2C317}%
\UTFC{2C35B}\UTFC{2C361}\UTFC{2C364}\UTFC{2C488}\UTFC{2C494}\UTFC{2C497}\UTFC{2C542}\UTFC{2C613}\UTFC{2C618}\UTFC{2C621}%
\UTFC{2C629}\UTFC{2C62B}\UTFC{2C62C}\UTFC{2C62D}\UTFC{2C62F}\UTFC{2C642}\UTFC{2C64A}\UTFC{2C64B}\UTFC{2C72C}\UTFC{2C72F}%
\UTFC{2C79F}\UTFC{2C7C1}\UTFC{2C7FD}\UTFC{2C8D9}\UTFC{2C8DE}\UTFC{2C8E1}\UTFC{2C8F3}\UTFC{2C907}\UTFC{2C90A}\UTFC{2C91D}%
\UTFC{2CA02}\UTFC{2CA0E}\UTFC{2CA7D}\UTFC{2CAA9}\UTFC{2CB29}\UTFC{2CB2D}\UTFC{2CB2E}\UTFC{2CB31}\UTFC{2CB38}\UTFC{2CB39}%
\UTFC{2CB3B}\UTFC{2CB3F}\UTFC{2CB41}\UTFC{2CB4A}\UTFC{2CB4E}\UTFC{2CB5A}\UTFC{2CB5B}\UTFC{2CB64}\UTFC{2CB69}\UTFC{2CB6C}%
\UTFC{2CB6F}\UTFC{2CB73}\UTFC{2CB76}\UTFC{2CB78}\UTFC{2CB7C}\UTFC{2CBB1}\UTFC{2CBBF}\UTFC{2CBC0}\UTFC{2CBCE}\UTFC{2CC56}%
\UTFC{2CC5F}\UTFC{2CCF5}\UTFC{2CCF6}\UTFC{2CCFD}\UTFC{2CCFF}\UTFC{2CD02}\UTFC{2CD03}\UTFC{2CD0A}\UTFC{2CD8B}\UTFC{2CD8D}%
\UTFC{2CD8F}\UTFC{2CD90}\UTFC{2CD9F}\UTFC{2CDA0}\UTFC{2CDA8}\UTFC{2CDAD}\UTFC{2CDAE}\UTFC{2CDD5}\UTFC{2CE18}\UTFC{2CE1A}%
\UTFC{2CE23}\UTFC{2CE26}\UTFC{2CE2A}\UTFC{2CE7C}\UTFC{2CE88}\UTFC{2CE93}

% end


{\bfseries%
[gt/bx]

\input{sp_cns_utf}

%
% This file is generated from the data of UniGB-UTF32
% in cid2code.txt (Version 09/21/2023)
% for Adobe-GB1-6
%
% Reference:
%   https://github.com/adobe-type-tools/cmap-resources/
%   Adobe-GB1-6/cid2code.txt
%
% A newer CMap may be required for some code points.
%


Adobe-GB1-2\\
\UTFC{20087}\UTFC{20089}\UTFC{200CC}\UTFC{215D7}\UTFC{2298F}\UTFC{20509}\UTFC{2099D}\UTFC{241FE}

Adobe-GB1-6\\
\UTFC{20164}\UTFC{20676}\UTFC{20CD0}\UTFC{2139A}\UTFC{21413}\UTFC{235CB}\UTFC{23C97}\UTFC{23C98}\UTFC{23E23}\UTFC{249DB}%
\UTFC{24A7D}\UTFC{24AC9}\UTFC{25532}\UTFC{25562}\UTFC{255A8}\UTFC{25ED7}\UTFC{26221}\UTFC{2648D}\UTFC{26676}\UTFC{2677C}%
\UTFC{26B5C}\UTFC{26C21}\UTFC{27FF9}\UTFC{28408}\UTFC{28678}\UTFC{28695}\UTFC{287E0}\UTFC{28B49}\UTFC{28C47}\UTFC{28C4F}%
\UTFC{28C51}\UTFC{28C54}\UTFC{28E99}\UTFC{29F7E}\UTFC{29F83}\UTFC{29F8C}\UTFC{2A7DD}\UTFC{2A8FB}\UTFC{2A917}\UTFC{2AA30}%
\UTFC{2AA36}\UTFC{2AA58}\UTFC{2AFA2}\UTFC{2B127}\UTFC{2B128}\UTFC{2B137}\UTFC{2B138}\UTFC{2B1ED}\UTFC{2B300}\UTFC{2B363}%
\UTFC{2B36F}\UTFC{2B372}\UTFC{2B37D}\UTFC{2B404}\UTFC{2B410}\UTFC{2B413}\UTFC{2B461}\UTFC{2B4E7}\UTFC{2B4EF}\UTFC{2B4F6}%
\UTFC{2B4F9}\UTFC{2B50D}\UTFC{2B50E}\UTFC{2B536}\UTFC{2B5AE}\UTFC{2B5AF}\UTFC{2B5B3}\UTFC{2B5E7}\UTFC{2B5F4}\UTFC{2B61C}%
\UTFC{2B61D}\UTFC{2B626}\UTFC{2B627}\UTFC{2B628}\UTFC{2B62A}\UTFC{2B62C}\UTFC{2B695}\UTFC{2B696}\UTFC{2B6AD}\UTFC{2B6ED}%
\UTFC{2B7A9}\UTFC{2B7C5}\UTFC{2B7E6}\UTFC{2B7F9}\UTFC{2B7FC}\UTFC{2B806}\UTFC{2B80A}\UTFC{2B81C}\UTFC{2B8B8}\UTFC{2BAC7}%
\UTFC{2BB5F}\UTFC{2BB62}\UTFC{2BB7C}\UTFC{2BB83}\UTFC{2BC1B}\UTFC{2BD77}\UTFC{2BD87}\UTFC{2BDF7}\UTFC{2BE29}\UTFC{2C029}%
\UTFC{2C02A}\UTFC{2C0A9}\UTFC{2C0CA}\UTFC{2C1D5}\UTFC{2C1D9}\UTFC{2C1F9}\UTFC{2C27C}\UTFC{2C288}\UTFC{2C2A4}\UTFC{2C317}%
\UTFC{2C35B}\UTFC{2C361}\UTFC{2C364}\UTFC{2C488}\UTFC{2C494}\UTFC{2C497}\UTFC{2C542}\UTFC{2C613}\UTFC{2C618}\UTFC{2C621}%
\UTFC{2C629}\UTFC{2C62B}\UTFC{2C62C}\UTFC{2C62D}\UTFC{2C62F}\UTFC{2C642}\UTFC{2C64A}\UTFC{2C64B}\UTFC{2C72C}\UTFC{2C72F}%
\UTFC{2C79F}\UTFC{2C7C1}\UTFC{2C7FD}\UTFC{2C8D9}\UTFC{2C8DE}\UTFC{2C8E1}\UTFC{2C8F3}\UTFC{2C907}\UTFC{2C90A}\UTFC{2C91D}%
\UTFC{2CA02}\UTFC{2CA0E}\UTFC{2CA7D}\UTFC{2CAA9}\UTFC{2CB29}\UTFC{2CB2D}\UTFC{2CB2E}\UTFC{2CB31}\UTFC{2CB38}\UTFC{2CB39}%
\UTFC{2CB3B}\UTFC{2CB3F}\UTFC{2CB41}\UTFC{2CB4A}\UTFC{2CB4E}\UTFC{2CB5A}\UTFC{2CB5B}\UTFC{2CB64}\UTFC{2CB69}\UTFC{2CB6C}%
\UTFC{2CB6F}\UTFC{2CB73}\UTFC{2CB76}\UTFC{2CB78}\UTFC{2CB7C}\UTFC{2CBB1}\UTFC{2CBBF}\UTFC{2CBC0}\UTFC{2CBCE}\UTFC{2CC56}%
\UTFC{2CC5F}\UTFC{2CCF5}\UTFC{2CCF6}\UTFC{2CCFD}\UTFC{2CCFF}\UTFC{2CD02}\UTFC{2CD03}\UTFC{2CD0A}\UTFC{2CD8B}\UTFC{2CD8D}%
\UTFC{2CD8F}\UTFC{2CD90}\UTFC{2CD9F}\UTFC{2CDA0}\UTFC{2CDA8}\UTFC{2CDAD}\UTFC{2CDAE}\UTFC{2CDD5}\UTFC{2CE18}\UTFC{2CE1A}%
\UTFC{2CE23}\UTFC{2CE26}\UTFC{2CE2A}\UTFC{2CE7C}\UTFC{2CE88}\UTFC{2CE93}

% end


}}

\end{document}
